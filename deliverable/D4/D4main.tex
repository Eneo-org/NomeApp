\documentclass[11pt]{extarticle}
\usepackage{float}
\usepackage[italian]{babel}
\usepackage[a4paper, left=3cm, right=3cm]{geometry}
\usepackage{titlesec}
\usepackage{hyperref}

% --- Section format ---
\titleformat{\section}
  {\normalfont\LARGE\bfseries}  % Font style: normal + Large + bold
  {\thesection}                 % Section number format
  {6pt}                         % Space between number and title
  {}

\title{\Huge TRENTO PARTECIPA - D4}
\author{\LARGE D'Angiò Enea, Mattarolo Alessandro, Nedeljkovic Ivan}
\date{January 2026}

\begin{document}

\maketitle
\tableofcontents

\newpage

\section{Introduzione}
Il seguente documento funge da \textbf{report finale} del nostro progetto \textbf{\texttt{"TRENTO PARTECIPA"}}. In particolare, riportiamo dei brevi commenti riguardanti l'organizzazione delle attività, i ruoli svolti dai membri del gruppo, la distribuzione delle ore di lavoro e le criticità incontrate. Inoltre, nell'ultima sezione, forniamo una proposta di valutazione.

\section{Organizzazione del lavoro}
Durante la realizzazione del progetto, il nostro gruppo ha variato il modo in cui operava. \\ Inizialmente, quando non avevamo ancora chiarito cosa volessimo ottenere e non si erano ancora ben definite delle competenze individuali e specifiche che ci permettessero di dividerci i compiti con un criterio, abbiamo lavorato tutti e tre contemporaneamente sulle stesse task. \\ Con l'avanzamento del progetto abbiamo invece deciso di suddividerci i lavori in base alle competenze e alle attitudini che i singoli membri hanno sviluppato: a partire da novembre, \textit{D'Angiò Enea} si è concentrato principalmente sull'implementazione del Backend, \textit{Nedeljkovic Ivan} si è preso carico del Frontend e \textit{Mattarolo Alessandro} si è occupato della parte di documentazione.

\paragraph{Incontri di gruppo.} Per discutere del progetto con una certa costanza e collaborazione reciproca, il gruppo si è ritrovato periodicamente dal vivo per lavorare insieme, discutere di criticità, dividersi gli impieghi futuri e occasionalmente discutere con il tutor assegnatoci. Durante il resto del tempo, l'organizzazione è stata svolta tramite una chat di gruppo.

\paragraph{Strumenti utilizzati.} La stesura della documentazione si è svolta in LaTeX tramite un progetto condiviso sull'editor Overleaf. Per quanto riguarda il codice, abbiamo organizzato il progetto su una repository GitHub.
Il calcolo delle ore di lavoro (riportato nella sezione \ref{d4:distribuzione_ore}) è stato tenuto aggiornato in ogni fase della realizzazione segnandole su un foglio di lavoro condiviso su Fogli Google, che si è poi evoluto in una bacheca in cui annotare anche idee e task da svolgere.

\newpage

\section{Ruoli e attività}
Viene qui riportata una tabella che descrive i principali ruoli svolti dai componenti del gruppo.
\newline
\begin{table}[H]
    \centering
    \begin{tabular}{|p{10em}|p{10em}|p{22em}|}
        % riga con nomi delle colonne
        \hline
        \textbf{Componente} & \textbf{Ruolo} & \textbf{Attività} \\
        % riga di enea
        \hline
        \textit{D'Angiò Enea} 
        &
        Progettista-analista/ Sviluppatore Backend
        &
        È il responsabile della progettazione del database, ha svolto la maggior parte della progettazione delle API, si è occupato dello sviluppo del Backend e ha lavorato al deployment. \\
        % riga di ivan
        \hline
        \textit{Nedeljkovic Ivan} 
        &
        Progettista-analista/ Sviluppatore Frontend
        &
        Si è occupato principalmente della stesura degli use case e della realizzazione degli use case diagrams, oltre che allo sviluppo del Frontend, al deployment e alla realizzazione del video finale. \\
        % riga di matta
        \hline
        \textit{Mattarolo Alessandro} 
        &
        Progettista-analista/ Revisionatore
        &
        Ha realizzato le slide per il pitch, le user stories, gli user flow, l'analisi e il diagramma dei componenti, il design del Frontend e il report finale, oltre a diverse revisioni del D1.\\
        \hline
    \end{tabular}
\end{table}
\noindent

\section{Carico e distribuzione del lavoro} \label{d4:distribuzione_ore}
Riportiamo di seguito l'ammontare delle ore di lavoro nei vari deliverable di ciascun componente del gruppo.
\newline
\begin{table}[H]
    \centering
    \begin{tabular}{|c|c|c|c|c|c|}
        % prima riga
        \hline
        \textbf{Componente} & \textbf{D1} & \textbf{D2} 
        & \textbf{D3} & \textbf{D4} & \textbf{TOT} \\
        % enea
        \hline
        \textit{D'Angiò Enea} & 38 & 66 & 2 & 1 & 112 \\
        % ivan
        \hline
        \textit{Nedeljkovic Ivan} & 42 & 64 & 2 & 4 & 112 \\
        % matta
        \hline
        \textit{Mattarolo Alessandro} & 64 & 20 & 20 & 8 & 112 \\
        % tot Dx
        \hline
        \textbf{Totale} & 144 & 150 & 24 & 13 & 336 \\
        \hline
    \end{tabular}
\end{table}
Come si può notare, le ore di lavoro totali di ogni componente sono omogenee, mentre variano i contributi nei diversi deliverable. Come già detto, \textit{D'Angiò Enea} e \textit{Nedeljkovic Ivan} hanno realizzato l'effettiva implementazione dell'applicazione, oltre alla documentazione tecnica, e per questa ragione le loro ore di lavoro nel D2 sono molte di più rispetto a quelle di \textit{Mattarolo Alessandro}, che ha invece dato contributi minori come la sezione "FrontEnd" del documento. Le ore non impiegate da quest'ultimo nel D2 sono state recuperate lavorando maggiormente su D1, D3 e D4. Questa scelta di divisione dei compiti ha avuto l'effetto di semplificare l'organizzazione del lavoro, sia perché in questo modo sono state parallelizzate le task in modo efficiente e sia perché ogni membro ha svolto le attività in cui i suoi sforzi risultavano più redditizi.

\newpage

\section{Criticità}
Il gruppo ha riscontrato alcune criticità durante lo sviluppo del progetto, principalmente riguardanti l'organizzazione del lavoro. Nello specifico, dopo una serie di incontri di gruppo nel mese di ottobre, ci siamo resi conto che c'erano molte incongruenze e aspetti non ben definiti nei requisiti funzionali, i quali ostacolavano la continuazione della progettazione. È stato quindi necessario fare un'ampia azione di revisione del D1 per correggere i problemi riscontrati. Nonostante la correzione non abbia richiesto un numero eccessivo di ore, ha sicuramente rallentato notevolmente la prima fase di lavoro. Una volta risolto questo ostacolo ci siamo dovuti riorganizzare per recuperare il tempo perso, dividendoci i restanti compiti in base alle nostre competenze e attitudini: in particolare, mentre \textit{D'Angiò Enea} e \textit{Nedeljkovic Ivan} si sono presi carico dell'implementazione concreta del codice, \textit{Mattarolo Alessandro} ha ultimato la fase di progettazione e analisi. Da questo momento in poi il progetto ha ricominciato ad avanzare regolarmente. \\ Nonostante le difficoltà incontrate, i membri del gruppo hanno sempre lavorato in sintonia e ognuno ha contribuito in modo significativo al completamento del progetto: come si può vedere dal monte ore finale, sebbene i membri non abbiano lavorato in maniera equa ai \underline{singoli} deliverable il lavoro è stato comunque equamente distribuito \underline{nel complesso}.\\ È inoltre importante riportare che ogni membro, singolarmente o in gruppo, ha svolto azioni di revisione sui lavori altrui, in modo da rilevare errori, perfezionare alcuni aspetti e garantire un buon risultato finale.

\newpage

\section{Autovalutazione}
Come già specificato in precedenza, il lavoro è stato spartito in maniera equa tra i componenti del gruppo, sebbene il lavoro sui singoli deliverable possa sembrare sbilanciato. Ogni membro ha avuto un ruolo determinante nella definizione, realizzazione e ultimazione del progetto. \textit{D'Angiò Enea}, responsabile della direzione del gruppo, si è occupato di parti molto importanti del progetto quali la maggior parte dell'implementazione della logica di Backend, contributi significativi nel D1 e buona parte della documentazione scritta nel D2. Anche \textit{Nedeljkovic Ivan} ha avuto un ruolo rilevante: è responsabile degli use case, dell'implementazione del Frontend e di alcune sezioni del D2. La partecipazione di \textit{Mattarolo Alessandro} è stata invece determinante per la stesura di buona parte della documentazione: ha revisionato e completato il D1 ed è il principale responsabile di D3 e D4. Sulla base dell'impegno e del tempo che abbiamo riposto in questo progetto e sul risultato complessivo ottenuto, riteniamo quindi ragionevole la seguente valutazione:

\begin{table}[H]
    \centering
    \begin{tabular}{|c|c|c|}
        \hline
        \textbf{Componente} & 
        \textbf{Matricola} &
        \textbf{Voto}  \\
        \hline
        \textit{D'Angiò Enea} & 254088 & 30 \\
        \hline
        \textit{Nedeljkovic Ivan} & 255329 & 30 \\
        \hline
        \textit{Mattarolo Alessandro} & 254087 & 30 \\
        \hline
    \end{tabular}
\end{table}

\end{document}
