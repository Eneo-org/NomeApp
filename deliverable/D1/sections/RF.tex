\section{Requisiti Funzionali}

\subsection{Requisiti funzionali degli utenti}

\RF{Accesso utente (Login).} \label{rf:login}
Il sistema deve consentire l’accesso agli utenti esclusivamente tramite i sistemi di identità digitale riconosciuti dallo Stato (SPID o CIE). L'autenticazione tramite SPID/CIE è richiesta per ogni sessione di un utente già profilato. 

\RF{Creazione profilo al primo accesso.} \label{rf:creazione_profilo}
Il sistema gestisce il primo accesso di un utente non ancora registrato. Al momento dell'autenticazione tramite SPID/CIE, il sistema recupera il Codice Fiscale dell'utente e procede alla verifica dei requisiti per la creazione del profilo, distinguendo due casistiche:

\subRF{Registrazione cittadini residenti.} \label{subRf:reg_cittadini}
Il sistema verifica automaticamente se l'utente risiede nel Comune di Trento, analizzando i dati forniti dall'identity provider. Se la verifica è positiva, il sistema richiede all'utente l'inserimento di un indirizzo e-mail valido. Una volta verificato l'indirizzo con l'inserimento di un codice OTP, viene creato un nuovo profilo per l'utente, a cui viene assegnato il \textbf{ruolo di "Cittadino"}.
Questo ruolo abilita l'utente a tutte le funzionalità partecipative descritte nella Sezione \ref{sec:RF_cittadini}.

\subRF{Registrazione amministratori non residenti.} \label{subRf:reg_admin}
Se l'utente non risulta residente a Trento, il sistema verifica se il suo Codice Fiscale è presente nella lista di amministratori pre-autorizzati (inseriti da un altro amministratore come definito in RF\ref{subRf:assegnazione_privilegi}).
\begin{itemize}
    \item \textbf{Esito positivo:} Se il Codice Fiscale è presente nella lista, il sistema richiede l'inserimento di un'e-mail e crea un profilo con il solo \textbf{ruolo di "Amministratore"}. Questo utente avrà accesso agli strumenti di gestione amministrativa (Sezione \ref{sec:RF_amministrazione}) ma non alle funzionalità partecipative riservate ai cittadini (Sezione \ref{sec:RF_cittadini}).
    \item \textbf{Esito negativo:} Se il Codice Fiscale non è presente nella lista dei pre-autorizzati, il sistema nega la registrazione e mostra un messaggio di errore specificando che l'accesso alla piattaforma è riservato ai residenti del Comune di Trento (questo poiché l'utente non è né residente a Trento, né un amministratore).
\end{itemize}


\RF{Consultazione della lista di iniziative.}  \label{rf:consultazione_iniziative}
Il sistema deve consentire a tutti gli utenti, inclusi i non registrati, di consultare un catalogo pubblico che raccoglie le preview delle iniziative presenti sul sito. Ogni iniziativa deve mostrare le sue informazioni di base: titolo, data di creazione, categoria, stato attuale, luogo, piattaforma di provenienza, numero di firme raccolte ed eventualmente un'immagine se è stata caricata. L'applicativo permette di visualizzare sia iniziative create attraverso l'applicativo stesso, sia iniziative presenti su altre piattaforme, sempre riguardanti il Comune di Trento. Queste ultime saranno raggiungibili grazie a un link di reindirizzamento. \\Gli utenti devono avere a disposizione strumenti di filtraggio (in base a piattaforma di provenienza, categoria, stato e periodo di creazione) e ordinamento (in base a numero di firme raccolte e data di creazione), oltre a una funzione di ricerca libera per parole chiave.


\RF{Consultazione di una singola iniziativa.}
\label{rf:consultazione_singola_iniz}
Il sistema deve permettere a tutti gli utenti, inclusi i non registrati, di accedere alla pagina di dettaglio di una singola iniziativa (cliccando su di essa dalla lista definita in RF\ref{rf:consultazione_iniziative}, o dalla Home). La pagina di dettaglio deve mostrare tutte le informazioni associate all'iniziativa, organizzate in modo chiaro. In particolare:
\begin{itemize}
    \item \textbf{Dati della proposta:} titolo, descrizione dettagliata, autore (se l'iniziativa è stata creata in piattaforma), categoria di appartenzenza (ad esempio "Ambiente", "Sport", etc.), luogo (se specificato) ed eventuali allegati (immagini o PDF) caricati dall'autore.
    \item \textbf{Dati sullo stato:} stato attuale del "ciclo di vita" dell'iniziativa (si veda RF\ref{rf:ciclo_vita_stati}), numero totale di firme raccolte (aggiornato in tempo reale se l'iniziativa è interna), data di scadenza (per le iniziative non importate nello stato "In corso"), piattaforma di provenienza (se l'iniziativa è importata).
    \item \textbf{Dati di risposta:} eventuali note ufficiali e documenti di risposta pubblicati dall'amministrazione.
\end{itemize}


\subsection{Requisiti funzionali per gli utenti autenticati}
\RF{Logout.} \label{rf:logout}
L'utente che ha fatto l'accesso al proprio account deve potersi anche disconnettere.

\subsection{Requisiti funzionali per l’amministrazione}
\label{sec:RF_amministrazione}

\RF{Gestione iniziative.}  \label{rf:gestione_iniziative}
L'amministratore deve poter gestire le iniziative per conto del Comune. In particolare:
\begin{itemize}
    \item L'amministratore deve poter \textbf{prorogare la data di scadenza} di un'iniziativa di 60 giorni in casi particolari.
    \item L'amministratore deve poter \textbf{rispondere} ad un'iniziativa: la risposta consiste in un cambiamento di stato (da "In corso" ad "Approvata" oppure "Respinta") e nella pubblicazione di una motivazione scritta per la scelta fatta (eventualmente allegando anche dei file).
\end{itemize} 
Una volta pubblicata, la risposta risulta visibile agli altri utenti e l'iniziativa non può più essere firmata.

\RF{Creazione bilancio partecipativo.}
\label{rf:creazione_bilancio_partecipativo}
Il sistema deve permettere al personale dell'amministrazione di creare e pubblicare un nuovo sondaggio di bilancio partecipativo. L'interfaccia di creazione richiederà all'amministratore di definire una domanda e un numero di possibili risposte compreso tra un minimo di 3 e un massimo di 5. L'amministratore dovrà inoltre impostare obbligatoriamente una data di scadenza per la votazione, assicurando il periodo di apertura non sia inferiore a 14 giorni. Il sistema deve impedire la pubblicazione di un nuovo bilancio qualora un altro risultasse già attivo, garantendo che non possano coesistere più bilanci partecipativi attivi simultaneamente.

\RF{Consultazione archivio bilanci partecipativi.}
\label{rf:consultazione_archivio_bilancio_partecipativo}
Il sistema deve fornire al personale dell'Amministrazione una sezione dedicata alla consultazione dell'archivio storico di tutti i bilanci partecipativi conclusi. Tale interfaccia deve essere accessibile esclusivamente agli utenti "Amministratore". Per ogni bilancio archiviato, il sistema deve mostrare in modo chiaro i dettagli della consultazione passata, includendo la domanda che era stata proposta, l'elenco delle opzioni disponibili e la data di scadenza. La vista dovrà riportare anche i risultati finali, specificando il numero totale di votanti e la distribuzione dei voti, sia in numero assoluto che in percentuale, per ciascuna opzione.

\RF{Gestione ruoli amministrativi.} \label{rf:ruoli_amministrativi}
Il sistema deve fornire agli utenti amministratori un'interfaccia dedicata per la gestione dei privilegi amministrativi della piattaforma.

\subRF{Visualizzazione e ricerca utenti.} \label{subRf:ricerca_utenti}
Il sistema deve permettere a un amministratore di visualizzare un elenco completo di tutti gli utenti che attualmente possiedono il ruolo di "Amministratore" ed utilizzare una funzione di ricerca tramite Codice Fiscale per verificare la presenza di un utente nell'elenco.

\subRF{Assegnazione privilegi e pre-autorizzazione.} \label{subRf:assegnazione_privilegi}
Il sistema deve fornire una funzione di ricerca tramite Codice Fiscale (separata da quella descritta in \ref{subRf:ricerca_utenti}) per identificare un utente da promuovere o pre-autorizzare. Una volta inserito un Codice Fiscale, il sistema verifica l'esistenza dell'utente nel database:
\begin{itemize}
    \item Se il Codice Fiscale inserito \textbf{corrisponde a un utente esistente, con ruolo di "Cittadino"}, l'amministratore può \textbf{promuoverlo}, assegnandogli il ruolo aggiuntivo di "Amministratore".
    \item Se il Codice Fiscale inserito \textbf{non corrisponde a un utente già registrato (presumibilmente non si tratta di un "Cittadino")}, l'amministratore può \textbf{pre-autorizzarlo} inserendolo in un'apposita lista. Questo Codice Fiscale sarà utilizzato per la successiva registrazione dell'amministratore non residente (come definito in RF\ref{subRf:reg_admin}).
\end{itemize}
L'utente promosso (o l'utente pre-autorizzato che effettua il primo accesso) ottiene immediatamente accesso a tutte le funzionalità riservate agli Amministratori (definite nella sezione: \ref{sec:RF_amministrazione}).

\subRF{Revoca privilegi.} \label{subRf:revoca_previlegi}
L'Amministratore deve poter rimuovere un utente dall'elenco degli amministratori per revocargli il ruolo di "Amministratore".
\begin{itemize}
    \item Se l'utente era anche "Cittadino", manterrà solo quel ruolo con i relativi privilegi (sezione \ref{sec:RF_cittadini}).
    \item Se l'utente era solo "Amministratore" (non residente), questo perderà l'accesso al sistema (poiché non soddisfa più i requisiti di RF\ref{subRf:reg_admin}).
    \item L'Amministratore deve poter anche annullare una pre-autorizzazione prima che l'utente non residente effettui il primo accesso.
\end{itemize}
Per ragioni di sicurezza, un amministratore non può revocare i privilegi a sé stesso.

\subRF{Bootstrapping amministratore.}
    \label{subRf:bootstrapping_amministratore}
L'esistenza del primo utente Amministratore ("super-utente") è garantita dalla configurazione manuale nel database al momento dell'installazione della piattaforma. Questo utente sarà poi il primo a nominare gli altri amministratori con le modalità descritte in RF\ref{subRf:assegnazione_privilegi}.

\subsection{Requisiti funzionali per i cittadini autenticati}
\label{sec:RF_cittadini}

\RF{Firma.}  \label{rf:firma}
Gli utenti autenticati con il ruolo di "Cittadino" devono poter sostenere le iniziative già pubblicate tramite un sistema di raccolta firme.
\begin{description}
    \item[Iniziative create nella piattaforma] Una volta espresso il supporto, il sistema deve aggiornare in tempo reale il numero totale di adesioni e rendere visibile il contributo dell’utente nella sua area personale. Per prevenire abusi, ogni cittadino può sostenere una specifica iniziativa una sola volta.
    \item[Iniziative importate] Tramite un link, l'utente può essere indirizzato alla pagina riguardante l'iniziativa a cui vuole partecipare. La partecipazione all'iniziativa viene demandata alla piattaforma esterna. Il nostro sistema deve essere in grado di monitorare il numero totale di adesioni alle iniziative importate.
\end{description}

\RF{Dashboard personale.}  \label{rf:dashboard_personale}
Ogni cittadino autenticato deve avere a disposizione una dashboard personale che riepiloghi le sue attività: iniziative create, iniziative firmate (esclusivamente quelle create nel nostro sito) e iniziative seguite (RF\ref{rf:tracciamento_stato}). La dashboard deve includere anche una sezione per le notifiche (RF\ref{rf:notifiche}).

\RF{Creazione di una iniziativa.}  \label{rf:creazione_iniziativa}
Il sistema deve permettere agli utenti con ruolo "Cittadino" di proporre una nuova iniziativa, specificandone il titolo, il luogo (se possibile), una descrizione ed una categoria di appartenenza. L'utente potrà inoltre aggiungere eventuali allegati (come immagini o PDF). Una volta pubblicata un'iniziativa, l'utente dovrà attendere un periodo di cool-down di 14 giorni prima di poterne creare un'altra.
Prima della pubblicazione viene effettuato un controllo dei duplicati (RF \ref{rf:controllo_duplicati}): se viene superato, l'iniziativa viene registrata, resa visibile alla comunità e le viene impostata una data di scadenza automatica di 60 giorni.

\RF{Tracciamento stato.}  \label{rf:tracciamento_stato}
Ogni cittadino autenticato deve poter seguire lo stato di avanzamento di un'iniziativa aggiungendola alla propria dashboard personale (RF\ref{rf:dashboard_personale}) nella sezione delle iniziative seguite.

\RF{Votazione del bilancio partecipativo.}
\label{rf:voto}
Ogni utente deve essere in grado di visualizzare i sondaggi di bilancio partecipativo proposti dal Comune. Se è un cittadino autenticato, deve essere anche in grado di votare una delle opzioni. L'utente può votare una sola opzione, senza possibilità di cambiare scelta dopo il voto. Dopo che l'utente ha votato, il numero di voti viene aggiornato in tempo reale. Al termine del periodo di votazioni, l'utente firmatario riceverà una notifica inerente i risultati del bilancio partecipativo a cui ha preso parte (RF\ref{rf:notifiche}). 

\subsection{Requisiti funzionali del sistema}

\RF{Controllo duplicati.} 
\label{rf:controllo_duplicati}
Prima della pubblicazione (RF\ref{rf:creazione_iniziativa}), ogni iniziativa deve superare un controllo anti-duplicati.
Il sistema confronta la proposta con quelle già esistenti tramite algoritmi di similarità testuale, per evitare petizioni duplicate o troppo simili che rischierebbero di sovraccaricare il server e frammentare i contributi dei firmatari. Se viene rilevata una similarità elevata, il sistema impedisce la pubblicazione e lo notifica all'utente.


\RF{Importazione dati esterni.}  \label{rf:import_dati_esterni}
La piattaforma deve permettere l’integrazione automatica con fonti esterne (come ParteciPa e Change.org). Vengono estratti i dati relativi all'iniziativa (gli stessi che vengono mostrati in RF\ref{rf:consultazione_singola_iniz}). Le iniziative importate devono essere chiaramente etichettate come provenienti da fonti esterne, per distinguerle dalle petizioni create direttamente in piattaforma. 

\RF{Aggiornamento iniziative esterne.} \label{rf:aggiornamento_dati_esterni}
Il sistema deve garantire che i dati delle iniziative importate (RF\ref{rf:import_dati_esterni}) vengano mantenuti aggiornati attraverso due modalità:
    \subRF{Aggiornamento standard.} Il sistema deve eseguire un controllo e un aggiornamento (batch) \textbf{giornaliero} per tutte le iniziative esterne presenti nel database, al fine di recepire nuovi stati o variazioni nel numero di firme.
    
    \subRF{Aggiornamento prioritario.} Se un utente compie un'interazione significativa con un'iniziativa esterna (ad esempio, aggiungendola alla propria dashboard per il tracciamento, come da RF\ref{rf:tracciamento_stato}), il sistema deve "monitorare" tale iniziativa. Le iniziative monitorate devono essere aggiornate con una \textbf{frequenza maggiore} rispetto all'aggiornamento standard (ad esempio a intervalli di poche ore), per riflettere tempestivamente i cambiamenti rilevanti per l'utente.


\RF{Definizione e ciclo di vita degli stati.} \label{rf:ciclo_vita_stati}
Ogni iniziativa, che sia stata creata sulla piattaforma o importata da fonti esterne, deve essere associata a uno stato che ne definisca la fase corrente nel ciclo di vita. Gli stati gestiti dal sistema sono:
\begin{itemize}
    \item \textbf{In corso:} Lo stato predefinito al momento della creazione, durante il quale l’iniziativa raccoglie le firme. Questo stato ha una durata naturale di 60 giorni a partire dalla data di pubblicazione (per le iniziative non importate). Durante questa fase, il Comune si impegna a fornire una risposta prima della scadenza (per le iniziative della piattaforma).
    \item \textbf{Approvata:} L'amministrazione ha preso in carico la richiesta.
    \item \textbf{Respinta:} L'amministrazione ha respinto la richiesta.
    \item \textbf{Scaduta:} Se l'iniziativa è stata creata nella nostra piattaforma ed ha raggiunto la data di scadenza senza ottenere alcuna risposta, oppure se l'iniziativa importata risulta scaduta nel sito originale.
\end{itemize}
Per particolari esigenze dell'Amministrazione, i termini di risposta possono essere prorogati. Le transizioni di stato sono gestite dagli amministratori (RF\ref{rf:gestione_iniziative}) o automaticamente dal sistema (ad esempio per scadenza).

\RF{Sistema di notifiche.} \label{rf:notifiche}
Il sistema deve inviare notifiche agli utenti nei seguenti casi:
\begin{itemize}
    \item Quando un'iniziativa creata o seguita dall'utente subisce un cambiamento di stato (ad esempio da "In corso" ad "Approvata");
    \item Quando termina un bilancio partecipativo in cui l'utente aveva votato (in tal caso la notifica informa l'utente dell'esito finale del bilancio). 
\end{itemize}
Le notifiche devono essere inviate sia all’interno della piattaforma (visualizzabili nella dashboard personale - RF\ref{rf:dashboard_personale}) sia tramite e-mail all'indirizzo definito durante la creazione del profilo (RF\ref{rf:creazione_profilo}).

\newpage
\titleformat{\paragraph}[block]{\normalfont\normalsize\bfseries}{\theparagraph}{1em}{}
\titlespacing*{\paragraph}{0em}{3.25ex plus 1ex minus .2ex}{0em}