\section{Requisiti Non Funzionali}

\RNF{Compatibilità.}  \label{rnf:compatibilita}
Il sistema deve essere pienamente utilizzabile sia da desktop che da dispositivi mobili (smartphone, tablet), con un design responsivo che adatti automaticamente le interfacce alle diverse dimensioni dello schermo. Deve essere garantito il supporto ai principali browser: Chrome (versione 80 o superiore), Firefox (versione 75 o superiore), Safari (versione 13 o superiore), Microsoft Edge (versione 85 o superiore).
Il design responsivo deve garantire la corretta fruizione dei contenuti su viewport con larghezza minima di 320px (mobile) fino a 1920px (desktop).

\RNF{Prestazioni.}  \label{rnf:prestazioni}
Il sistema deve garantire tempi di risposta rapidi per le operazioni critiche (login, caricamento iniziative, votazione), che devono essere completate entro un massimo di 2 secondi per il 95\% delle richieste. Tale livello di prestazione deve essere mantenuto anche in situazioni di carico elevato, garantendo la stabilità del sistema con almeno 100 utenti connessi simultaneamente 

\RNF{Scalabilità.}  \label{rnf:scalabilità}
L’architettura deve essere progettata per supportare un aumento progressivo del numero di utenti. Il sistema deve scalare fino a 20.000 utenti simultanei, mantenendo tempi di risposta conformi al requisito RNF\ref{rnf:prestazioni}. Devono essere previste politiche di bilanciamento del carico e possibilità di aggiungere risorse hardware senza modificare la logica applicativa.

\RNF{Affidabilità e disponibilità.}  \label{rnf:aff_e_disponibilita}
La piattaforma deve garantire un’affidabilità elevata, con una disponibilità minima del 99.5\% annuo (pari a circa 1 giorno e 20 ore di downtime massimo). Devono essere implementati sistemi di backup automatico giornaliero e procedure di Disaster Recovery con un RTO (Recovery Time Objective) inferiore a 2 ore e un RPO (Recovery Point Objective) massimo di 24 ore.

\RNF{Sicurezza.}  \label{rnf:sicurezza}
Il sistema deve rispettare le normative GDPR per la protezione dei dati personali. Le comunicazioni tra client e server devono essere cifrate tramite il protocollo TLS 1.2 o superiore. I dati sensibili a riposo nel database (Data at Rest) devono essere protetti tramite crittografia standard (es. AES-256). L'accesso ai dati di backend deve essere regolato da un sistema RBAC (Role-Based Access Control) che garantisca il principio del privilegio minimo. Le sessioni utente devono scadere automaticamente dopo un periodo di inattività configurabile. L’integrazione con SPID deve rispettare i protocolli di sicurezza stabiliti a livello nazionale. Il sistema deve assicurarsi che gli utenti non autorizzati non possano accedere tramite backend a dati a loro non accessibili. 

\RNF{Usabilità.} \label{rnf:usabilita} 
L’interfaccia deve essere intuitiva e utilizzabile da un cittadino medio senza necessità di formazione. Per un utente non autenticato (con accesso solo alle funzionalità base) il design deve permettere la comprensione delle feature entro 10 minuti. Per un utente autenticato, invece, deve essere possibile comprendere le principali funzionalità (creare un'iniziativa, votare a un bilancio partecipativo, firmare) entro un tempo massimo di 30 minuti di utilizzo autonomo. Riguardo all'utente amministratore, il sistema deve richiedere una curva di apprendimento superiore a 1 ora di formazione per la padronanza delle funzionalità amministrative (creazione bilancio partecipativo, risposta delle iniziative, ecc.). Devono essere adottate linee guida di design coerenti (colori, pulsanti, icone) per garantire un’esperienza uniforme.  
\RNF{Accessibilità.}  \label{rnf:accessibilita}
Il sistema deve essere conforme alle linee guida WCAG 2.1 livello AA, garantendo l’accesso anche a utenti con disabilità. Ciò include il supporto per screen reader, testi alternativi per immagini, contrasto elevato, navigazione da tastiera e font leggibili.  

\RNF{Manutenibilità.}  \label{rnf:manutenibilita}
Il software deve essere sviluppato seguendo principi di modularità e documentazione del codice e delle API (es. standard OpenAPI/Swagger) per facilitare l'integrazione e la manutenibilità futura, in modo da permettere futuri aggiornamenti e correzioni senza compromettere la stabilità del sistema. Devono essere previste procedure di test automatici per ridurre il rischio di regressioni.