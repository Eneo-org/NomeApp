\section{User Stories}

\textbf{Premessa:} in fase di implementazione alcune funzioni potrebbero venire semplificate. Queste semplificazioni, e in generale eventuali differenze tra progettazione e implementazione, verranno chiarite nel documento in cui tratteremo l'implementazione (D2).

\US{Accesso utente (Login)} \label{us:login}
Come \textit{utente già registrato} (cittadino o operatore del Comune) voglio poter accedere al mio profilo in modo da usufruire delle funzionalità a me riservate. \\ (Riferito a RF\ref{rf:login})
\paragraph{Criteri di accettazione:}
\begin{itemize}
    \item L'utente può visualizzare le opzioni d'accesso permesse e scegliere quale utilizzare dopo aver premuto sull'apposito pulsante.
    \item L'utente viene reindirizzato alla schermata di login anche quando prova a compiere un'azione da utente registrato come firmare, creare o seguire un'iniziativa.
    \item Selezionando un'opzione, il sistema reindirizza l'utente al provider scelto.
    \item Autenticandosi con la modalità scelta, vengono restituiti al sistema i dati identificativi dell'utente.
    \item Il sistema verifica l'esistenza dell'utente nel database. 
    \item Dopo un login riuscito, il sistema reindirizza l'utente alla home page.
    \item Se loggato, l'utente è abilitato a tutte le funzionalità a lui permesse.
    \item Se l'utente non risulta presente nel database del sistema, gli viene mostrato un messaggio informativo e viene guidato al flusso di creazione del profilo.
    \item In caso di errore di autenticazione viene mostrato un messaggio d'errore.
\end{itemize}
\paragraph{Tasks:}
\begin{itemize}
    \item Aggiungere alla UI un tasto per il login che conduca alla schermata dedicata con le opzioni d'accesso disponibili.
    \item Implementare il reindirizzamento alla schermata di login anche nel caso in cui un utente non autenticato provi a compiere azioni a lui non permesse.
    \item Implementare il reindirizzamento verso i provider e gestire il callback di ritorno.
    \item Testare l'accesso per gli utenti registrati (partendo ad esempio dal "super-utente" - RF\ref{subRf:bootstrapping_amministratore}), per assicurarsi che dopo l'autenticazione vengano reindirizzati alla home modificata in base ai loro privilegi.
    \item Testare inoltre che dopo il login gli utenti possano svolgere le attività che caratterizzano il loro ruolo.
    \item Testare il reindirizzamento alla fase di creazione del profilo in caso di account inesistente.
    \item Implementare e verificare i messaggi d'errore e informativi nei casi descritti.
\end{itemize}

\US{Creazione profilo al primo accesso} \label{us:profilo}
Come \textit{utente non ancora registrato} (cittadino o operatore del Comune) voglio poter creare un profilo per usufruire dei servizi dell'applicazione e svolgere il mio ruolo. \\ (Riferito a RF\ref{rf:creazione_profilo})
\paragraph{Criteri di accettazione:}
\begin{itemize}
    \item L'utente viene guidato al flusso di creazione del profilo quando non vengono trovati i suoi dati, forniti in fase di autenticazione, nel database.
    \item Il sistema verifica che l'utente sia residente nel Comune di Trento o sia presente nella lista di utenti pre-autorizzati (RF\ref{subRf:assegnazione_privilegi}). 
    \item Se la verifica va a buon fine, viene richiesto all'utente di fornire un indirizzo e-mail, che verrà integrato nel database.
    \item Se l'e-mail inserita è valida (ossia viene verificata tramite inserimento del codice OTP inviato all'indirizzo specificato entro lo scadere di un timer), viene creato il profilo dell'utente.
    \item A ogni nuovo accesso, il profilo dell'utente risulta presente nel database.
    \item Se la verifica di residenza o pre-autorizzazione non va a buon fine viene mostrato un messaggio d'errore.
    \item Se il codice OTP inserito non è corretto viene mostrato un messaggio d'errore.
    \item Se il codice OTP non viene inserito entro lo scadere del timer, vengono mostrati un messaggio d'errore e un pulsante per inviare un nuovo codice. L'utente non può digitare il codice dopo lo scadere del timer.
    \item L'utente può ritornare alla schermata di inserimento della mail e inserirne una nuova, in caso si fosse sbagliato in precedenza.
\end{itemize}
\paragraph{Tasks:}
\begin{itemize}
    \item Implementare il controllo di residenza o pre-autorizzazione fatto dal sistema.
    \item Implementare la richiesta dell'e-mail.
    \item Implementare la verifica dell'e-mail tramite codice OTP.
    \item Aggiungere un timer di validità del codice OTP.
    \item Fare in modo che i dati forniti in fase di autenticazione e la mail costituiscano una nuova tupla nel database.
    \item Verificare che il profilo risulti esistente nei successivi accessi dell'utente.
    \item Implementare e verificare i messaggi d'errore nelle casistiche descritte.
    \item Aggiungere un pulsante per mandare un nuovo codice OTP.
    \item Aggiungere un pulsante per tornare alla schermata per inserire una nuova e-mail.
\end{itemize}

\US{Visualizzazione della lista di iniziative} \label{us:listainiz}
Come \textit{utente} voglio poter visualizzare le varie iniziative presenti sull'applicazione, per farmi un'idea sulle problematiche del Comune di Trento. \\ (Riferito a RF\ref{rf:consultazione_iniziative})
\paragraph{Criteri di accettazione:}
\begin{itemize}
    \item L'utente riesce a visualizzare il catalogo pubblico di iniziative pubblicate, contenente le varie iniziative con i relativi dati.
\end{itemize}
\paragraph{Tasks:}
\begin{itemize}
    \item Aggiungere una pagina nella UI in cui siano visibili le varie iniziative con i relativi dati.
\end{itemize}

\US{Ricerca di iniziative} \label{us:ricerca}
Come \textit{utente} voglio poter cercare un'iniziativa inserendo delle parole chiave in una barra di ricerca, in modo da trovarla più facilmente. \\ (Riferito a RF\ref{rf:consultazione_iniziative})
\paragraph{Criteri di accettazione:}
\begin{itemize}
    \item L'utente può inserire parole chiave in una barra di ricerca.
    \item Dopo che l'utente ha inserito le parole chiave e premuto un apposito pulsante, la pagina viene aggiornata e vengono mostrate all'utente solo le iniziative correlate a quelle parole chiave.
    \item Se la ricerca non riconduce a nessuna iniziativa, viene mostrato un messaggio informativo.
\end{itemize}
\paragraph{Tasks:}
\begin{itemize}
    \item Aggiungere una barra di ricerca.
    \item Implementare la selezione delle iniziative da mostrare sulla base delle parole chiave.
    \item Verificare che dopo aver inserito le parole e premuto il pulsante di ricerca venga aggiornata la pagina e vengano mostrate le iniziative selezionate.
    \item Implementare e verificare il messaggio informativo in caso di mancanza di corrispondenze nella ricerca.
\end{itemize}

\US{Filtraggio di iniziative} \label{us:filtri}
Come \textit{utente} voglio applicare dei filtri alla mia ricerca di iniziative per affinarne i risultati. \\ (Riferito a RF\ref{rf:consultazione_iniziative})
\paragraph{Criteri di accettazione}
\begin{itemize}
    \item L'utente può applicare uno o più filtri alla sua ricerca.
    \item Dopo che l'utente ha selezionato i filtri, la pagina viene aggiornata e vengono mostrate le iniziative che soddisfano i criteri di ricerca.
    \item Se l'utente esegue una ricerca testuale, questa viene svolta sul sottoinsieme di iniziative che rispettano i filtri. Allo stesso modo, se vengono applicati dei filtri dopo una ricerca testuale, i criteri di ricerca si combinano.
    \item Se i filtri selezionati non riconducono a nessuna iniziativa, viene mostrato un messaggio informativo.
\end{itemize}
\textit{Nota:} chiaramente, l'utente può anche rimuovere filtri applicati precedentemente, con effetti analoghi.
\paragraph{Tasks:}
\begin{itemize}
    \item Aggiungere un'interfaccia di applicazione di filtri sulla pagina di visualizzazione delle iniziative.
    \item Implementare l'applicazione delle condizioni di filtraggio nella selezione delle iniziative da mostrare.
    \item Verificare che ricerca testuale e selezione dei filtri si combinino per mostrare il giusto sottoinsieme di iniziative.
    \item Implementare e verificare il messaggio informativo in caso di mancanza di corrispondenze nella ricerca.
\end{itemize}

\US{Consultazione di una singola iniziativa} \label{us:iniziativa}
Come \textit{utente} voglio poter visualizzare tutti i dettagli relativi a una specifica iniziativa. \\ (Riferito a RF\ref{rf:consultazione_singola_iniz})
\paragraph{Criteri di accettazione:}
\begin{itemize}
    \item L'utente, cliccando su un'iniziativa nella lista, viene reindirizzato a una pagina di dettaglio con tutte le informazioni associate ad essa (elencate in RF\ref{rf:consultazione_singola_iniz}). 
\end{itemize}
\paragraph{Tasks:}
\begin{itemize}
    \item Creare la pagina di dettaglio dell'iniziativa e implementarne il reindirizzamento quando l'utente clicca sulla preview dell'iniziativa nella lista.
\end{itemize}

\US{Aggiornamento dello stato di un'iniziativa (Risposta)} \label{us:risposta}
Come \textit{operatore del Comune} devo poter aggiornare lo stato di un'iniziativa, in modo da far vedere ai cittadini se il Comune l'ha esplicitamente respinta o accettata. \\ Inoltre, devo poter motivare la scelta allegando note e documenti di risposta, in modo che le motivazioni del Comune siano trasparenti ai cittadini. \\ (Riferito a RF\ref{rf:gestione_iniziative})
\paragraph{Criteri di accettazione:}
\begin{itemize}
    \item L'operatore, cliccando su un pulsante nella sua interfaccia di visualizzazione delle iniziative, deve poter compilare una risposta ad un'iniziativa, decidendo se approvarla o respingerla.
    \item Quando cambia lo stato ad un'iniziativa, l'operatore deve inserire delle note e allegare eventualmente dei documenti.
    \item Una volta premuto un pulsante di conferma, le modifiche apportate si riflettono sul database.
    \item Se l'amministratore conferma la pubblicazione senza aver compilato i campi obbligatori, viene mostrato un messaggio d'errore.
    \item Dopo le modifiche, tutti gli utenti possono vedere il nuovo stato dell'iniziativa e visualizzare le note / gli allegati che l'operatore ha caricato. 
    \item Dopo la risposta, gli utenti non possono firmare / seguire l'iniziativa.
\end{itemize}
\paragraph{Tasks:}
\begin{itemize}
    \item Aggiungere all'interfaccia dell'operatore un pulsante per rispondere a un'iniziativa che conduca a un box dove possa selezionarne il nuovo stato, digitare una nota di risposta e allegare documenti.
    \item Aggiungere un pulsante di conferma per applicare la modifica.
    \item Implementare l'aggiornamento del database.
    \item Implementare il messaggio d'errore.
    \item Testare che gli utenti visualizzino il cambiamento di stato e i caricamenti dell'operatore.
    \item Verificare che le nuove interazioni con l'iniziativa siano coerenti con il nuovo stato.
\end{itemize}

\US{Proroga data di scadenza} \label{us:proroga}
Come \textit{operatore del Comune} devo poter prorogare la data di scadenza di un'iniziativa in corso, nel caso il Comune necessiti di più tempo per decidere se accettarla o respingerla. \\ (Riferito a RF\ref{rf:gestione_iniziative})
\paragraph{Criteri di accettazione}
\begin{itemize}
    \item L'operatore, premendo un pulsante sulla sua interfaccia di visualizzazione di un'iniziativa, può prorogarne la data di scadenza di 60 giorni. 
    \item Una volta premuto un pulsante di conferma, le modifiche apportate si riflettono sul database.
    \item Dopo le modifiche, gli utenti visualizzano una nuova data di scadenza per quell'iniziativa.
\end{itemize}
\paragraph{Tasks:}
\begin{itemize}
    \item Aggiungere all'interfaccia di visualizzazione dell'iniziativa lato operatore un pulsante per prorogarne la data di scadenza.
    \item Implementare l'aggiornamento del database dopo la conferma delle modifiche.
    \item Testare che gli utenti visualizzino la nuova data di scadenza dopo l'aggiornamento.
\end{itemize}

\US{Creazione bilancio partecipativo} \label{us:creabp}
Come \textit{operatore del Comune} devo poter creare un sondaggio per il bilancio partecipativo per conto del Comune, in modo da coinvolgere direttamente i cittadini nelle scelte del Comune. \\ (Riferito a RF\ref{rf:creazione_bilancio_partecipativo})
\paragraph{Criteri di accettazione:}
\begin{itemize}
    \item Gli utenti amministratori dispongono di un'interfaccia per creare un nuovo bilancio partecipativo.
    \item L'interfaccia di creazione richiede di inserire una domanda, delle opzioni e una data di scadenza, rispettando i vincoli descritti in RF\ref{rf:creazione_bilancio_partecipativo}.
    \item Se l'operatore prova a pubblicare un sondaggio con dei campi incompleti o errati, oppure tenta di pubblicare un sondaggio quando ne è già attivo uno, vengono mostrati degli adeguati messaggi d'errore.
    \item Premendo un pulsante di conferma, viene pubblicato il sondaggio.
    \item Dopo la pubblicazione, il sondaggio è visualizzabile dagli altri utenti.
\end{itemize}
\paragraph{Tasks:}
\begin{itemize}
    \item Aggiungere alla UI dell'operatore l'interfaccia di creazione del bilancio partecipativo tramite compilazione di campi.
    \item Implementare l'aggiornamento del database con i dati del bilancio partecipativo pubblicato dopo la conferma di pubblicazione.
    \item Implementare e verificare i messaggi d'errore nelle casistiche elencate.
    \item Testare che il sondaggio sia visibile agli utenti dopo la pubblicazione.
\end{itemize}

\US{Consultazione Archivio Bilanci Partecipativi} \label{us:archiviobp}
Come \textit{operatore del Comune} voglio poter consultare i bilanci partecipativi conclusi, in modo da poter analizzare le scelte che sono state fatte in passato. \\ (Riferito a RF\ref{rf:consultazione_archivio_bilancio_partecipativo})
\paragraph{Criteri di accettazione:}
\begin{itemize}
    \item L'utente con ruolo di "Amministratore" dispone di un'interfaccia per visualizzare l'archivio storico dei bilanci partecipativi.
    \item Per ogni bilancio partecipativo vengono mostrati i dettagli rilevanti, come descritto in RF\ref{rf:consultazione_archivio_bilancio_partecipativo}.
\end{itemize}
\paragraph{Tasks:}
\begin{itemize}
    \item Aggiungere alla UI dell'operatore una sezione per la consultazione dell'archivio storico dei bilanci partecipativi con i relativi dati.
\end{itemize}

\US{Visualizzazione e Ricerca Utenti} \label{us:cercautente}
Come \textit{operatore del Comune} devo poter visualizzare un elenco completo degli utenti con ruolo "Amministratore", in modo da poter verificare l'esistenza del profilo di un operatore.\\ (Riferito a RF\ref{subRf:ricerca_utenti})
\paragraph{Criteri di accettazione:}
\begin{itemize}
    \item L'operatore può visualizzare un elenco degli utenti con ruolo "Amministratore" nella sua interfaccia utente.
    \item L'operatore può cercare un preciso utente inserendo il suo codice fiscale in una barra di ricerca.
    \item Se l'utente cercato è presente vengono mostrati i suoi dati, altrimenti viene mostrato un messaggio informativo.
\end{itemize}
\paragraph{Tasks:}
\begin{itemize}
    \item Aggiungere alla UI dell'operatore l'elenco degli utenti con ruolo "Amministratore".
    \item Aggiungere una barra di ricerca dove poter inserire il codice fiscale dell'utente cercato.
    \item Fare in modo che dopo aver premuto il tasto di invio venga aggiornata la pagina, mostrando il risultato della ricerca.
    \item Implementare e verificare il messaggio informativo in caso di utente non trovato.
\end{itemize}

\US{Promozione utente e Pre-autorizzazione} \label{us:promozione}
Come \textit{operatore del Comune} devo poter promuovere un utente da "Cittadino" ad "Amministratore", in modo che questo possa svolgere anche il suo ruolo da operatore nella piattaforma.\\ Inoltre, devo poter pre-autorizzare un utente, in modo da poter rendere un utente "Amministratore" anche se non è cittadino del Comune di Trento. \\(Riferito a RF\ref{subRf:assegnazione_privilegi}) 
\paragraph{Criteri di accettazione:}
\begin{itemize}
    \item Nell'interfaccia di visualizzazione degli utenti, l'operatore può digitare manualmente il codice fiscale dell'utente che vuole aggiungere alla lista degli amministratori.
    \item Se il formato dell'input non rispetta il formato del codice fiscale viene mostrato un messaggio d'errore.
    \item Se il codice fiscale inserito corrisponde ad un operatore già presente nella lista viene mostrato un messaggio informativo.
    \item Dopo l'inserimento, viene aggiornato il database: se il codice fiscale corrisponde a un cittadino, questo viene promosso; se invece non corrisponde a nessun utente, questo viene pre-autorizzato.
    \item L’utente promosso (o l’utente pre-autorizzato dopo il primo accesso) ottiene accesso a tutte le funzionalità riservate agli amministratori.
\end{itemize}
\paragraph{Tasks:}
\begin{itemize}
    \item Aggiungere una barra di inserimento dove l'operatore possa digitare il codice fiscale dell'utente da promuovere o pre-autorizzare.
    \item Implementare l'aggiornamento del database dopo l'inserimento.
    \item Implementare e verificare i messaggi informativi e d'errore nelle casistiche presentate.
    \item Verificare che l'utente promosso (o l’utente pre-autorizzato dopo il primo accesso) abbia effettivamente acquisito i privilegi da "Amministratore".
\end{itemize}

\US{Revoca privilegi} \label{us:revoca}
Come \textit{operatore del Comune} devo poter selezionare un utente "Amministratore" e revocargli i privilegi, in modo che una volta terminato il suo impiego in Comune non possa continuare ad agire sulla piattaforma come "Amministratore" (oppure per rimediare ad aggiunte errate nella lista degli amministratori). \\ (Riferito a RF\ref{subRf:revoca_previlegi})
\paragraph{Criteri di accettazione:}
\begin{itemize}
    \item Nell'elenco degli amministratori è presente un pulsante "Rimuovi" vicino ad ogni operatore, ad esclusione dell'operatore che visualizza l'interfaccia. Il pulsante serve a rimuovere il relativo operatore dalla lista degli amministratori.
    \item Premendo il pulsante viene aggiornato il database.
    \item Una volta rimosso dalla lista, se l'utente era anche "Cittadino" può continuare a usufruire dei servizi forniti ai cittadini, ma non potrà più svolgere il ruolo di operatore.
    \item Una volta rimosso dalla lista, se l'utente era solo "Amministratore" perde l'accesso al sistema.
\end{itemize}
\paragraph{Tasks:}
\begin{itemize}
    \item Aggiungere il pulsante "Rimuovi" vicino agli utenti nell'elenco.
    \item Implementare l'aggiornamento del database.
    \item Verificare che l'utente rimosso dalla lista non possa più svolgere il ruolo di operatore e che mantenga eventualmente i privilegi da cittadino.
\end{itemize}

\US{Firma} \label{us:firma}
Come \textit{cittadino del Comune di Trento} voglio poter firmare un'iniziativa, in modo da supportare la causa. \\ (Riferito a RF\ref{rf:firma})
\paragraph{Criteri di accettazione:}
\begin{itemize}
    \item Nel caso di iniziative interne alla piattaforma, ovviamente in corso, l'utente "Cittadino" deve poter firmare l'iniziativa cliccando un apposito pulsante (disabilitato per gli utenti autenticati non cittadini).
    \item Le informazioni relative alla nuova firma devono essere aggiunte al database.
    \item Nel caso di iniziative interne alla piattaforma, il numero totale di adesioni si deve aggiornare in tempo reale dopo una firma.
    \item Nel caso di iniziative interne alla piattaforma, quando un utente "Cittadino" firma un'iniziativa questa viene spostata nella sezione apposita della sua dashboard personale.
    \item Nel caso di iniziative interne alla piattaforma, dopo che il cittadino ha firmato, il tasto per firmare viene disabilitato in modo da impedirgli di apportare più adesioni alla stessa iniziativa. Gli viene inoltre mostrato un messaggio informativo.
    \item Nel caso di iniziative esterne alla piattaforma, un link permette il reindirizzamento del cittadino alla piattaforma di provenienza.
\end{itemize}
\paragraph{Tasks:}
\begin{itemize}
    \item Aggiungere alle pagine di dettaglio delle iniziative il pulsante per firmarle.
    \item Implementare l'aggiornamento dei dati nel database.
    \item Implementare lo spostamento dell'iniziativa nella dashboard del cittadino.
    \item Verificare che il cittadino firmatario visualizzi l'iniziativa firmata nella propria dashboard.
    \item Disabilitare il pulsante per firmare nell'interfaccia del cittadino firmatario dopo che ha firmato. Implementare e verificare il relativo messaggio informativo.
    \item Implementare il reindirizzamento a piattaforme esterne.
    \item Testare che gli utenti visualizzino dati aggiornati dopo una firma.
\end{itemize}

\US{Dashboard personale} \label{us:dashboard}
Come \textit{cittadino del Comune di Trento} voglio avere una dashboard personale per vedere le mie attività e ricevere aggiornamenti. \\ (Riferito a RF\ref{rf:dashboard_personale})
\paragraph{Criteri di accettazione:}
\begin{itemize}
    \item Il cittadino può vedere una dashboard che gli mostri le iniziative create, firmate e seguite e includa una sezione per le notifiche.
\end{itemize}
\paragraph{Tasks:}
\begin{itemize}
    \item Aggiungere la dashboard alla UI del cittadino.
\end{itemize}

\US{Creazione di un'iniziativa} \label{us:creainiz}
Come \textit{cittadino del Comune di Trento} voglio poter creare un'iniziativa per proporre un cambiamento nella mia città. Vorrei inoltre che quando provo a pubblicare un'iniziativa mi venga notificato se ne esistono già di molto simili, perché se ci fossero troppe iniziative uguali i voti si dividerebbero sulle varie iniziative invece di concentrarsi su una sola. \\ Come \textit{operatore del Comune} voglio che non ci siano più iniziative troppo simili tra loro, perché questo complicherebbe l'analisi dei dati e sovraccaricherebbe il database. \\ (Riferito a RF\ref{rf:creazione_iniziativa} e RF\ref{rf:controllo_duplicati})
\paragraph{Criteri di accettazione:}
\begin{itemize}
    \item L'utente con ruolo di cittadino può creare un'iniziativa, inserendo i dati elencati in RF\ref{rf:creazione_iniziativa} su un'interfaccia visualizzata dopo aver premuto un apposito pulsante (disabilitato per gli utenti autenticati non cittadini).
    \item Se il cittadino conferma la pubblicazione senza aver compilato i campi obbligatori, viene mostrato un messaggio d'errore.
    \item Se il cittadino tenta di creare una nuova iniziativa prima che finisca il periodo di cool-down di 14 giorni gli viene mostrato un messaggio d'errore.
    \item Dopo la pubblicazione, la data di scadenza viene impostata automaticamente a 60 giorni dal giorno di pubblicazione e lo stato viene impostato a "In corso".
    \item Prima della pubblicazione definitiva, l'iniziativa viene sottoposta a un controllo duplicati effettuato dal sistema.
    \item Se non viene superato il controllo viene mostrato un messaggio informativo.
    \item La nuova iniziativa viene resa visibile a tutti gli utenti.
    \item L'iniziativa viene aggiunta all'apposita sezione della dashboard dell'utente.
\end{itemize}
\paragraph{Tasks:}
\begin{itemize}
    \item Aggiungere alla UI dell'utente cittadino un pulsante per creare una nuova iniziativa.
    \item Creare un'interfaccia per la compilazione dei campi dell'iniziativa.
    \item Implementare l'impostazione automatica della data di scadenza e dello stato.
    \item Implementare l'algoritmo di controllo duplicati.
    \item Implementare e verificare i messaggi informativi e d'errore nelle casistiche presentate.
    \item Aggiornare il database inserendo i dati relativi alla nuova iniziativa.
    \item Testare che la nuova iniziativa, una volta superati i controlli e pubblicata, sia visibile a tutti gli utenti.  
    \item Verificare che il cittadino firmatario visualizzi l'iniziativa creata nella propria dashboard.
\end{itemize}

\US{Tracciamento stato} \label{us:tracciamento}
Come \textit{cittadino del Comune di Trento} voglio seguire lo stato d'avanzamento delle iniziative che mi interessano, in modo da sapere se vengono respinte o accettate. \\ (Riferito a RF\ref{rf:tracciamento_stato})
\paragraph{Criteri di accettazione:}
\begin{itemize}
    \item Il cittadino può salvare un'iniziativa nella propria dashboard cliccando su un apposito pulsante  (assente per gli utenti autenticati non cittadini).
    \item Premendo sul tasto una seconda volta, l'iniziativa viene rimossa dalla dashboard.
\end{itemize}
\paragraph{Tasks:}
\begin{itemize}
    \item Aggiungere all'interfaccia di visualizzazione dell'iniziativa un pulsante per aggiungerla in una sezione della dashboard personale.
    \item Implementare l'aggiunta dell'iniziativa nella dashboard del cittadino quando viene premuto il tasto.
    \item Implementare la rimozione della stessa dalla dashboard quando viene premuto nuovamente il tasto.
    \item Testare che i cambiamenti si riflettano nella dashboard del cittadino.
\end{itemize}

\US{Votazione al bilancio partecipativo} \label{us:voto}
Come \textit{cittadino del Comune di Trento} voglio poter votare ai sondaggi di bilancio partecipativo proposti dal Comune per far valere la mia opinione di cittadino. \\ (Riferito a RF\ref{rf:voto})
\paragraph{Criteri di accettazione:}
\begin{itemize}
    \item L'utente visualizza il bilancio partecipativo in atto, e se è un cittadino autenticato può votare una delle opzioni cliccando su di essa (altrimenti, i pulsanti sono disabilitati).
    \item Una volta votata, il sistema aumenta il numero di voti dell'opzione votata.
    \item Dopo aver votato, l'utente non può cambiare scelta.
\end{itemize}
\paragraph{Tasks:}
\begin{itemize}
    \item Aggiungere alla UI una sezione per il bilancio partecipativo, in cui il cittadino possa selezionare l'opzione scelta.
    \item Aggiornare il database dopo un voto.
\end{itemize}

\US{Importazione dati esterni} 
Come \textit{utente} voglio poter visualizzare iniziative relative al Comune di Trento anche se non sono state create sulla piattaforma, per poter consultare e firmare anche quelle. \\ (Riferito a RF\ref{rf:import_dati_esterni}) 
\paragraph{Criteri di accettazione:}
\begin{itemize}
    \item L'utente può visualizzare anche iniziative presenti in altre piattaforme.
\end{itemize}
\paragraph{Tasks:}
\begin{itemize}
    \item Implementare un metodo di raccolta dei dati esterni e il loro inserimento nel database.
    \item Testare che gli utenti possano visualizzare le iniziative esterne.
\end{itemize}

\US{Sistema di notifiche}
Come \textit{cittadino del Comune di Trento} voglio poter ricevere delle notifiche riguardanti eventi significativi, in modo da rimanere aggiornato. \\ (Riferito a RF\ref{rf:notifiche})
\paragraph{Criteri di accettazione:}
\begin{itemize}
    \item Il cittadino riceve una notifica (all'interno della piattaforma e per mail) nei casi elencati in RF\ref{rf:notifiche}.
    \item Il cittadino dispone dunque di una sezione della dashboard in cui possa consultare le notifiche.
\end{itemize}
\paragraph{Tasks:}
\begin{itemize}
    \item Aggiungere nella dashboard una sezione per le notifiche.
    \item Implementare l'invio delle notifiche ai cittadini.
\end{itemize}

\newpage 