\section{Use Case Diagram}
RF\ref{rf:login}: Accesso Utente \\
RF\ref{rf:creazione_profilo}: Creazione nuovo profilo

\begin{figure}[h!]
  \centering % \raggedright se si vuole mettere l'immagine a sinistra
  \fbox{\includegraphics[width=0.9\textwidth]{img/Login Diagram.png}}
  \label{fig:usecase_login}
\end{figure}

\paragraph{\large Use Case RF\ref{rf:login}: Accesso utente (Login)}
\noindent
1. L'utente inizialmente visualizza l'interfaccia di default dell'applicativo (quella dell'user con meno privilegi).\\
2. Per fare log-in, clicca su un pulsante apposito.\\
3. Il sistema reindirizza il cittadino al provider SPID o CIE per l’autenticazione. \\
4. L'utente accede alla propria dashboard. [eccezione 1]

\paragraph{Eccezioni}
\noindent
1. Se le credenziali non sono state registrate in precedenza, allora l'utente entrerà in fase di creazione profilo.

\paragraph{\large Use Case RF\ref{rf:creazione_profilo}: Creazione profilo al primo accesso}
\noindent
1. Il cittadino visualizza la pagina di accesso.\\
2. Seleziona la modalità di autenticazione: SPID o CIE.\\
3. Il sistema reindirizza il cittadino al provider SPID o CIE per l’autenticazione.\\
4. Dopo il login, vengono restituiti al sistema i dati identificativi (nome, cognome, codice fiscale, indirizzo di residenza, ecc.).\\
5. La piattaforma verifica automaticamente che la residenza sia nel territorio del Comune di Trento. [eccezione 1]\\
6. Una volta eseguita la verifica viene richiesto all'utente di fornire l'indirizzo email come specificato in RF\ref{rf:creazione_profilo} . \\
7. L'email server verifica la validità dei dati inseriti. [eccezione 2]\\
8. Se la verifica va a buon fine viene generato un nuovo account con i propri dati personali e il ruolo di "cittadino".

\paragraph{Eccezioni}
\noindent
1. Se l'utente non risiede nel Comune di Trento e non fa parte della lista di amministratori pre-autorizzati [estensione 1] %non so se ha senso metterla qui l'estensione
, la fase di verifica della piattaforma fallisce e viene mostrato all'utente un messaggio di errore con scritto che non soddisfa i requisiti necessari a registrarsi.\\
2. Se l'indirizzo email non viene confermato in tempo, l'utente dovrà ripetere la registrazione.

\paragraph{Estensioni}
\noindent
1. Viene creato un profilo con il solo ruolo di “Amministratore”.

\newpage \noindent
RF\ref{rf:consultazione_iniziative}: Consultazione delle iniziative\\
RF\ref{rf:consultazione_singola_iniz}: Consultazione di una singola iniziativa\\
\begin{figure}[h!]
  \centering
  \fbox{\includegraphics[width=0.9\textwidth]{img/iniziative.png}}
  \label{fig:iniziative}
\end{figure}

\paragraph{\large Use case RF\ref{rf:consultazione_iniziative}: Consultazione delle iniziative}
\noindent
1. L'utente visualizza come default la home page.\\
2. L'utente consulta il catalogo pubblico delle iniziative presenti
2. Nella barra di ricerca, una volta inserite le parole chiave e scelti strumenti aggiuntivi (vedi RF\ref{rf:consultazione_iniziative}), viene premuto il pulsante RICERCA. [Eccezione 1]\\
4. Il sistema, collegandosi al database, mostra le iniziative che rispettano i criteri di ricerca. [eccezione 1]\\

\paragraph{Eccezioni} \noindent
1. Se non viene trovata nessuna iniziativa che rispetta tali criteri, viene visualizzato un apposito messaggio.

\paragraph{\large Use case RF\ref{rf:consultazione_singola_iniz}: Consultazione di una singola iniziativa}
\noindent
1. L'utente visualizza la lista di iniziative che gli interessano. \\
2. L'utente seleziona una singola iniziativa cliccando su di essa. \\
3. Il sistema risponde mostrando una pagina di dettaglio con tutte le informazioni associate all’iniziativa (si veda RF4). \\

\noindent
RF5: Analisi delle richieste\\
RF6: Gestione iniziative
\begin{figure}[h!]
  \centering
  \fbox{\includegraphics[width=0.9\textwidth]{img/analisi_gestione.png}}
  \label{fig:analisi_gestione}
\end{figure}

\paragraph{\large Use case RF\ref{rf:analisi_richieste}: Analisi delle richieste}
\noindent
1. L’amministratore accede alla home page. \\
2. Seleziona l’icona relativa agli strumenti di analisi. \\
3. Sceglie una delle opzioni analitiche fornite in RF\ref{rf:analisi_richieste}. \\
4. Il server elabora i dati e restituisce tabelle e grafici interattivi aggiornati.

\paragraph{\large Use case RF\ref{rf:gestione_iniziative}: Gestione iniziative}
\noindent
1. L’amministratore accede alla home page. \\
2. Scorre la sezione “Iniziative in corso”. \\
3. Seleziona un’iniziativa da aggiornare. \\
4. Applica una o più modifiche tra quelle fornite in RF\ref{rf:gestione_iniziative}. [Estensione 1] \\
5. Conferma le modifiche premendo il pulsante “Salva”. \\
6. Il server riceve i dati aggiornati e li applica al database.

\paragraph{Estensioni} \noindent
1. In caso di modifica allo stato dell’iniziativa, l’amministratore deve inserire una motivazione, che sarà resa visibile ai cittadini nella scheda dell’iniziativa.

\newpage \noindent
RF\ref{rf:creazione_bilancio_partecipativo}: Creazione bilancio partecipativo \\
RF\ref{rf:consultazione_archivio_bilancio_partecipativo}: Consultazione Archivio Bilanci Partecipativi
\begin{figure}[h!]
  \centering
  \fbox{\includegraphics[width=0.9\textwidth]{img/bilanci_part.png}}
  \label{fig:bilanci_part}
\end{figure}

\paragraph{Use case RF\ref{rf:creazione_bilancio_partecipativo}: Creazione bilancio partecipativo}
\noindent
1. L’amministratore accede alla home page. \\
2. Visualizza l’interfaccia di gestione dei bilanci partecipativi. \\
3. Seleziona "Crea nuovo bilancio partecipativo". \\
4. Compila i campi richiesti \\
5. Conferma premendo sul pulsante "Conferma". \\
6. Se tutte le condizioni sono soddisfatte, il sistema registra e pubblica il bilancio, rendendolo visibile agli utenti. [Eccezione 1][Eccezione 2][Eccezione 3] \\
7. L’amministratore riceve conferma dell’avvenuta pubblicazione.

\paragraph{Eccezioni} \noindent
1. Se è già presente un bilancio partecipativo attivo, il sistema blocca la pubblicazione e notifica l’errore (“Esiste già un bilancio attivo”). \\
2. Se la durata è inferiore a 14 giorni, il sistema rifiuta la creazione e richiede la correzione delle date. \\
3. Se il numero di risposte non rientra tra 2 e 5, il sistema mostra un messaggio di validazione.

\paragraph{Use case RF\ref{rf:consultazione_archivio_bilancio_partecipativo}: Consultazione Archivio Bilanci Partecipativi}
\noindent
1. L’amministratore accede alla home page. \\
2. Entra nella sezione “Archivio bilanci partecipativi”. [Eccezione 1]\\
3. Il sistema mostra l’elenco dei bilanci partecipativi conclusi. \\
4. L’amministratore seleziona un bilancio specifico dall’elenco. \\
5. Il sistema recupera i dati corrispondenti dal database e li mostra. [Eccezione 2][Estensione 1]

\paragraph{Eccezioni} \noindent
1. Se il database non contiene bilanci archiviati, il sistema mostra un messaggio informativo (“Nessun bilancio concluso disponibile”). \\
2. In caso di errore nel recupero dei dati (es. connessione al database fallita), il sistema notifica l’amministratore e invita a riprovare.

\paragraph{Estensioni} \noindent
1. L’amministratore può eventualmente esportare i risultati o visualizzare grafici riepilogativi.\\

\newpage \noindent
RF\ref{rf:ruoli_amministrativi}: Gestione Ruoli Amministrativi
\begin{figure}[!h]
  \centering
  \fbox{\includegraphics[width=0.9\textwidth]{img/ruoli_amministrativi.png}}
  \label{fig:ruoli_amministrativi}
\end{figure}

\paragraph{Use case RF\ref{rf:ruoli_amministrativi}: Gestione Ruoli Amministrativi}
\noindent
1. L’amministratore accede alla home page. \\
2. Entra nella sezione “Gestione ruoli amministrativi” dal pannello di amministrazione. \\
3. Il sistema mostra l’elenco di tutti gli utenti con ruolo amministrativo, come descritto in RF\ref{subRf:ricerca_utenti}. [Estensione 1] [Eccezione 1] \\
4. Seleziona un utente della lista.
5. Applica una delle opzioni fornite in RF\ref{subRf:assegnazione_privilegi} o RF\ref{subRf:revoca_previlegi}. [Eccezione 2] \\
6. Il sistema aggiorna i dati nel database e conferma l’avvenuta modifica dei privilegi. [Eccezione 3]

\paragraph{Eccezioni} \noindent
1. Se il Codice Fiscale inserito non esiste e non si desidera pre-autorizzarlo, il sistema mostra un errore (“Utente non trovato nel sistema”).\\
2. Se un amministratore prova a revocare i propri privilegi, il sistema mostra un messaggio di blocco (“Operazione non consentita sull’utente corrente”).\\
3. Se si verifica un errore durante l’aggiornamento del database, il sistema notifica l’amministratore e annulla la modifica.

\paragraph{Estensioni} \noindent
1. L’amministratore può effettuare una ricerca con Codice Fiscale per individuare un utente da gestire.

\paragraph{\large Use case RF\ref{rf:firma}: Firma}
1. L'utente visualizza come default la home page. \\
2. Accede all’iniziativa che desidera sostenere. \\
3. Seleziona l’opzione “Firma / Sostieni l’iniziativa”. [Estensione 1][Eccezione 1] \\
4. Il sistema verifica che l’utente non abbia già firmato quella specifica iniziativa. \\
5. Il sistema registra la firma nel database.\\
6. Il numero totale di firme viene aggiornato in tempo reale e mostrato all’utente. [Estensione 2]

\paragraph{Eccezioni} \noindent
1. Se il cittadino tenta di firmare un’iniziativa interna già sostenuta, il sistema mostra un messaggio (“Hai già sostenuto questa iniziativa”) senza cambiare il conteggio.\\
2. Se l'iniziativa non è più firmabile (archiviata, respinta), il sistema blocca l'operazione e mostra un messaggio (“Questa iniziativa non è più attiva”).

\paragraph{Estensioni} \noindent
1. Se l'iniziativa proviene da una fonte esterna allora l'utente viene reindirizzato alla pagina originale. Una volta completata la firma, tramite sincronizzazione, API o scraping autorizzato, il sistema aggiorna in tempo reale il numero totale di adesioni anche nella propria piattaforma.\\
2. L’utente vede la firma registrata nella propria area personale.

\paragraph{\large Use case RF:\ref{rf:dashboard_personale}: Dashboard personale} \noindent
1. L'utente visualizza come default la home page. \\
2. Seleziona “Dashboard personale” dal menu principale. \\
3. Il sistema recupera dal database tutte le informazioni associate all’utente elencate in RF\ref{rf:dashboard_personale}. \\
4. Il sistema genera una vista riepilogativa. [Eccezione 1] \\
5. L’utente visualizza la dashboard contenente tutte le informazioni aggiornate.

\paragraph{Eccezioni} \noindent
1. Se l’utente non ha ancora: creato iniziative, supportato iniziative o seguito iniziative, il sistema mostra una dashboard vuota, con messaggi informativi e link utili per iniziare.

\paragraph{\large Use case RF\ref{rf:creazione_iniziativa}: Creazione di una iniziativa}
\noindent
1. L'utente, dalla home page, seleziona il tasto "Crea".\\
2. Il sistema mostra il form di compilazione con i campi richiesti in RF\ref{rf:consultazione_singola_iniz}.[Eccezione 1]\\
3. Il sistema genera un'anteprima della proposta.\\
4. L'utente conferma la creazione.\\
5. Il sistema esegue il controllo automatico di duplicati. [Eccezione 2] \\
6. L'iniziativa viene registrata nel database e resa visibile nella piattaforma con lo stato "in corso".

\paragraph{Eccezioni:}
\noindent
1. Se l'utente non inserisce tutti i dati obbligatori, il sistema segnala i campi mancanti.\\
2. Se viene rilevata una similarità elevata, il sistema mostra un avviso con le iniziative simili già presenti.

\paragraph{\large Use case RF\ref{rf:tracciamento_stato}: Tracciamento stato}
\noindent
1. L'utente visualizza come default la home page. \\
2. Seleziona un’iniziativa (interna o esterna, vedi RF10 e RF18). \\
3. Sceglie l’opzione “Segui iniziativa / Aggiungi alla dashboard personale”. \\
4. Il sistema aggiunge l’iniziativa all’elenco personale dell’utente (dashboard, RF13). \\
5. Il sistema monitora lo stato dell’iniziativa in base al suo flusso di vita definito in RF18. [Estensione 1] \\
8. L’utente visualizza sempre lo stato aggiornato dell’iniziativa nella propria dashboard personale.

\paragraph{Estensioni} \noindent
1. Se lo stato dell’iniziativa cambia il sistema notifica tempestivamente l’utente.

\paragraph{\large Use Case RF\ref{rf:voto}: Votazione del bilancio partecipativo} \noindent
1. L'utente visualizza come default la home page. \\
2. Apre la sezione dedicata ai bilanci partecipativi. \\
3. Il sistema mostra l’elenco dei bilanci partecipativi attualmente attivi. [Eccezione 1] \\
4. L’utente seleziona il sondaggio. [Eccezione 2] \\
5. Il sistema mostra il modulo di voto.
6. L’utente seleziona una delle opzioni proposte e conferma il voto.
7. Il sistema registra la preferenza dell’utente nel database e aggiorna il conteggio delle votazioni.
8. Al termine del voto, viene mostrato un messaggio di conferma.
9. Alla chiusura del periodo di votazione, il sistema invia automaticamente all’utente una notifica sui risultati finali, come previsto in RF19.

\paragraph{Eccezioni} \noindent
1. Se non esiste alcun bilancio attivo, il sistema mostra un messaggio: “Nessun bilancio partecipativo attivo al momento.” \\
2. Se l'utente ha già votato, il sistema mostra un avviso: “Hai già votato questo bilancio partecipativo.” \\
3. Se il database o il servizio interno non risponde, il sistema mostra un errore temporaneo.

\paragraph{\large Use case RF\ref{rf:controllo_duplicati}: Controllo duplicati}
1. L’utente compila il modulo per la creazione di una nuova iniziativa (titolo, descrizione, luogo, categoria, allegati).\\
2. Prima della pubblicazione, il sistema avvia il controllo automatico anti-duplicati.\\
3. Utilizzando algoritmi di similarità testuale, viene calcolato un punteggio di somiglianza.\\
4. Se il punteggio supera una soglia predefinita, il sistema considera la proposta duplicata o troppo simile [Eccezione 1].\\
5. Se la similarità è bassa, la proposta viene accettata e pubblicata normalmente.

\paragraph{Eccezioni:}
1. La pubblicazione viene bloccata e l’utente riceve una notifica con un elenco di iniziative simili già presenti.

\paragraph{\large Use case RF\ref{rf:import_dati_esterni}: Importazione dati esterni}
\noindent
1. Un processo pianificato avvia periodicamente la procedura di importazione. \\
2. Il sistema si connette alle API pubbliche o agli endpoint open data delle piattaforme esterne configurate. [Eccezione 1]\\
3. Vengono estratti i dati elencati in RF\ref{rf:analisi_richieste}. \\
4. Il sistema controlla la validità dei dati importati. [Eccezione 2]\\
5. Il sistema uniforma il formato dei dati a quello usato nella piattaforma aggiungendo anche un attributo di origine esterna.\\
6. Le iniziative importate vengono rese visibili sulla piattaforma.

\paragraph{Eccezioni}
\noindent
1. In caso di mancata connessione a una fonte, il sistema segnala l'errore e riprova al ciclo successivo.\\
2. Se i dati sono incompleti, i record vengono temporaneamente scartati e segnalati all'amministrazione.

\paragraph{\large Use case RF\ref{rf:aggiornamento_dati_esterni}:  Aggiornamento iniziative esterne} \noindent
1. Il sistema identifica tutte le iniziative esterne importate (RF16) presenti nel database.
2. Ogni iniziativa viene sincronizzata secondo la modalità configurata tra quelle fornite in RF\ref{rf:aggiornamento_dati_esterni}. \\
3. Il sistema aggiorna i dati dell’iniziativa sulla base delle informazioni ricevute dalle fonti esterne. [Eccezione 1][Eccezione 2][Eccezione 3][Eccezione 4]\\
4. Il database viene aggiornato e reso disponibile all’interfaccia utente e ai moduli di elaborazione (RF5, RF13).

\paragraph{Eccezioni} \noindent
1. Se il sito esterno non risponde: L’aggiornamento viene rimandato, l’iniziativa mantiene l’ultimo stato noto, il sistema registra un log di errore e ritenta al ciclo successivo. \\
2. Se i dati restituiti dalla fonte esterna non rispettano il formato previsto: Il sistema scarta l’aggiornamento, mantiene i valori precedenti e notifica un warning agli amministratori. \\
3. Se la piattaforma esterna impone limiti di frequenza: Il sistema scala automaticamente la frequenza di aggiornamento, l’iniziativa rimane nello stato attuale fino al prossimo tentativo consentito. \\
4. Se la fonte esterna segnala che l’iniziativa non esiste più: Il sistema segna l’iniziativa come “non più disponibile” o la archivia automaticamente.

\paragraph{\large Use case RF\ref{rf:ciclo_vita_stati}: Definizione e ciclo di vita degli stati} \noindent
1. Ogni iniziativa, sia creata dall’utente sulla piattaforma, sia importata da una fonte esterna (RF16), viene inizialmente associata a uno stato.
2. Se una iniziativa "in corso" non riceve una risposta entro i 60 giorni, il sistema la marca come “Archiviata”. [Estensione 1][Eccezione 1][Estensione 2] \\
3. Per le iniziative esterne, lo stato viene aggiornato automaticamente in base alle informazioni ricevute dal sito originale (RF\ref{rf:aggiornamento_dati_esterni}). [Eccezione 2]

\paragraph{Eccezioni}
1. Se l’amministratore richiede una proroga oltre i 120 giorni: il sistema rifiuta la modifica e indica il limite massimo normativo. \\
2. Se un’iniziativa importata presenta una data scaduta ma uno stato ancora “in corso”: il sistema risolve il conflitto portandola allo stato “Scaduta” e segnala l’incongruenza in un log.

\paragraph{Estensioni}
1. L’amministrazione può prorogare la durata dello stato “In corso” fino a un massimo di 120 giorni. \\
2. L’amministratore può modificare manualmente lo stato (ad esempio “Approvata”, “Respinta”) fornendo una motivazione visibile ai cittadini.

\paragraph{\large Use case RF\ref{rf:notifiche}: Sistema di notifiche}
1. Il sistema monitora eventi che richiedono la generazione di notifiche per gli utenti (RF\ref{rf:notifiche}).\\
2. Il sistema produce il messaggio di notifica. \\
3. La notifica viene inviata all’interno della piattaforma (centro notifiche), via e-mail all’indirizzo dell’utente registrato (RF\ref{subRf:reg_cittadini}). [Eccezione 1][Eccezione 2]\\
4. L’utente può visualizzare, leggere o segnare come lette le notifiche dal proprio profilo.

\paragraph{Eccezioni} \noindent
1. Se l’e-mail non può essere consegnata: La notifica interna viene comunque registrata, il sistema marca l’invio e-mail come fallito e programma un nuovo tentativo automatico. \\
2. Se durante la generazione della notifica si verifica un errore: L’evento viene registrato in un log, il sistema riprova automaticamente, l’utente non perde la notifica (ritento fino a successo).

%---------------------------------------------------------------------------------------------------------------------
\newpage 