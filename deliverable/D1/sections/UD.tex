\begingroup
% RIDEFINIZIONE LOCALE: Toglie gli spazi verticali a tutti gli elenchi di questa sezione
\let\oldenumerate\enumerate
\let\endoldenumerate\endenumerate
\renewenvironment{enumerate}{%
  \oldenumerate
  \setlength{\itemsep}{0pt}%
  \setlength{\parskip}{0pt}%
  \setlength{\parsep}{0pt}%
}{%
  \endoldenumerate
}

\section{Use Case Diagram}
RF\ref{rf:login}: Accesso Utente \\
RF\ref{rf:creazione_profilo}: Creazione nuovo profilo\\
RF\ref{rf:logout}: Logout

\begin{figure}[ht!]
  \centering % \raggedright se si vuole mettere l'immagine a sinistra
  \fbox{\includegraphics[width=0.9\textwidth]{img/RF1 RF2 RF5.png}}
  \label{fig:usecase_login}
\end{figure}

\paragraph{\large Use Case RF\ref{rf:login}: Accesso utente (Login)}
\begin{enumerate}
    \item L'utente visualizza la home page (vista standard al primo accesso per utente non registrato).
    \item L'utente clicca sul pulsante verde "Accedi" in alto a destra, oppure compie qualsiasi altra azione concessa solamente ai cittadini (creare iniziativa, firmare iniziativa...).
    \item Il sistema reindirizza l'utente alla schermata di accesso.
    \item L'utente sceglie tra il provider SPID o CIE per verificare la propria identità.
    \item L'utente completa l'autenticazione.
    \item L'utente ritorna alla home page avendo acquisito tutti i privilegi che gli spettano. [Eccezione 1]
\end{enumerate}

\paragraph{Eccezioni}
\begin{enumerate}
    \item Se le credenziali non sono state registrate in precedenza, allora l'utente entrerà in fase di creazione profilo.
\end{enumerate}

\paragraph{\large Use Case RF\ref{rf:creazione_profilo}: Creazione profilo al primo accesso}
\begin{enumerate}
    \item Dopo il login, vengono restituiti al sistema i dati identificativi (nome, cognome, codice fiscale, indirizzo di residenza, ecc.).
    \item La piattaforma verifica che la residenza sia nel territorio del Comune di Trento. [Eccezione 1]
    \item Viene richiesto all'utente di fornire l'indirizzo email come specificato in RF\ref{rf:creazione_profilo}.
    \item Il sistema manda un codice OTP all'indirizzo email fornito. [Estensione 2]
    \item L'utente inserisce il codice OTP ricevuto.
    \item Viene generato un nuovo account con i propri dati personali e il ruolo di "cittadino". [Eccezione 2] [Eccezione 3]
\end{enumerate}

\paragraph{Eccezioni}
\begin{enumerate}
    \item Se l'utente non risiede nel Comune di Trento e non fa parte della lista di amministratori pre-autorizzati [Estensione 1], la fase di verifica della piattaforma fallisce e viene mostrato all'utente un messaggio di errore con scritto che non soddisfa i requisiti necessari a registrarsi.
    \item Se il codice OTP inserito è sbagliato allora viene mostrato un messaggio di errore e l'utente dovrà riscrivere il codice correttamente.
    \item Se il codice OTP è scaduto allora verrà mostrato un messaggio di errore e l'utente dovrà cliccare su "invia nuovo codice OTP".
\end{enumerate}

\paragraph{Estensioni}
\begin{enumerate}
    \item Viene creato un profilo con il solo ruolo di “Amministratore” (nel caso l'utente fosse nella lista dei pre-autorizzati).
    \item L'utente può scegliere di cambiare indirizzo email se si accorge di averne scritto uno sbagliato.
\end{enumerate}

\paragraph{\large Use Case RF\ref{rf:logout}: Logout}
\begin{enumerate}
    \item L'utente autenticato si trova in una qualsiasi pagina della piattaforma.
    \item L'utente clicca sul pulsante di logout in alto a destra (icona raffigurante una freccia che esce da un riquadro), situato accanto al proprio nome.
    \item Il sistema termina la sessione corrente.
    \item L'utente viene reindirizzato alla Home Page visualizzando l'interfaccia pubblica per utenti non autenticati.
\end{enumerate}

\newpage \noindent
RF\ref{rf:consultazione_iniziative}: Consultazione delle iniziative\\
RF\ref{rf:consultazione_singola_iniz}: Consultazione di una singola iniziativa\\
\begin{figure}[ht!]
  \centering
  \fbox{\includegraphics[width=0.9\textwidth]{img/RF3 RF4.png}}
  \label{fig:iniziative}
\end{figure}

\paragraph{\large Use case RF\ref{rf:consultazione_iniziative}: Consultazione delle iniziative}
\begin{enumerate}
    \item L'utente visualizza come default la home page.
    \item L'utente consulta il catalogo pubblico delle iniziative presenti.
    \item Nella sezione filtri a sinistra, vengono scritte le parole chiave e utilizzati gli strumenti aggiuntivi (vedi RF\ref{rf:consultazione_iniziative}).
    \item Il sistema, collegandosi al database, mostra le iniziative che rispettano i criteri di ricerca in tempo reale. [Eccezione 1]
\end{enumerate}

\paragraph{Eccezioni}
\begin{enumerate}
    \item Se non viene trovata nessuna iniziativa che rispetta tali criteri, viene visualizzato un apposito messaggio.
\end{enumerate}

\paragraph{\large Use case RF\ref{rf:consultazione_singola_iniz}: Consultazione di una singola iniziativa}
\begin{enumerate}
    \item L'utente visualizza la lista di iniziative che gli interessano.
    \item L'utente seleziona una singola iniziativa cliccando su di essa.
    \item Il sistema risponde mostrando una pagina di dettaglio con tutte le informazioni associate all’iniziativa (si veda RF4).
\end{enumerate}

\noindent
RF6: Gestione iniziative
\begin{figure}[ht!]
  \centering
  \fbox{\includegraphics[width=0.9\textwidth]{img/RF5 RF6.png}}
  \label{fig:analisi_gestione}
\end{figure}

\paragraph{\large Use case RF\ref{rf:gestione_iniziative}: Gestione iniziative}
\begin{enumerate}
    \item L'amministratore visualizza la home page.
    \item L'amministratore clicca su "Area admin" presente nella barra di navigazione.
    \item L'amministratore sceglie "monitoraggio scadenze".
    \item L'amministratore seleziona un'iniziativa da aggiornare.
    \item L'amministratore complila tutti i campi richiesti descritti in RF\ref{rf:gestione_iniziative}. [Eccezione 1]
    \item Conferma le modifiche premendo il pulsante “Invia risposta definitiva”.
    \item Il server riceve i dati aggiornati e li applica al database.
\end{enumerate}

\paragraph{Eccezioni}
\begin{enumerate}
    \item Se non vengono compilati tutti i campi richiesti in base a come descritto in RF\ref{rf:gestione_iniziative} allora verrà mostrato un messaggio di errore.
\end{enumerate}

\newpage \noindent
RF\ref{rf:creazione_bilancio_partecipativo}: Creazione bilancio partecipativo \\
RF\ref{rf:consultazione_archivio_bilancio_partecipativo}: Consultazione Archivio Bilanci Partecipativi
\begin{figure}[ht!]
  \centering
  \fbox{\includegraphics[width=0.9\textwidth]{img/RF7 RF8.png}}
  \label{fig:bilanci_part}
\end{figure}

\paragraph{Use case RF\ref{rf:creazione_bilancio_partecipativo}: Creazione bilancio partecipativo}
\begin{enumerate}
    \item L'amministratore visualizza l'area admin.
    \item L'amministratore sceglie l'opzione "Crea bilancio".
    \item L'amministratore compila i campi richiesti.
    \item L'amministratore conferma premendo sul pulsante "Conferma".
    \item Il sistema registra e pubblica il bilancio, rendendolo visibile nella home page. [Eccezione 1][Eccezione 2][Eccezione 3]
    \item L’amministratore riceve conferma dell’avvenuta pubblicazione.
\end{enumerate}

\paragraph{Eccezioni}
\begin{enumerate}
    \item Se è già presente un bilancio partecipativo attivo, il sistema blocca la pubblicazione e notifica l'errore (“Esiste già un bilancio attivo”).
    \item Se la durata è inferiore a 14 giorni, il sistema rifiuta la creazione e richiede la correzione delle date.
    \item Se il numero di risposte non rientra tra 2 e 5, il sistema mostra un messaggio di validazione.
\end{enumerate}

\paragraph{Use case RF\ref{rf:consultazione_archivio_bilancio_partecipativo}: Consultazione Archivio Bilanci Partecipativi}
\begin{enumerate}
    \item L'amministratore visualizza l'area admin.
    \item L'amministratore entra nella sezione “Archivio bilanci”.
    \item Il sistema mostra l'elenco dei bilanci partecipativi conclusi. [Eccezione 1]
    \item L'amministratore seleziona un bilancio specifico dall'elenco.
    \item Il sistema recupera i dati corrispondenti dal database e li mostra.
\end{enumerate}

\paragraph{Eccezioni}
\begin{enumerate}
    \item Se il database non contiene bilanci archiviati, il sistema mostra un messaggio informativo (“Nessun bilancio concluso disponibile”).
\end{enumerate}

\newpage \noindent
RF\ref{rf:ruoli_amministrativi}: Gestione Ruoli Amministrativi
\begin{figure}[!ht]
  \centering
  \fbox{\includegraphics[width=0.9\textwidth]{img/RF9.png}}
  \label{fig:ruoli_amministrativi1}
\end{figure}

\paragraph{Use case RF\ref{rf:ruoli_amministrativi}: Gestione Ruoli Amministrativi}
\begin{enumerate}
    \item L'amministratore visualizza l'area admin.
    \item L'amministratore entra nella sezione “Gestione personale”.
    \item Il sistema mostra l'elenco di tutti gli utenti con ruolo amministrativo, come descritto in RF\ref{subRf:ricerca_utenti}. [Estensione 1] [Eccezione 1]
    \item L'amministratore applica una delle opzioni fornite in RF\ref{subRf:assegnazione_privilegi} o RF\ref{subRf:revoca_previlegi}.
    \item Il sistema aggiorna i dati nel database e conferma l'avvenuta modifica dei privilegi.
\end{enumerate}

\paragraph{Eccezioni}
\begin{enumerate}
    \item Se il Codice Fiscale inserito non è valido allora
    \item Se il Codice fiscale non corrisponde ad alcun amministratore esistente allora viene mostrato un messaggio apposito. [Estensione 1] 
\end{enumerate}

\paragraph{Estensioni}
\begin{enumerate}
    \item Se, invece, si desidera pre-autorizzare tale utente allora viene premuto il pulsante "Si, pre-autorizza".
\end{enumerate}

\noindent
RF\ref{rf:firma}: Firma\\
RF\ref{rf:dashboard_personale}: Dashboard personale\\
RF\ref{rf:voto}: Votazione del bilancio partecipativo
\begin{figure}[!ht]
  \centering
  \fbox{\includegraphics[width=0.9\textwidth]{img/RF10 RF11 RF14.png}}
  \label{fig:ruoli_amministrativi2}
\end{figure}

\paragraph{\large Use case RF\ref{rf:firma}: Firma}
\begin{enumerate}
    \item L'utente seleziona l'iniziativa che desidera sostenere.
    \item L'utente clicca l'opzione “Firma”. [Estensione 1][Eccezione 1]
    \item Il sistema registra la firma nel database.
    \item Il numero totale di firme viene aggiornato in tempo reale e mostrato all'utente.
\end{enumerate}

\paragraph{Eccezioni}
\begin{enumerate}
    \item Se il cittadino tenta di firmare un'iniziativa interna già sostenuta, il sistema mostra un messaggio (“Hai già sostenuto questa iniziativa”) senza cambiare il conteggio.
\end{enumerate}

\paragraph{Estensioni}
\begin{enumerate}
    \item Se l'iniziativa proviene da una fonte esterna allora l'utente viene reindirizzato alla pagina originale. Una volta completata la firma, tramite sincronizzazione, API o scraping autorizzato, il sistema aggiorna in tempo reale il numero totale di adesioni anche nella propria piattaforma.
\end{enumerate}

\paragraph{\large Use case RF:\ref{rf:dashboard_personale}: Dashboard personale}
\begin{enumerate}
    \item L'utente visualizza come default la home page.
    \item L'utente seleziona “Dashboard” dalla barra di navigazione.
    \item Il sistema recupera dal database tutte le informazioni associate all'utente elencate in RF\ref{rf:dashboard_personale}.
    \item Il sistema genera una vista riepilogativa. [Eccezione 1]
\end{enumerate}

\paragraph{Eccezioni}
\begin{enumerate}
    \item Se l'utente non ha ancora: creato iniziative, supportato iniziative o seguito iniziative, il sistema mostra una dashboard vuota.
\end{enumerate}

\paragraph{\large Use Case RF\ref{rf:voto}: Votazione del bilancio partecipativo}
\begin{enumerate}
    \item L'utente visualizza come default la home page.
    \item Il sistema mostra il bilancio partecipativo attualmente attivo. [Eccezione 1]
    \item L'utente seleziona una delle opzioni proposte.
    \item Il sistema registra la preferenza dell'utente nel database e aggiorna il conteggio delle votazioni. [Eccezione 2]
    \item Viene mostrato un messaggio di conferma.
    \item Alla chiusura del periodo di votazione, il sistema invia automaticamente all'utente una notifica sui risultati finali, come previsto in RF\ref{rf:notifiche}.
\end{enumerate}

\paragraph{Eccezioni}
\begin{enumerate}
    \item Se non esiste alcun bilancio attivo, la lista di iniziative in evidenza sarà spostata più in alto.
    \item Se l'utente ha già votato, il sistema non permette di cliccare una seconda volta.
\end{enumerate}

\noindent
RF\ref{rf:creazione_iniziativa}: Creazione di una iniziativa\\
RF\ref{rf:tracciamento_stato}: Tracciamento stato

\begin{figure}[!ht]
  \centering
  \fbox{\includegraphics[width=0.9\textwidth]{img/RF12 RF13.png}}
  \label{fig:ruoli_amministrativi3}
\end{figure}

\paragraph{\large Use case RF\ref{rf:creazione_iniziativa}: Creazione di una iniziativa}
\begin{enumerate}
    \item L'utente, dalla home page, seleziona il tasto "Crea la tua iniziativa".
    \item Il sistema mostra il form di compilazione con i campi richiesti in RF\ref{rf:consultazione_singola_iniz}. 
    \item L'utente inserisce i dati necessari.
    \item L'utente conferma la creazione. [Eccezione 1] [Eccezione 2]
    \item Il sistema esegue il controllo automatico di duplicati. [Eccezione 3]
    \item L'iniziativa viene registrata nel database e resa visibile nella piattaforma con lo stato "in corso".
\end{enumerate}

\paragraph{Eccezioni:}
\begin{enumerate}
    \item Se l'utente non inserisce tutti i dati obbligatori, il sistema segnala i campi mancanti.
    \item Se l'utente ha già creato un iniziativa negli scorsi 14 giorni allora viene mostrato un messaggio di errore.
    \item Se viene rilevata una similarità elevata, il sistema mostra un avviso.
\end{enumerate}

\paragraph{\large Use case RF\ref{rf:tracciamento_stato}: Tracciamento stato}
\begin{enumerate}
    \item L'utente visualizza come default la home page.
    \item Seleziona un'iniziativa (interna o esterna).
    \item Sceglie l'opzione “Segui aggiornamenti”.
    \item Il sistema aggiunge l'iniziativa all'elenco personale dell'utente.
    \item Il sistema monitora lo stato dell'iniziativa in base al suo flusso di vita definito in RF\ref{rf:ciclo_vita_stati}.
    \item L'utente riceve una notifica nella dashboard se lo stato dell'iniziativa che ha seguito cambia.
\end{enumerate}

\paragraph{\large Use case RF\ref{rf:controllo_duplicati}: Controllo duplicati}
\begin{enumerate}
    \item L’utente compila il modulo per la creazione di una nuova iniziativa (titolo, descrizione, luogo, categoria, allegati).
    \item Prima della pubblicazione, il sistema avvia il controllo automatico anti-duplicati.
    \item Utilizzando algoritmi di similarità testuale, viene calcolato un punteggio di somiglianza.
    \item Se il punteggio supera una soglia predefinita, il sistema considera la proposta duplicata o troppo simile [Estensione 1].
    \item Se la similarità è bassa, la proposta viene accettata e pubblicata normalmente.
\end{enumerate}

\paragraph{Eccezioni:}
\begin{enumerate}
    \item La pubblicazione viene bloccata e l'utente riceve una notifica con un elenco di iniziative simili già presenti.
\end{enumerate}

\paragraph{\large Use case RF\ref{rf:import_dati_esterni}: Importazione dati esterni}
\begin{enumerate}
    \item Un processo pianificato avvia periodicamente la procedura di importazione.
    \item Il sistema si connette alle API pubbliche o agli endpoint open data delle piattaforme esterne configurate. [Eccezione 1]
    \item Vengono estratti i dati elencati in RF\ref{rf:consultazione_singola_iniz}.
    \item Il sistema controlla la validità dei dati importati. [Eccezione 2]
    \item Il sistema uniforma il formato dei dati a quello usato nella piattaforma aggiungendo anche un attributo di origine esterna.
    \item Le iniziative importate vengono rese visibili sulla piattaforma.
\end{enumerate}

\paragraph{Eccezioni}
\begin{enumerate}
    \item In caso di mancata connessione a una fonte, il sistema segnala l'errore e riprova al ciclo successivo.
    \item Se i dati sono incompleti, i record vengono temporaneamente scartati e segnalati all'amministrazione.
\end{enumerate}

\paragraph{\large Use case RF\ref{rf:aggiornamento_dati_esterni}: Aggiornamento iniziative esterne}
\begin{enumerate}
    \item Il sistema identifica tutte le iniziative esterne importate presenti nel database.
    \item Ogni iniziativa viene sincronizzata secondo la modalità configurata tra quelle fornite in RF\ref{rf:aggiornamento_dati_esterni}.
    \item Il sistema aggiorna i dati dell'iniziativa sulla base delle informazioni ricevute dalle fonti esterne. [Eccezione 1][Eccezione 2][Eccezione 3][Eccezione 4]
    \item Il database viene aggiornato.
\end{enumerate}

\paragraph{Eccezioni}
\begin{enumerate}
    \item Se il sito esterno non risponde: L'aggiornamento viene rimandato, l'iniziativa mantiene l'ultimo stato noto, il sistema registra un log di errore e ritenta al ciclo successivo.
    \item Se i dati restituiti dalla fonte esterna non rispettano il formato previsto: Il sistema scarta l'aggiornamento, mantiene i valori precedenti e notifica con un warning agli amministratori.
    \item Se la piattaforma esterna impone limiti di frequenza: Il sistema scala automaticamente la frequenza di aggiornamento, l'iniziativa rimane nello stato attuale fino al prossimo tentativo consentito.
    \item Se la fonte esterna segnala che l'iniziativa non esiste più: Il sistema segna l'iniziativa come “non più disponibile” o la archivia automaticamente.
\end{enumerate}

\paragraph{\large Use case RF\ref{rf:notifiche}: Sistema di notifiche}
\begin{enumerate}
    \item Il sistema monitora eventi che richiedono la generazione di notifiche per gli utenti (RF\ref{rf:notifiche}).
    \item Il sistema produce il messaggio di notifica.
    \item La notifica viene inviata all'interno della piattaforma e via e-mail all'indirizzo dell'utente registrato.
    \item L'utente può visualizzare e segnare come lette le notifiche dalla propria dashboard.
\end{enumerate}

\newpage 
\endgroup