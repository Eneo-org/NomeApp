\begingroup
% RIDEFINIZIONE LOCALE: Toglie gli spazi verticali a tutti gli elenchi di questa sezione
\let\oldenumerate\enumerate
\let\endoldenumerate\endenumerate
\renewenvironment{enumerate}{%
  \oldenumerate
  \setlength{\itemsep}{0pt}%
  \setlength{\parskip}{0pt}%
  \setlength{\parsep}{0pt}%
}{%
  \endoldenumerate
}

\section{Use Case Diagram}
RF\ref{rf:login}: Accesso Utente \\
RF\ref{rf:creazione_profilo}: Creazione nuovo profilo\\
RF\ref{rf:logout}: Logout

\begin{figure}[ht!]
  \centering % \raggedright se si vuole mettere l'immagine a sinistra
  \fbox{\includegraphics[width=0.9\textwidth]{img/RF1 RF2 RF5.png}}
  \label{fig:usecase_login}
\end{figure}

\paragraph{\large Use Case RF\ref{rf:login}: Accesso utente (Login)}
\begin{enumerate}
    \item L'utente visualizza l'interfaccia di default dell'applicativo (vista standard per utente non registrato).
    \item L'utente clicca sul pulsante verde "Accedi" in alto a destra, oppure clicca su "Crea iniziativa" (venendo reindirizzato alla schermata di accesso).
    \item Il sistema reindirizza il cittadino al provider SPID o CIE per l’autenticazione.
    \item L'utente accede alla home page acquisendo tutti i privilegi da cittadino. [Eccezione 1][Eccezione 2]
\end{enumerate}

\paragraph{Eccezioni}
\begin{enumerate}
    \item Se le credenziali non sono state registrate in precedenza, allora l'utente entrerà in fase di creazione profilo.
    \item Se l'utente effettua il login con SPID o CIE ma non risulta residente a Trento, viene reindirizzato alla home page visualizzando un messaggio di errore e mantenendo la vista standard (senza privilegi da cittadino).
\end{enumerate}

\paragraph{\large Use Case RF\ref{rf:creazione_profilo}: Creazione profilo al primo accesso}
\begin{enumerate}
    \item Il cittadino visualizza la pagina di accesso.
    \item Seleziona la modalità di autenticazione: SPID o CIE.
    \item Il sistema reindirizza il cittadino al provider SPID o CIE per l’autenticazione.
    \item Dopo il login, vengono restituiti al sistema i dati identificativi (nome, cognome, codice fiscale, indirizzo di residenza, ecc.).
    \item La piattaforma verifica automaticamente che la residenza sia nel territorio del Comune di Trento. [Eccezione 1]
    \item Una volta eseguita la verifica viene richiesto all'utente di fornire l'indirizzo email come specificato in RF\ref{rf:creazione_profilo}.
    \item L'email server verifica la validità dei dati inseriti. [Eccezione 2]
    \item Se la verifica va a buon fine viene generato un nuovo account con i propri dati personali e il ruolo di "cittadino".
\end{enumerate}

\paragraph{Eccezioni}
\begin{enumerate}
    \item Se l'utente non risiede nel Comune di Trento e non fa parte della lista di amministratori pre-autorizzati [Estensione 1], la fase di verifica della piattaforma fallisce e viene mostrato all'utente un messaggio di errore con scritto che non soddisfa i requisiti necessari a registrarsi.
    \item Se l'indirizzo email non viene confermato in tempo, l'utente dovrà ripetere la registrazione.
\end{enumerate}

\paragraph{Estensioni}
\begin{enumerate}
    \item Viene creato un profilo con il solo ruolo di “Amministratore” (nel caso l'utente fosse nella lista dei pre-autorizzati).
\end{enumerate}

\paragraph{\large Use Case RF\ref{rf:logout}: Logout}
\begin{enumerate}
    \item L'utente autenticato si trova in una qualsiasi pagina della piattaforma.
    \item L'utente clicca sul pulsante di logout in alto a destra (icona raffigurante una freccia che esce da un riquadro), situato accanto al proprio nome.
    \item Il sistema termina la sessione corrente (invalidando o rimuovendo il token di accesso locale).
    \item L'utente viene reindirizzato alla Home Page visualizzando l'interfaccia pubblica per utenti non autenticati.
\end{enumerate}

\newpage \noindent
RF\ref{rf:consultazione_iniziative}: Consultazione delle iniziative\\
RF\ref{rf:consultazione_singola_iniz}: Consultazione di una singola iniziativa\\
\begin{figure}[ht!]
  \centering
  \fbox{\includegraphics[width=0.9\textwidth]{img/RF3 RF4.png}}
  \label{fig:iniziative}
\end{figure}

\paragraph{\large Use case RF\ref{rf:consultazione_iniziative}: Consultazione delle iniziative}
\begin{enumerate}
    \item L'utente visualizza come default la home page.
    \item L'utente consulta il catalogo pubblico delle iniziative presenti.
    \item Nella barra di ricerca, una volta inserite le parole chiave e scelti strumenti aggiuntivi (vedi RF\ref{rf:consultazione_iniziative}), viene premuto il pulsante RICERCA. [Eccezione 1]
    \item Il sistema, collegandosi al database, mostra le iniziative che rispettano i criteri di ricerca. [Eccezione 1]
\end{enumerate}

\paragraph{Eccezioni}
\begin{enumerate}
    \item Se non viene trovata nessuna iniziativa che rispetta tali criteri, viene visualizzato un apposito messaggio.
\end{enumerate}

\paragraph{\large Use case RF\ref{rf:consultazione_singola_iniz}: Consultazione di una singola iniziativa}
\begin{enumerate}
    \item L'utente visualizza la lista di iniziative che gli interessano.
    \item L'utente seleziona una singola iniziativa cliccando su di essa.
    \item Il sistema risponde mostrando una pagina di dettaglio con tutte le informazioni associate all’iniziativa (si veda RF4).
\end{enumerate}

\noindent
RF6: Gestione iniziative
\begin{figure}[ht!]
  \centering
  \fbox{\includegraphics[width=0.9\textwidth]{img/RF5 RF6.png}}
  \label{fig:analisi_gestione}
\end{figure}

\paragraph{\large Use case RF\ref{rf:gestione_iniziative}: Gestione iniziative}
\begin{enumerate}
    \item L’amministratore accede alla home page.
    \item Scorre la sezione “Iniziative in corso”.
    \item Seleziona un’iniziativa da aggiornare.
    \item Applica una o più modifiche tra quelle fornite in RF\ref{rf:gestione_iniziative}. [Estensione 1]
    \item Conferma le modifiche premendo il pulsante “Salva”.
    \item Il server riceve i dati aggiornati e li applica al database.
\end{enumerate}

\paragraph{Estensioni}
\begin{enumerate}
    \item In caso di modifica allo stato dell’iniziativa, l’amministratore deve inserire una motivazione, che sarà resa visibile ai cittadini nella scheda dell’iniziativa.
\end{enumerate}

\newpage \noindent
RF\ref{rf:creazione_bilancio_partecipativo}: Creazione bilancio partecipativo \\
RF\ref{rf:consultazione_archivio_bilancio_partecipativo}: Consultazione Archivio Bilanci Partecipativi
\begin{figure}[ht!]
  \centering
  \fbox{\includegraphics[width=0.9\textwidth]{img/RF7 RF8.png}}
  \label{fig:bilanci_part}
\end{figure}

\paragraph{Use case RF\ref{rf:creazione_bilancio_partecipativo}: Creazione bilancio partecipativo}
\begin{enumerate}
    \item L’amministratore accede alla home page.
    \item Visualizza l’interfaccia di gestione dei bilanci partecipativi.
    \item Seleziona "Crea nuovo bilancio partecipativo".
    \item Compila i campi richiesti.
    \item Conferma premendo sul pulsante "Conferma".
    \item Se tutte le condizioni sono soddisfatte, il sistema registra e pubblica il bilancio, rendendolo visibile agli utenti. [Eccezione 1][Eccezione 2][Eccezione 3]
    \item L’amministratore riceve conferma dell’avvenuta pubblicazione.
\end{enumerate}

\paragraph{Eccezioni}
\begin{enumerate}
    \item Se è già presente un bilancio partecipativo attivo, il sistema blocca la pubblicazione e notifica l’errore (“Esiste già un bilancio attivo”).
    \item Se la durata è inferiore a 14 giorni, il sistema rifiuta la creazione e richiede la correzione delle date.
    \item Se il numero di risposte non rientra tra 2 e 5, il sistema mostra un messaggio di validazione.
\end{enumerate}

\paragraph{Use case RF\ref{rf:consultazione_archivio_bilancio_partecipativo}: Consultazione Archivio Bilanci Partecipativi}
\begin{enumerate}
    \item L’amministratore accede alla home page.
    \item Entra nella sezione “Archivio bilanci partecipativi”. [Eccezione 1]
    \item Il sistema mostra l’elenco dei bilanci partecipativi conclusi.
    \item L’amministratore seleziona un bilancio specifico dall’elenco.
    \item Il sistema recupera i dati corrispondenti dal database e li mostra. [Eccezione 2][Estensione 1]
\end{enumerate}

\paragraph{Eccezioni}
\begin{enumerate}
    \item Se il database non contiene bilanci archiviati, il sistema mostra un messaggio informativo (“Nessun bilancio concluso disponibile”).
    \item In caso di errore nel recupero dei dati (es. connessione al database fallita), il sistema notifica l’amministratore e invita a riprovare.
\end{enumerate}

\paragraph{Estensioni}
\begin{enumerate}
    \item L’amministratore può eventualmente esportare i risultati o visualizzare grafici riepilogativi.
\end{enumerate}

\newpage \noindent
RF\ref{rf:ruoli_amministrativi}: Gestione Ruoli Amministrativi
\begin{figure}[!ht]
  \centering
  \fbox{\includegraphics[width=0.9\textwidth]{img/RF9.png}}
  \label{fig:ruoli_amministrativi1}
\end{figure}

\paragraph{Use case RF\ref{rf:ruoli_amministrativi}: Gestione Ruoli Amministrativi}
\begin{enumerate}
    \item L’amministratore accede alla home page.
    \item Entra nella sezione “Gestione ruoli amministrativi” dal pannello di amministrazione.
    \item Il sistema mostra l’elenco di tutti gli utenti con ruolo amministrativo, come descritto in RF\ref{subRf:ricerca_utenti}. [Estensione 1] [Eccezione 1]
    \item Seleziona un utente della lista.
    \item Applica una delle opzioni fornite in RF\ref{subRf:assegnazione_privilegi} o RF\ref{subRf:revoca_previlegi}. [Eccezione 2]
    \item Il sistema aggiorna i dati nel database e conferma l’avvenuta modifica dei privilegi. [Eccezione 3]
\end{enumerate}

\paragraph{Eccezioni}
\begin{enumerate}
    \item Se il Codice Fiscale inserito non esiste e non si desidera pre-autorizzarlo, il sistema mostra un errore (“Utente non trovato nel sistema”).
    \item Se un amministratore prova a revocare i propri privilegi, il sistema mostra un messaggio di blocco (“Operazione non consentita sull’utente corrente”).
    \item Se si verifica un errore durante l’aggiornamento del database, il sistema notifica l’amministratore e annulla la modifica.
\end{enumerate}

\paragraph{Estensioni}
\begin{enumerate}
    \item L’amministratore può effettuare una ricerca con Codice Fiscale per individuare un utente da gestire.
\end{enumerate}

\noindent
RF\ref{rf:firma}: Firma\\
RF\ref{rf:dashboard_personale}: Dashboard personale\\
RF\ref{rf:voto}: Votazione del bilancio partecipativo
\begin{figure}[!ht]
  \centering
  \fbox{\includegraphics[width=0.9\textwidth]{img/RF10 RF11 RF14.png}}
  \label{fig:ruoli_amministrativi2}
\end{figure}

\paragraph{\large Use case RF\ref{rf:firma}: Firma}
\begin{enumerate}
    \item L'utente visualizza come default la home page.
    \item Accede all’iniziativa che desidera sostenere.
    \item Seleziona l’opzione “Firma / Sostieni l’iniziativa”. [Estensione 1][Eccezione 1]
    \item Il sistema verifica che l’utente non abbia già firmato quella specifica iniziativa.
    \item Il sistema registra la firma nel database.
    \item Il numero totale di firme viene aggiornato in tempo reale e mostrato all’utente. [Estensione 2]
\end{enumerate}

\paragraph{Eccezioni}
\begin{enumerate}
    \item Se il cittadino tenta di firmare un’iniziativa interna già sostenuta, il sistema mostra un messaggio (“Hai già sostenuto questa iniziativa”) senza cambiare il conteggio.
    \item Se l'iniziativa non è più firmabile (archiviata, respinta), il sistema blocca l'operazione e mostra un messaggio (“Questa iniziativa non è più attiva”).
\end{enumerate}

\paragraph{Estensioni}
\begin{enumerate}
    \item Se l'iniziativa proviene da una fonte esterna allora l'utente viene reindirizzato alla pagina originale. Una volta completata la firma, tramite sincronizzazione, API o scraping autorizzato, il sistema aggiorna in tempo reale il numero totale di adesioni anche nella propria piattaforma.
    \item L’utente vede la firma registrata nella propria area personale.
\end{enumerate}

\paragraph{\large Use case RF:\ref{rf:dashboard_personale}: Dashboard personale}
\begin{enumerate}
    \item L'utente visualizza come default la home page.
    \item Seleziona “Dashboard personale” dal menu principale.
    \item Il sistema recupera dal database tutte le informazioni associate all’utente elencate in RF\ref{rf:dashboard_personale}.
    \item Il sistema genera una vista riepilogativa. [Eccezione 1]
    \item L’utente visualizza la dashboard contenente tutte le informazioni aggiornate.
\end{enumerate}

\paragraph{Eccezioni}
\begin{enumerate}
    \item Se l’utente non ha ancora: creato iniziative, supportato iniziative o seguito iniziative, il sistema mostra una dashboard vuota, con messaggi informativi e link utili per iniziare.
\end{enumerate}

\paragraph{\large Use Case RF\ref{rf:voto}: Votazione del bilancio partecipativo}
\begin{enumerate}
    \item L'utente visualizza come default la home page.
    \item Apre la sezione dedicata ai bilanci partecipativi.
    \item Il sistema mostra l’elenco dei bilanci partecipativi attualmente attivi. [Eccezione 1]
    \item L’utente seleziona il sondaggio. [Eccezione 2]
    \item Il sistema mostra il modulo di voto.
    \item L’utente seleziona una delle opzioni proposte e conferma il voto.
    \item Il sistema registra la preferenza dell’utente nel database e aggiorna il conteggio delle votazioni.
    \item Al termine del voto, viene mostrato un messaggio di conferma.
    \item Alla chiusura del periodo di votazione, il sistema invia automaticamente all’utente una notifica sui risultati finali, come previsto in RF19.
\end{enumerate}

\paragraph{Eccezioni}
\begin{enumerate}
    \item Se non esiste alcun bilancio attivo, il sistema mostra un messaggio: “Nessun bilancio partecipativo attivo al momento.”
    \item Se l'utente ha già votato, il sistema mostra un avviso: “Hai già votato questo bilancio partecipativo.”
    \item Se il database o il servizio interno non risponde, il sistema mostra un errore temporaneo.
\end{enumerate}

\noindent
RF\ref{rf:creazione_iniziativa}: Creazione di una iniziativa\\
RF\ref{rf:tracciamento_stato}: Tracciamento stato

\begin{figure}[!ht]
  \centering
  \fbox{\includegraphics[width=0.9\textwidth]{img/RF12 RF13.png}}
  \label{fig:ruoli_amministrativi3}
\end{figure}

\paragraph{\large Use case RF\ref{rf:creazione_iniziativa}: Creazione di una iniziativa}
\begin{enumerate}
    \item L'utente, dalla home page, seleziona il tasto "Crea".
    \item Il sistema mostra il form di compilazione con i campi richiesti in RF\ref{rf:consultazione_singola_iniz}. [Eccezione 1]
    \item Il sistema genera un'anteprima della proposta.
    \item L'utente conferma la creazione.
    \item Il sistema esegue il controllo automatico di duplicati. [Eccezione 2]
    \item L'iniziativa viene registrata nel database e resa visibile nella piattaforma con lo stato "in corso".
\end{enumerate}

\paragraph{Eccezioni:}
\begin{enumerate}
    \item Se l'utente non inserisce tutti i dati obbligatori, il sistema segnala i campi mancanti.
    \item Se viene rilevata una similarità elevata, il sistema mostra un avviso con le iniziative simili già presenti.
\end{enumerate}

\paragraph{\large Use case RF\ref{rf:tracciamento_stato}: Tracciamento stato}
\begin{enumerate}
    \item L'utente visualizza come default la home page.
    \item Seleziona un’iniziativa (interna o esterna, vedi RF10 e RF18).
    \item Sceglie l’opzione “Segui iniziativa / Aggiungi alla dashboard personale”.
    \item Il sistema aggiunge l’iniziativa all’elenco personale dell’utente (dashboard, RF13).
    \item Il sistema monitora lo stato dell’iniziativa in base al suo flusso di vita definito in RF18. [Estensione 1]
    \item L’utente visualizza sempre lo stato aggiornato dell’iniziativa nella propria dashboard personale.
\end{enumerate}

\paragraph{Estensioni}
\begin{enumerate}
    \item Se lo stato dell’iniziativa cambia il sistema notifica tempestivamente l’utente.
\end{enumerate}

\paragraph{\large Use case RF\ref{rf:controllo_duplicati}: Controllo duplicati}
\begin{enumerate}
    \item L’utente compila il modulo per la creazione di una nuova iniziativa (titolo, descrizione, luogo, categoria, allegati).
    \item Prima della pubblicazione, il sistema avvia il controllo automatico anti-duplicati.
    \item Utilizzando algoritmi di similarità testuale, viene calcolato un punteggio di somiglianza.
    \item Se il punteggio supera una soglia predefinita, il sistema considera la proposta duplicata o troppo simile [Eccezione 1].
    \item Se la similarità è bassa, la proposta viene accettata e pubblicata normalmente.
\end{enumerate}

\paragraph{Eccezioni:}
\begin{enumerate}
    \item La pubblicazione viene bloccata e l’utente riceve una notifica con un elenco di iniziative simili già presenti.
\end{enumerate}

\paragraph{\large Use case RF\ref{rf:import_dati_esterni}: Importazione dati esterni}
\begin{enumerate}
    \item Un processo pianificato avvia periodicamente la procedura di importazione.
    \item Il sistema si connette alle API pubbliche o agli endpoint open data delle piattaforme esterne configurate. [Eccezione 1]
    \item Vengono estratti i dati elencati in RF\ref{rf:analisi_richieste}.
    \item Il sistema controlla la validità dei dati importati. [Eccezione 2]
    \item Il sistema uniforma il formato dei dati a quello usato nella piattaforma aggiungendo anche un attributo di origine esterna.
    \item Le iniziative importate vengono rese visibili sulla piattaforma.
\end{enumerate}

\paragraph{Eccezioni}
\begin{enumerate}
    \item In caso di mancata connessione a una fonte, il sistema segnala l'errore e riprova al ciclo successivo.
    \item Se i dati sono incompleti, i record vengono temporaneamente scartati e segnalati all'amministrazione.
\end{enumerate}

\paragraph{\large Use case RF\ref{rf:aggiornamento_dati_esterni}: Aggiornamento iniziative esterne}
\begin{enumerate}
    \item Il sistema identifica tutte le iniziative esterne importate (RF16) presenti nel database.
    \item Ogni iniziativa viene sincronizzata secondo la modalità configurata tra quelle fornite in RF\ref{rf:aggiornamento_dati_esterni}.
    \item Il sistema aggiorna i dati dell’iniziativa sulla base delle informazioni ricevute dalle fonti esterne. [Eccezione 1][Eccezione 2][Eccezione 3][Eccezione 4]
    \item Il database viene aggiornato e reso disponibile all’interfaccia utente e ai moduli di elaborazione (RF5, RF13).
\end{enumerate}

\paragraph{Eccezioni}
\begin{enumerate}
    \item Se il sito esterno non risponde: L’aggiornamento viene rimandato, l’iniziativa mantiene l’ultimo stato noto, il sistema registra un log di errore e ritenta al ciclo successivo.
    \item Se i dati restituiti dalla fonte esterna non rispettano il formato previsto: Il sistema scarta l’aggiornamento, mantiene i valori precedenti e notifica un warning agli amministratori.
    \item Se la piattaforma esterna impone limiti di frequenza: Il sistema scala automaticamente la frequenza di aggiornamento, l’iniziativa rimane nello stato attuale fino al prossimo tentativo consentito.
    \item Se la fonte esterna segnala che l’iniziativa non esiste più: Il sistema segna l’iniziativa come “non più disponibile” o la archivia automaticamente.
\end{enumerate}

\paragraph{\large Use case RF\ref{rf:ciclo_vita_stati}: Definizione e ciclo di vita degli stati}
\begin{enumerate}
    \item Ogni iniziativa, sia creata dall’utente sulla piattaforma, sia importata da una fonte esterna (RF16), viene inizialmente associata a uno stato.
    \item Se una iniziativa "in corso" non riceve una risposta entro i 60 giorni, il sistema la marca come “Archiviata”. [Estensione 1][Eccezione 1][Estensione 2]
    \item Per le iniziative esterne, lo stato viene aggiornato automaticamente in base alle informazioni ricevute dal sito originale (RF\ref{rf:aggiornamento_dati_esterni}). [Eccezione 2]
\end{enumerate}

\paragraph{Eccezioni}
\begin{enumerate}
    \item Se l’amministratore richiede una proroga oltre i 120 giorni: il sistema rifiuta la modifica e indica il limite massimo normativo.
    \item Se un’iniziativa importata presenta una data scaduta ma uno stato ancora “in corso”: il sistema risolve il conflitto portandola allo stato “Scaduta” e segnala l’incongruenza in un log.
\end{enumerate}

\paragraph{Estensioni}
\begin{enumerate}
    \item L’amministrazione può prorogare la durata dello stato “In corso” fino a un massimo di 120 giorni.
    \item L’amministratore può modificare manualmente lo stato (ad esempio “Approvata”, “Respinta”) fornendo una motivazione visibile ai cittadini.
\end{enumerate}

\paragraph{\large Use case RF\ref{rf:notifiche}: Sistema di notifiche}
\begin{enumerate}
    \item Il sistema monitora eventi che richiedono la generazione di notifiche per gli utenti (RF\ref{rf:notifiche}).
    \item Il sistema produce il messaggio di notifica.
    \item La notifica viene inviata all’interno della piattaforma (centro notifiche), via e-mail all’indirizzo dell’utente registrato (RF\ref{subRf:reg_cittadini}). [Eccezione 1][Eccezione 2]
    \item L’utente può visualizzare, leggere o segnare come lette le notifiche dal proprio profilo.
\end{enumerate}

\paragraph{Eccezioni}
\begin{enumerate}
    \item Se l’e-mail non può essere consegnata: La notifica interna viene comunque registrata, il sistema marca l’invio e-mail come fallito e programma un nuovo tentativo automatico.
    \item Se durante la generazione della notifica si verifica un errore: L’evento viene registrato in un log, il sistema riprova automaticamente, l’utente non perde la notifica (ritento fino a successo).
\end{enumerate}

%---------------------------------------------------------------------------------------------------------------------
\newpage 
\endgroup