\section{Deployment}
Un'istanza del backend è ospitata sulla piattaforma render.com ed è disponibile al link \url{https://trentopartecipa.onrender.com}. Il database relazionale invece è attualmente ospitato sulla piattaforma DigitalOcean che grazie all'account education di github abbiamo potuto utilizzare gratuitamente. Il frontend invece è deployato separatamente al link \url{trentopartecipa.me}.

L'utilizzo del piano gratuito (Free Tier) su Render comporta una limitazione prestazionale intrinseca nota come \textbf{Cold Start}: in caso di inattività prolungata (superiore ai 15 minuti), la piattaforma sospende l'istanza per risparmiare risorse. Di conseguenza, la prima richiesta successiva a un periodo di pausa potrebbe richiedere circa 50-60 secondi per il riavvio del servizio; le richieste successive avverranno invece con latenza standard.

Per superare invece le restrizioni funzionali (rete e storage) imposte dalla piattaforma, sono state adottate le seguenti soluzioni architetturali:

\begin{itemize}
    \item \textbf{Migrazione da SMTP a API HTTP (Resend):}
    Poiché Render blocca le porte SMTP in uscita, abbiamo sostituito l'invio tradizionale delle email con la libreria \textbf{Resend}. Questo servizio permette di inviare i codici OTP per il login tramite chiamate API HTTP (REST), aggirando le restrizioni di rete.

    \item \textbf{Verifica del Dominio e Configurazione DNS:}
    Per superare il blocco di sicurezza di Resend (che impedisce l'invio a indirizzi terzi senza verifica), abbiamo registrato il dominio \texttt{trentopartecipa.me} tramite \textit{Namecheap}. 
    Abbiamo quindi configurato i record DNS \textbf{SPF} e \textbf{DKIM} per autorizzare l'invio delle email e i record \textbf{A} per puntare agli indirizzi IP di GitHub Pages, servendo così anche il frontend dal dominio personalizzato.

    \item \textbf{Persistenza dei Media (Cloudinary):}
    Dato che il filesystem di Render è effimero (i file locali vengono persi allo spegnimento dell'istanza), le immagini caricate dagli utenti vengono salvate sul cloud di \textbf{Cloudinary}, memorizzando nel database solo l'URL remoto sicuro.
\end{itemize}

\paragraph{Configurazione CI/CD}
La configurazione CI/CD basata sulle GitHub Actions esegue il deploy del banch main solo quando vengono superati tutti i test con successo. 





\subsection{Accesso docenti}
I docenti potranno testare interamente l'applicazione sia tramite il normale login con google (che permette l'accesso come utente cittadino) sia tramite un accesso hardcoded utilizzando il codice secret. Sarà poi possibile scegliere il tipo di utente nell'applicativo al link: \url{https://trentopartecipa.me/teacher-login}. 

È importante specificare che questo tipo di accesso è a solo scopo di testing, in modo da poter velocizzare la revisione di tutte le funzionalità per utenti che non hanno direttamente accesso al database. 
\begin{center}
    \item \textbf{Secret: } \path{x7K9pLm2Q5nRt8vW4yB1jD6fG3hC0zXs}
\end{center}


