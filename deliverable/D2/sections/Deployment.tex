\section{Deployment}
Un'istanza del backend è ospitata sulla piattaforma render.com ed è disponibile al link \url{https://trentopartecipa.onrender.com}. Il database relazionale invece è attualmente ospitato sulla piattaforma DigitalOcean che grazie all'account education di github abbiamo potuto utilizzare gratuitamente. Il frontend invece è deployato separatamente al link \url{trentopartecipa.me}.
Visto che da qualche mese l'applicativo render.com impedisce l'utilizzo del protocollo SMTP per la versione gratuita, abbiamo risolto il problema in questo modo: 
%da aggiungere

\paragraph{Configurazione CI/CD}
La configurazione CI/CD basata sulle GitHub Actions esegue il deploy del banch main solo quando vengono superati tutti i test con successo. 





\subsection{Accesso docenti}
I docenti potranno testare interamente l'applicazione sia tramite il normale login con google (che permette l'accesso come utente cittadino) sia tramite un accesso hardcoded utilizzando il codice secret. Sarà poi possibile scegliere il tipo di utente nell'applicativo al link: \url{https://trentopartecipa.me/teacher-login}. 

È importante specificare che questo tipo di accesso è a solo scopo di testing, in modo da poter velocizzare la revisione di tutte le funzionalità per utenti che non hanno direttamente accesso al database. 
\begin{center}
    \item \textbf{Secret: } \path{x7K9pLm2Q5nRt8vW4yB1jD6fG3hC0zXs}
\end{center}


