\section{FrontEnd}
Di seguito viene illustrata la parte FrontEnd del progetto.
\textbf{Premessa: } le immagini riportate mostrano l'applicazione con il tema "dark", ma è disponibile anche una versione "light", ottenibile premendo l'icona del sole raffigurata nella parte più a destra dell'header.

\subsection{Dettaglio dell'header}
\begin{figure}[H]
    \centering
    \includegraphics[width=0.9\linewidth]{images/0header.jpeg}
    \caption{Header di un cittadino amministratore autenticato}
    \label{fig:screen_header}
\end{figure}
Nella Figura \ref{fig:screen_header} è riportato l'header presente nelle varie pagine dell'applicazione. Come già specificato nel Desing del FrontEnd (si veda il documento D1), il suo aspetto varia in funzione della tipologia d'utente. Nell'esempio riportato, ad esempio, assumiamo che l'utente sia autenticato e che abbia sia i privilegi da cittadino sia quelli da amministratore. Le altre varianti dell'header sono:
\begin{itemize}
    \item Se l’utente non `e autenticato, visualizza solo le sezioni ”Home” e ”Iniziative” nella barra in alto. Inoltre, al posto della campanella delle notifiche, dell’icona del suo profilo, del suo nome e del pulsante di logout visualizza un pulsante con la scritta ”Accedi”.
    \item Se l'utente autenticato non ha il ruolo di amministratore, non visualizza la sezione "Area Admin".
    \item Se l'utente non è un cittadino, non visualizza la sezione "Dashboard" e la campanella delle notifiche.
\end{itemize}
Cliccando su una delle sezioni, l'utente viene reindirizzato alla pagina dedicata. Se l'utente clicca sulla campanella visualizza le notifiche recenti. 

\subsection{Home Page}
\begin{figure}[H]
    \centering
    \includegraphics[width=0.9\linewidth]{images/1home.jpeg}
    \caption{Home Page di Trento Partecipa}
    \label{fig:screen_home}
\end{figure}
La Figura \ref{fig:screen_home} raffigura la home page di \textbf{\texttt{"TRENTO PARTECIPA"}}. Osserviamo che nell'header, oltre alla lista di sezioni in cui l'utente può navigare, c'è pure il logo e il nome dell'applicazione e una barra di ricerca il cui scopo è quello di permettere la ricerca di iniziative tramite parole chiave. \\Nella home sono inoltre presenti il bilancio partecipativo attualmente in corso, il pulsante per creare una nuova iniziativa e una preview della lista di iniziative, ridotta alle iniziative in corso con più firme raccolte.
\newpage
\begin{itemize}
    \item \textbf{Bilancio partecipativo -} Come da specifiche, il bilancio partecipativo riporta la sua data di scadenza, le varie opzioni votabili, il numero di voti totale (in basso) e per opzione (a lato di ogni opzione), associati alla percentuale sul totale. I pulsanti per votare sono disabilitati per gli utenti che non sono cittadini autenticati. 
    \item \textbf{Pulsante per creare un'iniziativa -} Cliccando il tasto, l'utente cittadino viene condotto alla schermata in Figura \ref{fig:screen_nuova_iniziativa}. Qualora un utente senza quel ruolo provi a votare, viene condotto alla schermata di login (Figura \ref{fig:screen_login}).
    \item \textbf{Iniziative più popolari -} Rispecchiano le caratteristiche generali di tutte le iniziative, discusse successivamente in relazione alla Figura \ref{fig:screen_iniziative}.
\end{itemize}


\subsection{Schermata di login}
\begin{figure}[H]
    \centering
    \includegraphics[width=0.5\linewidth]{images/2login.jpeg}
    \caption{Schermata di login}
    \label{fig:screen_login}
\end{figure}
Nella Figura \ref{fig:screen_login} è raffigurata la schermata di login dell'applicazione, a cui l'utente viene condotto quando clicca sul tasto "Accedi" presente nell'header o, in alcuni casi, quando tenta di compiere delle azioni che possono essere compiute solo da utenti autenticati (ad esempio creare una nuova iniziativa). Date le complicanze introdotte dall'integrazione di SPID/CIE nel sistema, ci siamo limitati a implementare l'accesso tramite Google. Se l'utente sta eseguendo il suo primo accesso, veiene condotto alla schermata di creazione di un nuovo profilo (Figura \ref{fig:screen_nuovo_profilo}), altrimenti viene ricondotto alla home ma con i privilegi aggiornati in base al suo ruolo.

\newpage
\subsection{Schermata di creazione di un nuovo profilo}
\begin{figure}[H]
    \centering
    \includegraphics[width=0.5\linewidth]{images/3.1crea_profilo1.jpeg}
\end{figure}
\begin{figure}[H]
    \centering
    \includegraphics[width=0.5\linewidth]{images/3.2crea_profilo2.jpeg}
    \caption{Le due fasi di creazione di un nuovo profilo}
    \label{fig:screen_nuovo_profilo}
\end{figure}
In Figura \ref{fig:screen_nuovo_profilo} sono raffigurate le due fasi di creazione di un nuovo profilo da parte di un utente. L'utente inserisce il proprio recapito email e attende poi di ricevere un codice OTP da inserire per la verifica di sicurezza. Nella prima fase sono presenti, chiaramente, il box per digitare l'email e il pulsante per l'invio del codice OTP. Nella seconda schermata sono presenti un timer che indica il tempo rimasto prima che il codice scada, i box per l'inserimento delle cifre del codice, il pulsante per confermare la propria registrazione e un pulsante "Ho sbagliato email" per tornare alla fase precedente.


\subsection{Schermata di creazione di una nuova iniziativa}
\begin{figure}[H]
    \centering
    \includegraphics[width=0.9\linewidth]{images/4nuova_iniz.jpeg}
    \caption{Schermata di creazione di una nuova iniziativa}
    \label{fig:screen_nuova_iniziativa}
\end{figure}
In Figura \ref{fig:screen_nuova_iniziativa} è raffigurata la schermata da cui l'utente può compilare i campi per creare e pubblicare una nuova iniziativa. Sono presenti i form per digitare il titolo, l'eventuale luogo e la descrizione, selezionare una categoria e allegare eventualmente delle immagini o dei documenti. In basso a destra sono raffigurati i pulsanti per annullare o confermare la pubblicazione.


\subsection{Schermata di visualizzazione delle iniziative}
\begin{figure}[H]
    \centering
    \includegraphics[width=0.9\linewidth]{images/5lista_iniz.jpeg}
    \caption{Schermata di visualizzazione delle iniziative}
    \label{fig:screen_iniziative}
\end{figure}
In Figura \ref{fig:screen_iniziative} è raffigurata la pagina sotto la voce "Iniziative", accessibile da qualsiasi utente. In essa sono presenti le preview delle varie iniziative presenti in piattaforma e un box dove l'utente può applicare dei parametri di ricerca per visualizzare distintamente un certo sottoinsieme di iniziative.

\newpage
\subsubsection*{Dettaglio preview di un'iniziativa}
\begin{figure}[H]
    \centering
    \includegraphics[width=0.6\linewidth]{images/5.2dettaglio_preview_iniz.jpeg}
\end{figure}
Ogni preview di iniziativa riporta un titolo, la categoria, la data di pubblicazione, il numero di firme raccolte ed eventualmente un'immagine e il luogo. Se l'iniziativa è stata importata da una piattaforma esterna, questa viene indicata esplicitamente (si vedano ad esempio le preview presenti nella home in Figura \ref{fig:screen_home}). La preview presenta inoltra una label che ne indica lo stato (ad esempio "In corso") e un pulsante per seguirla, ossia aggiungerla alla dashboard personale. Se un utente non cittadino prova a seguire un'iniziativa, viene condotto alla schermata da cui fare il login (Figura \ref{fig:screen_login}). Cliccando sulla preview, si apre la pagina di dettaglio dell'iniziativa (Figura \ref{fig:screen_dettaglio_iniziativa}).

\subsubsection*{Dettaglio selezione dei criteri di ricerca}
\begin{figure}[H]
    \centering
    \includegraphics[width=0.3\linewidth]{images/5.1dettaglio_filtri.jpeg}
\end{figure}
Oltre alla ricerca per parole chiave, possibile grazie alla barra di ricerca nell'header, l'utente può applicare anche dei criteri di ricerca tramite l'interfaccia in figura: è possibile filtrare le iniziative per stato, piattaforma, categoria, intervallo temporale (riferito alla data di creazione delle iniziative) e ordinarle in base al valore di certi parametri. I filtri possono essere rimossi manualmente o con un apposito pulsante "Azzera filtri".

\subsection{Schermata di visualizzazione della singola iniziativa}
\begin{figure}[H]
    \centering
    \includegraphics[width=0.8\linewidth]{images/6.1singola_iniz.jpeg}
    \caption{Schermata di visualizzazione di una singola iniziativa}
    \label{fig:screen_dettaglio_iniziativa}
\end{figure}
In Figura \ref{fig:screen_dettaglio_iniziativa} è raffigurata la pagina di dettaglio di un'iniziativa. Come da specifiche riporta titolo, stato, categoria, eventuale luogo, data di creazione e di scadenza, autore, descrizione, numero di firme, eventuali immagini ed allegati e, sotto la descrizione, la risposta del Comune, se presente (Figura \ref{fig:screen_risposta_pubblicata}). Sono poi presenti un tasto per copiare il link dell'iniziativa, in caso l'utente volesse condividerla, e il pulsante per firmarla. Se un utente non cittadino tenta di firmare, viene condotto alla schermata di login (Figura \ref{fig:screen_login}). Per tornare alla lista di preview, l'utente può cliccare su "Torna Indietro", in alto a sinistra. Se l'iniziativa è stata importata da una piattaforma esterna, il pulsante per firmare conduce l'utente alla piattaforma di provenienza, ed ha il seguente aspetto:
\begin{figure}[H]
    \centering
    \includegraphics[width=0.2\linewidth]{images/6.2singola_iniz_esterna.jpeg}
\end{figure}

\subsection{Dashboard personale}
\begin{figure}[H]
    \centering
    \includegraphics[width=0.9\linewidth]{images/7.1dashboard_iniziative.jpeg}
\end{figure}
\begin{figure}[H]
    \centering
    \includegraphics[width=0.9\linewidth]{images/7.2_dashboard_notifiche.jpeg}
    \caption{Dashboard personale}
    \label{fig:screen_dashboard}
\end{figure}
In Figura \ref{fig:screen_dashboard} sono raffigurate le schermate della dashboard personale di un cittadino, dalla quale può visualizzare le iniziative da lui create, firmate e seguite, oltre alle notifiche che gli sono giunte (visualizzabili anche cliccando sull'icona della campanella nell'header).

\newpage
\subsection{Area Admin}
\begin{figure}[H]
    \centering
    \includegraphics[width=0.9\linewidth]{images/8areaadmin.jpeg}
    \caption{Area Admin}
    \label{fig:screen_area_admin}
\end{figure}
In Figura \ref{fig:screen_area_admin} è raffigurata l'area admin, accessibile dagli utenti amministratori. L'area admin consiste in un pannello di controllo dal quale l'utente può scegliere che attività svolgere. Selezionando una tra le quattro aree l’utente sarà ricondotto alla pagina dedicata a quell’attività. Le pagine in questione sono descritte di seguito (Figure \ref{fig:screen_monitoraggio}, \ref{fig:screen_creaBP}, \ref{fig:screen_gestione_personale}, \ref{fig:screen_archivioBP}).

\subsection{Schermata di monitoraggio delle iniziative}
\begin{figure}[H]
    \centering
    \includegraphics[width=0.9\linewidth]{images/9monitoraggio_iniz.jpeg}
    \caption{Schermata di monitoraggio delle iniziative}
    \label{fig:screen_monitoraggio}
\end{figure}
In Figure \ref{fig:screen_monitoraggio} è raffigurata la schermata da cui l'utente amministratore può monitorare le iniziative in corso, in attesa di risposta. Ogni iniziativa riporta un timer che indica il tempo rimasto prima che scada. Sono inoltre presenti, per ogni iniziativa, un pulsante per prorogarne la data di scadenza di 60 giorni e un pulsante per visualizzarla e scrivere una risposta (Figura \ref{fig:screen_risposta}).

\subsection{Schermata di compilazione di una risposta del Comune}
\begin{figure}[H]
    \centering
    \includegraphics[width=0.9\linewidth]{images/10.1risposta.jpeg}
    \caption{Schermata di compilazione di una risposta del Comune}
    \label{fig:screen_risposta}
\end{figure}
In Figura \ref{fig:screen_risposta} è raffigurata la schermata da cui l'utente amministratore può compilare una risposta per un'iniziativa. Il box con i campi da compilare appare all'amministratore sotto le informazioni presenti nella pagina di dettaglio dell'iniziativa selezionata. L'amministratore deve selezionare l'esito finale dell'iniziativa ("Approvata" o "Respinta"), scrivere una motivazione ufficiale e allegare eventualmente dei documenti. Tramite un apposito pulsante può poi pubblicarla. Anche qui è presente un pulsante per prorogare la data di scadenza dell'iniziativa. 
\newpage
Una volta pubblicata, la risposta apparirà agli altri utenti nella pagina di dettaglio dell'iniziativa (Figura \ref{fig:screen_dettaglio_iniziativa}), come nella figura sotto:
\begin{figure}[H]
    \centering
    \includegraphics[width=0.9\linewidth]{images/10.2singola_inizi_con_risposta.jpeg}
    \caption{Risposta del Comune all'iniziativa}
    \label{fig:screen_risposta_pubblicata}
\end{figure}

\subsection{Schermata di creazione del bilancio partecipativo}
\begin{figure}[H]
    \centering
    \includegraphics[width=0.9\linewidth]{images/11crea_bp.jpeg}
    \caption{Schermata di creazione del bilancio partecipativo}
    \label{fig:screen_creaBP}
\end{figure}
In Figura \ref{fig:screen_creaBP} è raffigurata la schermata da cui l'utente amministratore può compilare i campi relativi a un nuovo bilancio partecipativo che intende pubblicare: questi campi, digitabili nei form, includono il titolo, la data di scadenza e un certo numero di opzioni compreso tra 3 e 5. Un apposito pulsante permette di aggiungere un'opzione, se possibile. Sono poi presenti dei pulsanti per annullare o confermare l'operazione.

\subsection{Schermata di gestione del personale}
\begin{figure}[H]
    \centering
    \includegraphics[width=0.9\linewidth]{images/12.1lista_amministratori.jpeg}
    \caption{Schermata di gestione del personale}
    \label{fig:screen_gestione_personale}
\end{figure}
In Figura \ref{fig:screen_gestione_personale} è raffigurata la schermata dalla quale l'utente amministratore può visualizzare la lista completa degli utenti amministratori con le loro informazioni (cognome, nome, codice fiscale, email). Una barra di ricerca permette di digitare un codice fiscale per verificarne la presenza nell'elenco. Per ogni utente nella lista, ad esclusione dell'utente stesso, è presente un pulsante "Rimuovi" che serve per revocare i privilegi da amministratore ad un certo utente. Per aggiungere un nuovo amministratore (promuovendo un cittadino o creando una pre-autorizzazione), l'utente può cliccare il pulsante raffigurante il \textbf{"+}, in alto a destra: così facendo apparirà un box (come nella figura sotto) dove l'utente potrà digitare il codice fiscale dell'utente che intende promuovere.
\begin{figure}[H]
    \centering
    \includegraphics[width=0.5\linewidth]{images/12.2ricerca_cf.jpeg}
\end{figure}

\subsection{Archivio dei bilanci partecipativi}
\begin{figure}[H]
    \centering
    \includegraphics[width=0.9\linewidth]{images/13.1archivio_bp.jpeg}
    \caption{Archivio dei bilanci partecipativi}
    \label{fig:screen_archivioBP}
\end{figure}
In Figura \ref{fig:screen_archivioBP} è raffigurata la schermata da cui l'utente amministratore può consultare l'archivio dei bilanci partecipativi conclusi. Ogni bilancio partecipativo riporta titolo, data di scadenza, numero di opzioni e un pulsante per visualizzarne i risultati. Cliccandolo si apre una schermata come la seguente:
\begin{figure}[H]
    \centering
    \includegraphics[width=0.7\linewidth]{images/13.2risultati_bp.jpeg}
    \caption{Risultati di un bilancio partecipativo concluso}
    \label{fig:screen_risultati}
\end{figure}
In Figura \ref{fig:screen_risultati} sono raffigurati i risultati di un bilancio partecipativo concluso: vengono riportati il titolo del sondaggio, la data di scadenza, il numero totale dei voti e le opzioni. A fianco ad ogni opzione sono riportati il numero di voti e la percentuale sul totale, rispecchiata dalla barra verde sotto l'opzione. Per tornare all'archivio, l'utente può cliccare sul pulsante in basso a sinistra.
