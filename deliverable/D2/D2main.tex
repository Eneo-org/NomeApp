\documentclass[11pt]{extarticle}
\usepackage{graphicx} % Required for inserting images
\usepackage{float}
\usepackage[italian]{babel}
\usepackage[a4paper, left=3cm, right=3cm]{geometry}
\usepackage{titlesec}
\usepackage{hyperref}

%-- gestione estetica codice yaml
\usepackage{minted}
\usepackage{xcolor}


\usepackage[utf8]{inputenc}
\usepackage[T1]{fontenc}
\usepackage[italian]{babel}

\usepackage{xltabular} 
\usepackage{ragged2e} 
\usepackage{longtable}
\usepackage{array} % Serve per definire meglio le colonne
% Definiamo un nuovo tipo di colonna "Y" che è come la "X" 
% ma allineata a sinistra (evita spazi enormi tra le parole)
% Permette di spezzare stringhe lunghe (URL, codice, variabili) ovunque
\usepackage{xurl}
\usepackage{dirtree}
\newcolumntype{L}{>{\hsize=1.9\hsize\raggedright\arraybackslash}X} % Large (Precondizioni)
\newcolumntype{M}{>{\hsize=0.9\hsize\raggedright\arraybackslash}X} % Medium (Descrizione)
\newcolumntype{S}{>{\hsize=0.72\hsize\raggedright\arraybackslash}X} % Small (Dati e Risultati)

% 2. FONT PIÙ PICCOLO
% Impostiamo il font della tabella a \footnotesize o \scriptsize per far entrare il testo
\footnotesize
% Configurazione estetica
\definecolor{bg}{rgb}{0.95,0.95,0.95} % Sfondo grigio chiaro

% Opzioni globali per minted
\setminted{
    style=friendly,      % Stile colori (prova anche 'monokai', 'manni', 'vs')
    bgcolor=bg,          % Colore di sfondo
    frame=lines,         % Linee sopra e sotto
    framesep=2mm,        % Spazio tra frame e codice
    fontsize=\footnotesize,
    linenos=true         % Numeri di riga
}

% --- path per repo organization


% --- Section format ---
\titleformat{\section}
  {\normalfont\LARGE\bfseries}  % Font style: normal + Large + bold
  {\thesection}                 % Section number format
  {6pt}                         % Space between number and title
  {}

\title{\Huge TRENTO PARTECIPA - D2}
\author{\LARGE D'Angiò Enea, Mattarolo Alessandro, Nedeljkovic Ivan}
\date{January 2026}

\begin{document}

\maketitle

\tableofcontents
\newpage

\section{Introduzione}
Il seguente documento riporta le informazioni relative all'\textbf{implementazione} del progetto \textbf{\texttt{"TRENTO PARTECIPA"}}. 
\newline
\newline
È importante premettere che alcune funzionalità descritte nel documento D1 non sono state completamente implementate per vari motivi, principalmente legati a fattori di semplicità e tempistiche. In particolare:
\begin{itemize}
    \item Sebbene i requisiti funzionali prevedano l'utilizzo di SPID o CIE per il login, in questa fase di sviluppo abbiamo optato per sostituirli con l'accesso tramite Google. Tale scelta è motivata principalmente da vincoli di natura burocratica: l'integrazione in ambiente di produzione di SPID/CIE richiede infatti che l'ente gestore dell'applicazione ottenga la qualifica formale di Service Provider accreditato presso l'AgID o il Ministero dell'Interno. Questo iter comporta oneri economici, tempistiche burocratiche e requisiti legali non gestibili dal nostro gruppo. Abbiamo quindi deciso di usare l'accesso tramite Google, per simulare comunque l'accesso tramite un identity provider esterno, mentre l'integrazione con SPID/CIE potrà eventualmente essere realizzata in futuro. La conseguenza di questa scelta, in termini di funzionalità, è il fatto che non è più possibile verificare la residenza di un utente durante l'atto di creazione di un profilo. Dunque, a meno di operazioni hard-coded e cambiamenti manuali nel database, qualunque utente non pre-autorizzato che fa il primo accesso all'applicativo ottiene il ruolo di "Cittadino".
    \item Non è stato implementato l'algoritmo di controllo duplicati nella fase di creazione di un'iniziativa.
    \item dati esterni
    \item rnf
\end{itemize}


\section{Web APIs}
Le specifiche delle API sono state documentate secondo lo standard OpenAPI 3.0.1 e la documentazione è consultabile pubblicamente al seguente indirizzo: 
\url{https://380vqprxjy.apidog.io}

\paragraph{Progettazione e scelte architetturali}
L'architettura segue i principi RESTful, organizzando le risorse in modo gerarchico. In fase di progettazione è stato deciso di standardizzare le risposte di errore, ottimizzare la gestione delle collezioni (tramite l'utilizzo di metadati per distinguere la pagina ed il numero di elementi necessari) e utilizzare ampiamente schemi ed ereditarietà dei componenti (abbiamo per esempio creato versioni standard e dettagliate per vari schemi in modo da ridurre i dati inutilizzati). Abbiamo inoltre fatto attenzione nell'aggiungere regole di validazione all'interno schemi delle api, come per esempio l'uso di enum per gli stati delle iniziative.

Le specifiche delle APIs sono disponibili al seguente indirizzo:
\url{https://github.com/Eneo-org/NomeApp/blob/main/server/APIsDocumentazione/AllAPIs.yaml} 
e sono riportate di seguito: 

\begin{minted}{yaml}
openapi: 3.0.1
info:
  title: Default module
  description: ''
  version: 1.0.0
tags:
  - name: Initiatives
  - name: ParticipatoryBudgets
  - name: Users
  - name: Users/Auth
  - name: Filters
paths:
  /initiatives:
    get:
      summary: initiativesList
      deprecated: false
      description: >-
        Restituisce la lista delle iniziative. Supporta parametri per
        l'applicazione di filtri alla ricerca, per la ricerca testuale e
        l'ordinamento.
      tags:
        - Initiatives
      parameters:
        - name: currentPage
          in: query
          description: Numero della pagina per la paginazione.
          required: false
          example: 1
          schema:
            type: integer
        - name: objectsPerPage
          in: query
          description: Numero di elementi per pagina.
          required: false
          example: 10
          schema:
            type: integer
        - name: status
          in: query
          description: Filtro per lo stato dell'iniziativa.
          required: false
          example: In corso
          schema:
            type: string
        - name: search
          in: query
          description: Ricerca libera nel titolo, descrizione e luogo.
          required: false
          example: Parco
          schema:
            type: string
        - name: categoryID
          in: query
          description: ID univoco della categoria (filtro per categoria).
          required: false
          example: Sport
          schema:
            type: integer
        - name: sortBy
          in: query
          description: Codice relativo ad un attributo su cui bisogna ordinare.
          required: false
          example: 1
          schema:
            type: integer
        - name: order
          in: query
          description: La direzione dell'ordinamento (asc o desc).
          required: false
          example: desc
          schema:
            type: string
            enum:
              - asc
              - desc
        - name: minSignatures
          in: query
          description: Limite inferiore di firme da mostrare.
          required: false
          example: 0
          schema:
            type: integer
        - name: maxSignatures
          in: query
          description: Limite superiore di firme da mostrare.
          required: false
          example: 1000
          schema:
            type: integer
      responses:
        '200':
          description: ''
          content:
            application/json:
              schema:
                type: object
                properties:
                  data:
                    type: array
                    items:
                      $ref: '#/components/schemas/Initiative'
                    minItems: 100
                    maxItems: 100
                  meta:
                    $ref: '#/components/schemas/MetaDataList'
                required:
                  - data
                  - meta
          headers: {}
        '400':
          description: ''
          content:
            application/json:
              schema:
                $ref: '#/components/schemas/GenericError'
          headers: {}
      security: []
    post:
      summary: createInitiative
      deprecated: false
      description: Permette a un cittadino autenticato di creare una nuova iniziativa.
      tags:
        - Initiatives
      parameters:
        - name: X-Mock-User-Id
          in: header
          description: Simula l'utente loggato
          required: true
          example: 1
          schema:
            type: integer
      requestBody:
        content:
          multipart/form-data:
            schema:
              type: object
              properties:
                title:
                  description: Titolo dell'iniziativa.
                  type: string
                  example: ''
                description:
                  description: Descrizione dettagliata della proposta.
                  type: string
                  example: ''
                place:
                  description: Luogo o area di interesse.
                  type: string
                  example: ''
                categoryId:
                  type: integer
                  description: ID univoco della categoria.
                  example: 0
                attachments:
                  format: binary
                  type: string
                  description: Eventuali allegati (Immagini o PDF).
                  example: ''
        required: true
      responses:
        '201':
          description: ''
          content:
            application/json:
              schema:
                $ref: '#/components/schemas/DetailedInitiative'
          headers: {}
        '400':
          description: ''
          content:
            application/json:
              schema:
                $ref: '#/components/schemas/GenericError'
          headers: {}
        '401':
          description: ''
          content:
            application/json:
              schema:
                $ref: '#/components/schemas/GenericError'
          headers: {}
        '403':
          description: ''
          content:
            application/json:
              schema:
                $ref: '#/components/schemas/GenericError'
          headers: {}
        '409':
          description: ''
          content:
            application/json:
              schema:
                type: object
                properties: {}
          headers: {}
        '422':
          description: ''
          content:
            application/json:
              schema:
                $ref: '#/components/schemas/GenericError'
          headers: {}
      security: []
  /initiatives/{id}:
    get:
      summary: initiativeDetails
      deprecated: false
      description: >-
        Restituisce i dettagli completi di una singola iniziativa selezionata
        dall'utente.
      tags:
        - Initiatives
      parameters:
        - name: id
          in: path
          description: ID univoco dell'iniziativa.
          required: true
          example: 0
          schema:
            type: integer
      responses:
        '200':
          description: ''
          content:
            application/json:
              schema:
                $ref: '#/components/schemas/DetailedInitiative'
          headers: {}
        '400':
          description: ''
          content:
            application/json:
              schema:
                $ref: '#/components/schemas/GenericError'
          headers: {}
        '404':
          description: ''
          content:
            application/json:
              schema:
                $ref: '#/components/schemas/GenericError'
          headers: {}
      security: []
    patch:
      summary: changeExpirationDate
      deprecated: false
      description: >-
        Permette all'amministrazione di prorogare la data di scadenza di
        un'iniziativa.
      tags:
        - Initiatives
      parameters:
        - name: id
          in: path
          description: ID univoco dell'iniziativa.
          required: true
          example: 0
          schema:
            type: integer
        - name: X-Mock-User-Id
          in: header
          description: ''
          required: true
          example: 0
          schema:
            type: integer
      requestBody:
        content:
          application/json:
            schema:
              type: object
              properties:
                expirationDate:
                  type: string
                  format: date
                  description: Nuova data di scadenza.
              required:
                - expirationDate
        required: true
      responses:
        '200':
          description: ''
          content:
            application/json:
              schema:
                $ref: '#/components/schemas/DetailedInitiative'
          headers: {}
        '401':
          description: ''
          content:
            application/json:
              schema:
                $ref: '#/components/schemas/GenericError'
          headers: {}
        '403':
          description: ''
          content:
            application/json:
              schema:
                $ref: '#/components/schemas/GenericError'
          headers: {}
        '404':
          description: ''
          content:
            application/json:
              schema:
                $ref: '#/components/schemas/GenericError'
          headers: {}
        '422':
          description: ''
          content:
            application/json:
              schema:
                $ref: '#/components/schemas/GenericError'
          headers: {}
      security: []
  /initiatives/{id}/responses:
    post:
      summary: createReply
      deprecated: false
      description: >-
        Permette all'amministrazione di inviare una risposta ufficiale a
        un'iniziativa, cambiandone lo stato da 'In corso' a 'Respinta' oppure
        'Approvata'.
      tags:
        - Initiatives
      parameters:
        - name: id
          in: path
          description: ID univoco dell'iniziativa.
          required: true
          example: 0
          schema:
            type: integer
        - name: X-Mock-User-Id
          in: header
          description: ''
          required: true
          example: 0
          schema:
            type: integer
      requestBody:
        content:
          multipart/form-data:
            schema:
              type: object
              properties:
                status:
                  description: Nuovo stato dell'iniziativa ('Respinta' o 'Approvata').
                  type: string
                  example: ''
                motivations:
                  description: Testo di esposizione delle motivazioni della risposta.
                  type: string
                  example: ''
                attachments:
                  format: binary
                  type: string
                  description: Eventuali allegati (immagini o PDF).
                  example: ''
        required: true
      responses:
        '201':
          description: ''
          content:
            application/json:
              schema:
                $ref: '#/components/schemas/Reply'
          headers: {}
        '400':
          description: ''
          content:
            application/json:
              schema:
                $ref: '#/components/schemas/GenericError'
          headers: {}
        '401':
          description: ''
          content:
            application/json:
              schema:
                $ref: '#/components/schemas/GenericError'
          headers: {}
        '403':
          description: ''
          content:
            application/json:
              schema:
                $ref: '#/components/schemas/GenericError'
          headers: {}
        '409':
          description: ''
          content:
            application/json:
              schema:
                $ref: '#/components/schemas/GenericError'
          headers: {}
      security: []
  /initiatives/{id}/signatures:
    post:
      summary: signInitiative
      deprecated: false
      description: Permette a un cittadino di firmare un'iniziativa.
      tags:
        - Initiatives
      parameters:
        - name: id
          in: path
          description: ID univoco dell'iniziativa.
          required: true
          example: 0
          schema:
            type: integer
        - name: X-Mock-User-Id
          in: header
          description: ''
          required: true
          example: 0
          schema:
            type: integer
      responses:
        '201':
          description: ''
          content:
            application/json:
              schema:
                $ref: '#/components/schemas/InitiativeSignature'
          headers: {}
        '401':
          description: ''
          content:
            application/json:
              schema:
                $ref: '#/components/schemas/GenericError'
          headers: {}
        '403':
          description: ''
          content:
            application/json:
              schema:
                $ref: '#/components/schemas/GenericError'
          headers: {}
        '404':
          description: ''
          content:
            application/json:
              schema:
                $ref: '#/components/schemas/GenericError'
          headers: {}
        '409':
          description: ''
          content:
            application/json:
              schema:
                $ref: '#/components/schemas/GenericError'
          headers: {}
      security: []
  /initiatives/{id}/follows:
    post:
      summary: saveIniziative
      deprecated: false
      description: >-
        Salva un'iniziativa nella lista delle iniziative seguite dall'utente
        affinché possa monitorarne lo stato.
      tags:
        - Initiatives
      parameters:
        - name: id
          in: path
          description: ID univoco dell'iniziativa.
          required: true
          example: 0
          schema:
            type: integer
        - name: X-Mock-User-Id
          in: header
          description: ''
          required: true
          example: 0
          schema:
            type: integer
      responses:
        '200':
          description: ''
          content:
            application/json:
              schema:
                $ref: '#/components/schemas/SavedInitiative'
          headers: {}
        '401':
          description: ''
          content:
            application/json:
              schema:
                $ref: '#/components/schemas/GenericError'
          headers: {}
        '404':
          description: ''
          content:
            application/json:
              schema:
                $ref: '#/components/schemas/GenericError'
          headers: {}
        '409':
          description: ''
          content:
            application/json:
              schema:
                $ref: '#/components/schemas/GenericError'
          headers: {}
      security: []
    delete:
      summary: removeFromSavedInitiatives
      deprecated: false
      description: Rimuove l'iniziativa dalla lista delle iniziative seguite.
      tags:
        - Initiatives
      parameters:
        - name: id
          in: path
          description: ID univoco dell'iniziativa.
          required: true
          example: 0
          schema:
            type: integer
        - name: X-Mock-User-Id
          in: header
          description: ''
          required: true
          example: 0
          schema:
            type: integer
      responses:
        '200':
          description: ''
          content:
            application/json:
              schema:
                type: object
                properties:
                  message:
                    type: string
                  initiativeId:
                    type: integer
                required:
                  - message
                  - initiativeId
          headers: {}
        '401':
          description: ''
          content:
            application/json:
              schema:
                type: object
                properties: {}
          headers: {}
        '404':
          description: ''
          content:
            application/json:
              schema:
                type: object
                properties: {}
          headers: {}
      security: []
  /initiatives/admin/expiring:
    get:
      summary: getExpiringInitiatives
      deprecated: false
      description: >-
        Restituisce le iniziative in scadenza entro 7 giorni. Solo per
        amministratori.
      tags:
        - Initiatives
      parameters:
        - name: currentPage
          in: query
          description: Numero della pagina per la paginazione.
          required: false
          example: 1
          schema:
            type: integer
        - name: objectsPerPage
          in: query
          description: Numero di elementi per pagina.
          required: false
          example: 5
          schema:
            type: integer
        - name: X-Mock-User-Id
          in: header
          description: Simula l'utente loggato (deve essere admin)
          required: true
          example: 1
          schema:
            type: integer
      responses:
        '200':
          description: Lista delle iniziative in scadenza.
          content:
            application/json:
              schema:
                type: object
                properties:
                  data:
                    type: array
                    items:
                      $ref: '#/components/schemas/Initiative'
                  meta:
                    $ref: '#/components/schemas/MetaDataList'
                required:
                  - data
                  - meta
          headers: {}
        '401':
          description: Utente non autenticato.
          content:
            application/json:
              schema:
                $ref: '#/components/schemas/GenericError'
          headers: {}
        '403':
          description: L'utente non ha privilegi di amministratore.
          content:
            application/json:
              schema:
                $ref: '#/components/schemas/GenericError'
          headers: {}
        '500':
          description: Errore del server.
          content:
            application/json:
              schema:
                $ref: '#/components/schemas/GenericError'
          headers: {}
      security: []
  /initiatives/cooldown:
    get:
      summary: checkCooldown
      deprecated: false
      description: >-
        Verifica se l'utente può creare una nuova iniziativa (controlla il
        periodo di cooldown di 14 giorni dall'ultima iniziativa creata).
      tags:
        - Initiatives
      parameters:
        - name: X-Mock-User-Id
          in: header
          description: Simula l'utente loggato
          required: true
          example: 1
          schema:
            type: integer
      responses:
        '200':
          description: Informazioni sul cooldown.
          content:
            application/json:
              schema:
                type: object
                properties:
                  canCreate:
                    type: boolean
                    description: Indica se l'utente può creare una nuova iniziativa.
                  daysRemaining:
                    type: integer
                    description: >-
                      Numero di giorni rimanenti prima di poter creare una nuova
                      iniziativa.
                    nullable: true
                  lastInitiativeDate:
                    type: string
                    format: date
                    description: Data dell'ultima iniziativa creata.
                    nullable: true
                required:
                  - canCreate
          headers: {}
        '401':
          description: Utente non autenticato.
          content:
            application/json:
              schema:
                $ref: '#/components/schemas/GenericError'
          headers: {}
        '500':
          description: Errore del server.
          content:
            application/json:
              schema:
                $ref: '#/components/schemas/GenericError'
          headers: {}
      security: []
  /participatory-budgets:
    post:
      summary: createParticipatoryBudget
      deprecated: false
      description: Creazione di un nuovo bilancio partecipativo.
      tags:
        - ParticipatoryBudgets
      parameters:
        - name: X-Mock-User-Id
          in: header
          description: ''
          required: true
          example: 0
          schema:
            type: integer
      requestBody:
        content:
          application/json:
            schema:
              $ref: '#/components/schemas/ParticipatoryBudget'
        required: true
      responses:
        '201':
          description: ''
          content:
            application/json:
              schema:
                type: object
                properties:
                  options:
                    type: array
                    items:
                      $ref: '#/components/schemas/PBOption'
                    minItems: 3
                    maxItems: 3
                    description: Array di opzioni.
                  id:
                    type: integer
                    description: ID univoco del bilancio partecipativo.
                  creatorId:
                    type: integer
                    description: ID univoco del creatore del bilancio partecipativo.
                  title:
                    type: string
                    description: Titolo del bilancio partecipativo.
                  createdAt:
                    type: string
                    format: date-time
                    description: Data di creazione.
                  expirationDate:
                    type: string
                    format: date
                    description: Data di scadenza.
                  votedOptionId:
                    type: integer
                    description: ID dell'opzione votata.
                required:
                  - votedOptionId
                  - options
                  - id
                  - creatorId
                  - title
                  - createdAt
                  - expirationDate
          headers: {}
        '400':
          description: ''
          content:
            application/json:
              schema:
                type: object
                properties: {}
          headers: {}
        '401':
          description: ''
          content:
            application/json:
              schema:
                $ref: '#/components/schemas/GenericError'
          headers: {}
        '403':
          description: ''
          content:
            application/json:
              schema:
                $ref: '#/components/schemas/GenericError'
          headers: {}
        '409':
          description: ''
          content:
            application/json:
              schema:
                $ref: '#/components/schemas/GenericError'
          headers: {}
      security: []
    get:
      summary: participatoryBudgetsArchive
      deprecated: false
      description: Restituisce l'archivio storico dei bilanci partecipativi conclusi.
      tags:
        - ParticipatoryBudgets
      parameters:
        - name: currentPage
          in: query
          description: Numero della pagina per la paginazione.
          required: false
          example: 1
          schema:
            type: integer
        - name: objectsPerPage
          in: query
          description: Numero di elementi per pagina.
          required: false
          example: 10
          schema:
            type: integer
        - name: X-Mock-User-id
          in: header
          description: ''
          required: true
          example: 0
          schema:
            type: integer
      responses:
        '200':
          description: ''
          content:
            application/json:
              schema:
                type: object
                properties:
                  data:
                    type: array
                    items:
                      $ref: '#/components/schemas/ParticipatoryBudget'
                    minItems: 1
                    maxItems: 5
                  meta:
                    $ref: '#/components/schemas/MetaDataList'
                required:
                  - data
                  - meta
          headers: {}
        '401':
          description: ''
          content:
            application/json:
              schema:
                $ref: '#/components/schemas/GenericError'
          headers: {}
        '403':
          description: ''
          content:
            application/json:
              schema:
                $ref: '#/components/schemas/GenericError'
          headers: {}
      security: []
  /participatory-budgets/active:
    get:
      summary: participatoryBudgetDetails
      deprecated: false
      description: Restituisce i dettagli di un bilancio partecipativo.
      tags:
        - ParticipatoryBudgets
      parameters:
        - name: status
          in: query
          description: Filtro per stato (attivo o non).
          required: false
          schema:
            type: string
            enum:
              - active
              - notActive
        - name: X-Mock-User-Id
          in: header
          description: ''
          required: true
          example: 0
          schema:
            type: integer
      responses:
        '200':
          description: ''
          content:
            application/json:
              schema:
                type: object
                properties:
                  votedOptionId:
                    type: integer
                    description: >-
                      Identifica l'opzione che l'utente ha votato se ha già
                      votato. 
                    nullable: true
                  options:
                    type: array
                    items:
                      $ref: '#/components/schemas/PBOption'
                    minItems: 3
                    maxItems: 3
                    description: Array di opzioni.
                  id:
                    type: integer
                    description: ID univoco del bilancio partecipativo.
                  creatorId:
                    type: integer
                    description: ID univoco del creatore del bilancio partecipativo.
                  title:
                    type: string
                    description: Titolo del bilancio partecipativo.
                  createdAt:
                    type: string
                    format: date-time
                    description: Data di creazione.
                  expirationDate:
                    type: string
                    format: date
                    description: Data di scadenza.
                required:
                  - options
                  - id
                  - creatorId
                  - title
                  - createdAt
                  - expirationDate
          headers: {}
      security: []
  /participatory-budgets/{id}/votes:
    post:
      summary: voteParticipatoryBudget
      deprecated: false
      description: >-
        Permette a un cittadino di esprimere il proprio voto per una delle
        opzioni del bilancio partecipativo.
      tags:
        - ParticipatoryBudgets
      parameters:
        - name: id
          in: path
          description: ID univoco del bilancio partecipativo.
          required: true
          example: 0
          schema:
            type: integer
        - name: X-Mock-User-Id
          in: header
          description: ''
          required: true
          example: 0
          schema:
            type: integer
      requestBody:
        content:
          application/json:
            schema:
              type: object
              properties:
                position:
                  type: integer
                  minimum: 2
                  maximum: 5
                  description: Posizione dell'opzione scelta.
              required:
                - position
        required: true
      responses:
        '200':
          description: ''
          content:
            application/json:
              schema:
                $ref: '#/components/schemas/ParticipatoryBudget'
          headers: {}
        '400':
          description: ''
          content:
            application/json:
              schema:
                $ref: '#/components/schemas/GenericError'
          headers: {}
        '401':
          description: ''
          content:
            application/json:
              schema:
                $ref: '#/components/schemas/GenericError'
          headers: {}
        '403':
          description: ''
          content:
            application/json:
              schema:
                $ref: '#/components/schemas/GenericError'
          headers: {}
        '404':
          description: ''
          content:
            application/json:
              schema:
                $ref: '#/components/schemas/GenericError'
          headers: {}
        '409':
          description: ''
          content:
            application/json:
              schema:
                $ref: '#/components/schemas/GenericError'
          headers: {}
      security: []
  /auth/google:
    post:
      summary: loginAndAuth
      deprecated: false
      description: >-
        Autenticazione utente tramite provider esterno. Verifica l'identità e
        restituisce il token di sessione.
      tags:
        - Users/Auth
      parameters: []
      requestBody:
        content:
          application/json:
            schema:
              type: object
              properties:
                tokenId:
                  type: string
                  description: Token fornito dal provider di identità.
              required:
                - tokenId
        required: true
      responses:
        '200':
          description: ''
          content:
            application/json:
              schema:
                $ref: '#/components/schemas/AuthResponse'
          headers: {}
        '400':
          description: ''
          content:
            application/json:
              schema:
                $ref: '#/components/schemas/GenericError'
          headers: {}
        '401':
          description: ''
          content:
            application/json:
              schema:
                $ref: '#/components/schemas/GenericError'
          headers: {}
        '404':
          description: ''
          content:
            application/json:
              schema:
                title: ''
                type: object
                properties:
                  status:
                    type: string
                  googleData:
                    type: object
                    properties:
                      firstName:
                        type: string
                      lastName:
                        type: string
                      email:
                        type: string
                      fiscalCode:
                        type: string
                    required:
                      - firstName
                      - lastName
                      - email
                      - fiscalCode
                required:
                  - status
                  - googleData
          headers: {}
        '500':
          description: ''
          content:
            application/json:
              schema:
                $ref: '#/components/schemas/GenericError'
          headers: {}
      security: []
  /auth/logout:
    post:
      summary: logout
      deprecated: false
      description: Termina la sessione dell'utente corrente.
      tags:
        - Users/Auth
      parameters: []
      responses:
        '200':
          description: ''
          content:
            application/json:
              schema:
                type: object
                properties:
                  message:
                    type: string
                required:
                  - message
          headers: {}
      security: []
  /auth/otp:
    post:
      summary: richiestaOTP
      deprecated: false
      description: >-
        Invia un codice OTP (One-Time Password) via email per la verifica
        dell'identità durante la registrazione di nuovi utenti.
      tags:
        - Users/Auth
      parameters: []
      requestBody:
        content:
          application/json:
            schema:
              type: object
              properties:
                email:
                  type: string
              required:
                - email
        required: true
      responses:
        '201':
          description: ''
          content:
            application/json:
              schema:
                type: object
                properties:
                  message:
                    type: string
                required:
                  - message
          headers: {}
      security: []
  /auth/register:
    post:
      summary: userRegistration
      deprecated: false
      description: >-
        Registrazione di un nuovo utente (Cittadino o Amministratore
        pre-autorizzato) dopo la verifica OTP.
      tags:
        - Users/Auth
      parameters: []
      requestBody:
        content:
          application/json:
            schema:
              type: object
              properties:
                firstName:
                  type: string
                  description: Nome dell'utente.
                lastName:
                  type: string
                  description: Cognome dell'utente.
                fiscalCode:
                  type: string
                  description: Codice fiscale dell'utente.
                email:
                  type: string
                  description: Email dell'utente.
                otp:
                  type: string
                  description: Codice OTP ricevuto via email.
              required:
                - firstName
                - lastName
                - fiscalCode
                - email
                - otp
        required: true
      responses:
        '201':
          description: Utente registrato con successo.
          content:
            application/json:
              schema:
                $ref: '#/components/schemas/AuthResponse'
          headers: {}
        '400':
          description: Richiesta non valida o OTP errato.
          content:
            application/json:
              schema:
                $ref: '#/components/schemas/GenericError'
          headers: {}
        '409':
          description: Utente già esistente.
          content:
            application/json:
              schema:
                $ref: '#/components/schemas/GenericError'
          headers: {}
        '500':
          description: Errore del server.
          content:
            application/json:
              schema:
                $ref: '#/components/schemas/GenericError'
          headers: {}
      security: []
  /users/me/initiatives:
    get:
      summary: initiativesDashboard
      deprecated: false
      description: >-
        Restituisce i dati per la dashboard personale dell'utente (iniziative
        firmate, create o seguite).
      tags:
        - Users
      parameters:
        - name: relation
          in: query
          description: Tipo di relazione (signed, created, followed).
          required: false
          example: created
          schema:
            type: string
            enum:
              - created
              - signed
              - followed
        - name: currentPage
          in: query
          description: Numero della pagina per la paginazione.
          required: false
          schema:
            type: integer
        - name: objectsPerPage
          in: query
          description: Numero di elementi per pagina.
          required: false
          schema:
            type: integer
        - name: X-Mock-User-Id
          in: header
          description: ''
          required: true
          example: 0
          schema:
            type: integer
      responses:
        '200':
          description: ''
          content:
            application/json:
              schema:
                type: object
                properties:
                  data:
                    type: array
                    items:
                      $ref: '#/components/schemas/DetailedInitiative'
                  meta:
                    $ref: '#/components/schemas/MetaDataList'
                required:
                  - data
                  - meta
          headers: {}
        '400':
          description: ''
          content:
            application/json:
              schema:
                $ref: '#/components/schemas/GenericError'
          headers: {}
        '401':
          description: ''
          content:
            application/json:
              schema:
                $ref: '#/components/schemas/GenericError'
          headers: {}
      security: []
  /users/me/notifications:
    get:
      summary: notificationsList
      deprecated: false
      description: >-
        Restituisce la lista delle notifiche dell'utente loggato, con
        possibilità di filtrare per lette/non lette.
      tags:
        - Users
      parameters:
        - name: read
          in: query
          description: Se false, restituisce solo le notifiche non lette.
          required: false
          example: 'false'
          schema:
            type: boolean
        - name: currentPage
          in: query
          description: Numero della pagina per la paginazione.
          required: false
          schema:
            type: integer
        - name: objectsPerPage
          in: query
          description: Numero di elementi per pagina.
          required: false
          schema:
            type: integer
        - name: X-Mock-User-Id
          in: header
          description: ''
          required: true
          example: 0
          schema:
            type: integer
        - name: ETag
          in: header
          description: >-
            Specifica la versione corrispondente della modifica, serve per
            controllare la concorrenza delle risorse alla modifica dello stato. 
          required: false
          example: '"v1"'
          schema:
            type: string
      responses:
        '200':
          description: ''
          content:
            application/json:
              schema:
                type: object
                properties:
                  data:
                    type: array
                    items:
                      $ref: '#/components/schemas/Notification'
                  meta:
                    $ref: '#/components/schemas/MetaDataList'
                required:
                  - data
                  - meta
          headers: {}
        '400':
          description: ''
          content:
            application/json:
              schema:
                $ref: '#/components/schemas/GenericError'
          headers: {}
        '401':
          description: ''
          content:
            application/json:
              schema:
                $ref: '#/components/schemas/GenericError'
          headers: {}
      security: []
  /users/me/notifications/{id}:
    patch:
      summary: readNotification
      deprecated: false
      description: Segna una specifica notifica come letta.
      tags:
        - Users
      parameters:
        - name: id
          in: path
          description: ID univoco della notifica.
          required: true
          example: 0
          schema:
            type: integer
        - name: X-Mock-User-Id
          in: header
          description: ''
          required: true
          example: 0
          schema:
            type: integer
      requestBody:
        content:
          application/json:
            schema:
              type: object
              properties:
                isRead:
                  type: boolean
                  default: true
                  description: Impostare a true per segnare come letta.
              required:
                - isRead
        required: true
      responses:
        '200':
          description: ''
          content:
            application/json:
              schema:
                $ref: '#/components/schemas/Notification'
          headers: {}
        '400':
          description: ''
          content:
            application/json:
              schema:
                $ref: '#/components/schemas/GenericError'
          headers: {}
        '401':
          description: ''
          content:
            application/json:
              schema:
                $ref: '#/components/schemas/GenericError'
          headers: {}
        '403':
          description: ''
          content:
            application/json:
              schema:
                $ref: '#/components/schemas/GenericError'
          headers: {}
        '404':
          description: ''
          content:
            application/json:
              schema:
                $ref: '#/components/schemas/GenericError'
          headers: {}
      security: []
  /users/me:
    get:
      summary: getUser
      deprecated: false
      description: Restituisce i dettagli del profilo dell'utente loggato.
      tags:
        - Users
      parameters:
        - name: X-Mock-User-Id
          in: header
          description: ''
          required: false
          example: ''
          schema:
            type: string
      responses:
        '200':
          description: ''
          content:
            application/json:
              schema:
                $ref: '#/components/schemas/User'
          headers: {}
        '400':
          description: ''
          content:
            application/json:
              schema:
                $ref: '#/components/schemas/GenericError'
          headers: {}
        '401':
          description: ''
          content:
            application/json:
              schema:
                $ref: '#/components/schemas/GenericError'
          headers: {}
        '403':
          description: ''
          content:
            application/json:
              schema:
                $ref: '#/components/schemas/GenericError'
          headers: {}
        '404':
          description: ''
          content:
            application/json:
              schema:
                $ref: '#/components/schemas/GenericError'
          headers: {}
      security: []
  /users:
    post:
      summary: userRegistration
      deprecated: false
      description: >-
        Registrazione di un nuovo utente (Cittadino o Amministratore) al primo
        accesso.
      tags:
        - Users
      parameters: []
      requestBody:
        content:
          application/json:
            schema:
              type: object
              properties:
                email:
                  type: string
                  description: Email inserita dall'utente.
                googleToken:
                  type: string
                otp:
                  type: string
              required:
                - email
                - googleToken
                - otp
        required: true
      responses:
        '201':
          description: ''
          content:
            application/json:
              schema:
                $ref: '#/components/schemas/User'
          headers: {}
        '400':
          description: ''
          content:
            application/json:
              schema:
                $ref: '#/components/schemas/GenericError'
          headers: {}
        '401':
          description: ''
          content:
            application/json:
              schema:
                $ref: '#/components/schemas/GenericError'
          headers: {}
        '403':
          description: ''
          content:
            application/json:
              schema:
                $ref: '#/components/schemas/GenericError'
          headers: {}
        '409':
          description: ''
          content:
            application/json:
              schema:
                $ref: '#/components/schemas/GenericError'
          headers: {}
      security: []
    get:
      summary: showAdminUsers
      deprecated: false
      description: >-
        Permette agli amministratori di visualizzare e cercare nella lista degli
        utenti registrati.
      tags:
        - Users
      parameters:
        - name: isAdmin
          in: query
          description: Filtra per mostrare solo gli admin.
          required: false
          schema:
            type: boolean
        - name: X-Mock-User-Id
          in: header
          description: ''
          required: true
          example: 0
          schema:
            type: integer
      responses:
        '200':
          description: ''
          content:
            application/json:
              schema:
                type: object
                properties:
                  data:
                    type: array
                    items:
                      type: object
                      properties:
                        firstName:
                          type: string
                        lastName:
                          type: string
                        fiscalCode:
                          type: string
                      required:
                        - firstName
                        - lastName
                        - fiscalCode
                required:
                  - data
          headers: {}
        '401':
          description: ''
          content:
            application/json:
              schema:
                $ref: '#/components/schemas/GenericError'
          headers: {}
        '403':
          description: ''
          content:
            application/json:
              schema:
                $ref: '#/components/schemas/GenericError'
          headers: {}
      security: []
  /users/{id}:
    patch:
      summary: changePrivileges
      deprecated: false
      description: Permette di modificare i privilegi di un utente.
      tags:
        - Users
      parameters:
        - name: id
          in: path
          description: ID univoco dell'utente.
          required: true
          example: 0
          schema:
            type: integer
        - name: X-Mock-User-Id
          in: header
          description: ''
          required: true
          example: ''
          schema:
            type: string
      requestBody:
        content:
          application/json:
            schema:
              type: object
              properties:
                isAdmin:
                  type: boolean
                  description: Di default false.
              required:
                - isAdmin
        required: true
      responses:
        '200':
          description: ''
          content:
            application/json:
              schema:
                type: object
                properties: {}
          headers: {}
      security: []
  /users/me/notifications/mark-all-as-read:
    patch:
      summary: markAllNotificationsAsRead
      deprecated: false
      description: Segna tutte le notifiche dell'utente corrente come lette.
      tags:
        - Users
      parameters:
        - name: X-Mock-User-Id
          in: header
          description: Simula l'utente loggato
          required: true
          example: 1
          schema:
            type: integer
      requestBody:
        content:
          application/json:
            schema:
              type: object
              properties: {}
      responses:
        '200':
          description: Tutte le notifiche sono state segnate come lette.
          content:
            application/json:
              schema:
                type: object
                properties:
                  message:
                    type: string
                  updatedCount:
                    type: integer
                    description: Numero di notifiche aggiornate.
                required:
                  - message
                  - updatedCount
          headers: {}
        '401':
          description: Utente non autenticato.
          content:
            application/json:
              schema:
                $ref: '#/components/schemas/GenericError'
          headers: {}
        '500':
          description: Errore del server.
          content:
            application/json:
              schema:
                $ref: '#/components/schemas/GenericError'
          headers: {}
      security: []
  /users/admin/pre-authorize:
    post:
      summary: preAuthorizeAdmin
      deprecated: false
      description: >-
        Pre-autorizza un codice fiscale per la registrazione come
        amministratore. Solo gli amministratori possono utilizzare questa API.
      tags:
        - Users
      parameters:
        - name: X-Mock-User-Id
          in: header
          description: Simula l'utente loggato (deve essere admin)
          required: true
          example: 1
          schema:
            type: integer
      requestBody:
        content:
          application/json:
            schema:
              type: object
              properties:
                fiscalCode:
                  type: string
                  description: Codice fiscale dell'amministratore da pre-autorizzare.
              required:
                - fiscalCode
        required: true
      responses:
        '201':
          description: Amministratore pre-autorizzato con successo.
          content:
            application/json:
              schema:
                type: object
                properties:
                  message:
                    type: string
                  fiscalCode:
                    type: string
                required:
                  - message
                  - fiscalCode
          headers: {}
        '400':
          description: Richiesta non valida.
          content:
            application/json:
              schema:
                $ref: '#/components/schemas/GenericError'
          headers: {}
        '401':
          description: Utente non autenticato.
          content:
            application/json:
              schema:
                $ref: '#/components/schemas/GenericError'
          headers: {}
        '403':
          description: L'utente non ha privilegi di amministratore.
          content:
            application/json:
              schema:
                $ref: '#/components/schemas/GenericError'
          headers: {}
        '409':
          description: Codice fiscale già pre-autorizzato.
          content:
            application/json:
              schema:
                $ref: '#/components/schemas/GenericError'
          headers: {}
        '500':
          description: Errore del server.
          content:
            application/json:
              schema:
                $ref: '#/components/schemas/GenericError'
          headers: {}
      security: []
  /categories:
    get:
      summary: categoriesList
      deprecated: false
      description: Restituisce la lista delle categorie.
      tags:
        - Filters
      parameters: []
      responses:
        '200':
          description: ''
          content:
            application/json:
              schema:
                type: object
                properties:
                  data:
                    type: array
                    items:
                      $ref: '#/components/schemas/Category'
                required:
                  - data
          headers: {}
        '500':
          description: ''
          content:
            application/json:
              schema:
                $ref: '#/components/schemas/GenericError'
          headers: {}
      security: []
  /platforms:
    get:
      summary: platformsList
      deprecated: false
      description: Restituisce la lista delle piattaforme.
      tags:
        - Filters
      parameters: []
      responses:
        '200':
          description: ''
          content:
            application/json:
              schema:
                type: object
                properties:
                  data:
                    type: array
                    items:
                      $ref: '#/components/schemas/Platform'
                required:
                  - data
          headers: {}
        '500':
          description: ''
          content:
            application/json:
              schema:
                $ref: '#/components/schemas/GenericError'
          headers: {}
      security: []
components:
  schemas:
    DetailedInitiative:
      properties:
        id:
          description: ID univoco dell'iniziativa.
          type: integer
          minimum: 1
          maximum: 2147483647
        title:
          description: Titolo dell'iniziativa.
          type: string
          maxLength: 255
        description:
          description: Descrizione dettagliata della proposta.
          type: string
          maxLength: 65535
        place:
          description: Luogo o area di interesse.
          default: 'NULL'
          type: string
          maxLength: 64
          nullable: true
        status:
          type: string
          description: Stato attuale dell'iniziativa.
          default: '''In corso'''
          enum:
            - In corso
            - Approvata
            - Respinta
            - Scaduta
          nullable: true
        signatures:
          description: Numero di firme raccolte.
          default: '0'
          type: integer
          minimum: -2147483648
          maximum: 2147483647
          nullable: true
        creationDate:
          type: string
          description: Data di creazione.
          format: date
        expirationDate:
          description: Data di scadenza.
          default: 'NULL'
          type: string
          format: date
          nullable: true
        authorId:
          description: 'ID univoco dell''autore. '
          default: 'NULL'
          type: integer
          minimum: -2147483648
          maximum: 2147483647
          nullable: true
        categoryId:
          description: ID univoco della categoria.
          type: integer
          minimum: -2147483648
          maximum: 2147483647
        platformId:
          description: ID univoco della piattaforma di provenienza.
          default: 'NULL'
          type: integer
          minimum: -2147483648
          maximum: 2147483647
          nullable: true
        externalURL:
          description: URL di reindirizzamento alla piattaforma esterna.
          default: 'NULL'
          type: string
          maxLength: 500
          nullable: true
        attachments:
          type: array
          items:
            $ref: '#/components/schemas/Attachment'
          description: Array contenente gli eventuali allegati all'iniziativa.
          nullable: true
        reply:
          $ref: '#/components/schemas/Reply'
          description: Eventuale risposta del Comune.
          nullable: true
      type: object
      required:
        - id
        - title
        - description
        - creationDate
        - categoryId
    ParticipatoryBudget:
      type: object
      properties:
        options:
          type: array
          items:
            $ref: '#/components/schemas/PBOption'
          minItems: 3
          maxItems: 3
          description: Array di opzioni.
        id:
          type: integer
          description: ID univoco del bilancio partecipativo.
        creatorId:
          type: integer
          description: ID univoco del creatore del bilancio partecipativo.
        title:
          type: string
          description: Titolo del bilancio partecipativo.
        createdAt:
          type: string
          format: date-time
          description: Data di creazione.
        expirationDate:
          type: string
          format: date
          description: Data di scadenza.
      required:
        - options
        - id
        - creatorId
        - title
        - createdAt
        - expirationDate
    AuthResponse:
      type: object
      properties:
        accessToken:
          type: string
        user:
          $ref: '#/components/schemas/User'
        newUserId:
          type: boolean
      required:
        - accessToken
        - user
        - newUserId
    GenericError:
      type: object
      properties:
        timeStamp:
          type: string
          format: date-time
          description: Data e ora dell'errore.
        message:
          type: string
          description: Messaggio dell'errore.
        details:
          type: array
          items:
            type: object
            properties:
              field:
                type: string
              issue:
                type: string
            required:
              - field
              - issue
          nullable: true
      required:
        - timeStamp
        - message
    MetaDataList:
      type: object
      properties:
        currentPage:
          type: integer
          description: Numero della pagina per la paginazione.
        objectsPerPage:
          type: integer
          description: Numero di elementi per pagina.
        totalObjects:
          type: integer
          description: Numero totale di oggetti.
        totalPages:
          type: integer
          description: Numero totale di pagine.
      required:
        - currentPage
        - objectsPerPage
        - totalObjects
        - totalPages
    AttachmentNoID:
      properties:
        fileName:
          description: ''
          type: string
          maxLength: 255
        filePath:
          description: ''
          type: string
          maxLength: 500
        fileType:
          description: ''
          default: 'NULL'
          type: string
          maxLength: 50
          nullable: true
        uploadedAt:
          description: ''
          default: current_timestamp()
          type: string
      type: object
      required:
        - fileName
        - filePath
        - uploadedAt
    Attachment:
      properties:
        id:
          description: ID univoco dell'allegato.
          type: integer
          minimum: 1
          maximum: 2147483647
        fileName:
          description: Nome del file.
          type: string
          maxLength: 255
        filePath:
          description: Path del file.
          type: string
          maxLength: 500
        fileType:
          description: Tipo di file.
          default: 'NULL'
          type: string
          maxLength: 50
          nullable: true
        uploadedAt:
          description: Data di caricamento.
          default: current_timestamp()
          type: string
      type: object
      required:
        - id
        - fileName
        - filePath
        - uploadedAt
    Category:
      properties:
        id:
          description: ID univoco della categoria.
          type: integer
          minimum: 1
          maximum: 2147483647
        name:
          description: Nome della categoria.
          type: string
          maxLength: 100
      type: object
      required:
        - id
        - name
    InitiativeSignature:
      properties:
        userId:
          description: ID dell'utente firmatario.
          type: integer
          minimum: -2147483648
          maximum: 2147483647
        initiativeId:
          description: ID dell'iniziativa firmata.
          type: integer
          minimum: -2147483648
          maximum: 2147483647
        signatureDate:
          description: Data della firma.
          default: current_timestamp()
          type: string
      type: object
      required:
        - userId
        - initiativeId
        - signatureDate
    Initiative:
      properties:
        id:
          description: ID univoco dell'iniziativa.
          type: integer
          minimum: 1
          maximum: 2147483647
        title:
          description: Titolo dell'iniziativa.
          type: string
          maxLength: 255
        place:
          description: Luogo o area di interesse.
          default: 'NULL'
          type: string
          maxLength: 64
          nullable: true
        status:
          type: string
          description: Stato attuale dell'iniziativa.
          default: '''In corso'''
          enum:
            - In corso
            - Approvata
            - Respinta
            - Scaduta
          nullable: true
        signatures:
          description: Numero di firme raccolte.
          default: '0'
          type: integer
          minimum: -2147483648
          maximum: 2147483647
          nullable: true
        creationDate:
          type: string
          description: Data di creazione.
          format: date
        expirationDate:
          description: Data di scadenza.
          default: 'NULL'
          type: string
          format: date
          nullable: true
        authorId:
          description: 'ID univoco dell''autore. '
          default: 'NULL'
          type: integer
          minimum: -2147483648
          maximum: 2147483647
          nullable: true
        categoryId:
          description: ID univoco della categoria.
          type: integer
          minimum: -2147483648
          maximum: 2147483647
        platformId:
          description: ID univoco della piattaforma.
          default: 'NULL'
          type: integer
          minimum: -2147483648
          maximum: 2147483647
          nullable: true
        externalURL:
          description: URL di reindirizzamento alla piattaforma esterna.
          default: 'NULL'
          type: string
          maxLength: 500
          nullable: true
        attachment:
          $ref: '#/components/schemas/Attachment'
          description: Immagine
          nullable: true
      type: object
      required:
        - id
        - title
        - creationDate
        - categoryId
    SavedInitiative:
      properties:
        userId:
          description: ID dell'utente che ha salvato l'iniziativa.
          type: integer
          minimum: -2147483648
          maximum: 2147483647
        initiativeId:
          description: ID dell'iniziativa salvata.
          type: integer
          minimum: -2147483648
          maximum: 2147483647
        savedAt:
          description: Data di salvataggio.
          default: current_timestamp()
          type: string
      type: object
      required:
        - userId
        - initiativeId
        - savedAt
    Notification:
      properties:
        id:
          description: ID univoco della notifica.
          type: integer
          minimum: 1
          maximum: 2147483647
        text:
          description: Contenuto della notifica.
          type: string
          maxLength: 65535
        isRead:
          type: boolean
          description: Flag che indica se è stata letta o no.
          nullable: true
        creationDate:
          description: Data di creazione.
          default: current_timestamp()
          type: string
        linkRef:
          description: ''
          default: 'NULL'
          type: string
          maxLength: 255
          nullable: true
      type: object
      required:
        - id
        - text
        - creationDate
    PBOption:
      properties:
        id:
          description: ID univoco dell'opzione.
          type: integer
          minimum: 1
          maximum: 2147483647
        text:
          description: Testo di descrizione dell'opzione.
          type: string
          maxLength: 250
        position:
          description: >-
            Posizione relativa alla numerazione nell'elenco di opzioni del
            bilancio.
          type: integer
          minimum: -2147483648
          maximum: 2147483647
      type: object
      required:
        - id
        - text
        - position
    Platform:
      properties:
        id:
          description: ID univoco della piattaforma.
          type: integer
          minimum: 1
          maximum: 2147483647
        platformName:
          description: Nome della piattaforma.
          type: string
          maxLength: 100
        iconPath:
          description: Path dell'icona.
          default: 'NULL'
          type: string
          maxLength: 255
          nullable: true
        platformLink:
          description: Link di reindirizzamento alla piattaforma.
          default: 'NULL'
          type: string
          maxLength: 255
          nullable: true
      type: object
      required:
        - id
        - platformName
    Reply:
      properties:
        id:
          description: ID univoco della risposta.
          type: integer
          minimum: 1
          maximum: 2147483647
        initiativeId:
          description: ID univoco dell'iniziativa a cui è rivolta la risposta.
          type: integer
          minimum: -2147483648
          maximum: 2147483647
        adminId:
          description: ID univoco dell'amministratore che ha fornito la risposta.
          default: 'NULL'
          type: integer
          minimum: -2147483648
          maximum: 2147483647
          nullable: true
        replyText:
          description: Testo di esposizione delle motivazioni della risposta.
          type: string
          maxLength: 65535
        creationDate:
          description: Data di creazione.
          default: current_timestamp()
          type: string
        attachments:
          type: array
          items:
            $ref: '#/components/schemas/AttachmentNoID'
          description: Eventuali allegati (immagini o PDF)
          nullable: true
      type: object
      required:
        - id
        - initiativeId
        - replyText
        - creationDate
    User:
      properties:
        id:
          description: ID univoco dell'utente.
          type: integer
          minimum: 1
          maximum: 2147483647
        firstName:
          description: Nome dell'utente.
          type: string
          maxLength: 100
        lastName:
          description: Cognome dell'utente.
          type: string
          maxLength: 100
        fiscalCode:
          description: Codice fiscale dell'utente.
          type: string
          maxLength: 16
        email:
          description: Email dell'utente.
          type: string
          maxLength: 150
        isAdmin:
          type: boolean
          description: >-
            Flag che determina se l'utente può svolgere il ruolo di
            amministratore.
        isCitizen:
          type: boolean
          description: Flag che determina se l'utente può svolgere il ruolo di cittadino.
        createdAt:
          description: Data di registrazione.
          default: current_timestamp()
          type: string
      type: object
      required:
        - id
        - firstName
        - lastName
        - fiscalCode
        - email
        - createdAt
        - isAdmin
        - isCitizen
  responses: {}
  securitySchemes: {}
servers: []
security: []
\end{minted}
\section{Implementazione}
L’applicazione è stata sviluppata adottando un’architettura client-server disaccoppiata, basata interamente sull'ecosistema JavaScript.


Per la parte di Backend, è stato utilizzato il runtime environment Node.js con il framework Express.js per la creazione delle API RESTful. Questa scelta tecnologica è stata dettata dalla necessità di gestire un elevato numero di operazioni di I/O asincrone (come upload di file e invio email) in modo non bloccante.


Per la persistenza dei dati è stato adottato MySQL. La scelta di un RDBMS risponde alla natura intrinsecamente strutturata e stabile del dominio applicativo (gestione istituzionale), rendendo non necessaria la flessibilità schema-less delle soluzioni NoSQL. Questa architettura garantisce la rigorosa integrità referenziale e la consistenza transazionale (ACID) indispensabili per la validità di firme e votazioni. L'interazione con il database è gestita dalla libreria mysql2, che assicura efficienza e protezione contro SQL Injection tramite l'uso di prepared statements.


Per la parte di Frontend, lo sviluppo è stato realizzato con il framework Vue.js 3 (Composition API). È stato scelto Vite come strumento di build per la sua velocità di compilazione e l'Hot Module Replacement (HMR) istantaneo. La gestione dello stato applicativo è centralizzata tramite Pinia, mentre l'interfaccia utente è stata costruita seguendo un approccio a componenti modulari.

%   ----    repo organizations --- 
\subsection{Organizzazione delle repositories}
\dirtree{%
.1 / .
.2 .github/ \DTcomment{Configurazioni specifiche di GitHub}.
.3 workflows/ \DTcomment{Automazioni e pipeline (GitHub Actions)}.
.4 cicd.yml \DTcomment{Pipeline CI/CD per test e deploy automatico su Render}.
.2 client/ \DTcomment{Frontend Application (Vue.js + Vite)}.
.3 src/.
.4 assets/ \DTcomment{Immagini e stili compilati}.
.4 components/ \DTcomment{Componenti UI riutilizzabili (Cards, Toasts)}.
.4 composables/ \DTcomment{Logica riutilizzabile (Hooks custom)}.
.4 router/ \DTcomment{Configurazione delle rotte Vue Router}.
.4 stores/ \DTcomment{State Management (Pinia stores)}.
.4 utils/ \DTcomment{Funzioni di utilità (es. dateUtils)}.
.4 views/ \DTcomment{Pagine principali (Admin e User views)}.
.4 App.vue \DTcomment{Componente Root}.
.4 main.js \DTcomment{Entry point frontend}.
.3 vite.config.js \DTcomment{Configurazione bundler Vite}.
.3 package.json \DTcomment{Dipendenze Frontend}.
.2 deliverable/ \DTcomment{Documentazione}.
.3 D*/ \DTcomment{Deliverable numero * (main .tex, pdf, sections e immagini)}.
.2 server/ \DTcomment{Backend Application}.
.3 APIsDocumentazione/ \DTcomment{Specifiche API (OpenAPI/Swagger)}.
.3 DatabaseRelazionale/ \DTcomment{Contiene lo script SQL per la creazion del DB}.
.3 backend/ \DTcomment{Server Node.js}.
.4 \_\_tests\_\_/ \DTcomment{Test di integrazione e unità}.
.4 data/ \DTcomment{Dati statici esterni}.
.4 src/.
.5 config/ \DTcomment{Configurazione DB e costanti}.
.5 controllers/ \DTcomment{Logica di business e gestione richieste}.
.5 middleware/ \DTcomment{Auth, upload e gestione errori}.
.5 routes/ \DTcomment{Definizione endpoint API}.
.5 services/ \DTcomment{Logica complessa (es. importService)}.
.5 utils/ \DTcomment{Utility backend}.
.5 validators/ \DTcomment{Validazione schemi dati}.
.5 app.js \DTcomment{Configurazione applicazione Express}.
.5 server.js \DTcomment{Entry point server}.
.4 uploads/ \DTcomment{File caricati dagli utenti}.
.5 initiatives/ \DTcomment{Allegati iniziative}.
.5 replies/ \DTcomment{Allegati risposte}.
.4 package.json \DTcomment{Dipendenze Backend}.
.2 README.md \DTcomment{Documentazione generale}.
}


%   ----    branching strategy --- 
\subsection{Branching strategy e organizzazione del lavoro}

Per la gestione del versionamento del codice è stata utilizzata la piattaforma GitHub. Data la dimensione del team e l'interdipendenza tra i moduli backend e frontend, abbiamo optato per una strategia di Trunk-Based Development.

A differenza di strategie complesse come GitFlow (spesso sovrabbondanti per team ristretti), lo sviluppo si è concentrato su un unico ramo principale, il branch main. Questa scelta ha garantito:
\begin{itemize}
    \item \textbf{Integrazione Continua}: Ogni funzionalità completata è stata immediatamente integrata nel codice base, evitando il "merge hell" tipico dei branch a lunga vita.
    \item \textbf{Risoluzione Rapida dei Conflitti}: Lavorando sullo stesso ramo, eventuali conflitti tra il lavoro dei membri (es. modifiche concorrenti allo stesso file) venivano evidenziati e risolti immediatamente.
\end{itemize}

\subsubsection{Suddivisione Logica del Lavoro}
Sebbene il lavoro sia confluito tecnicamente sullo stesso branch, lo sviluppo è stato logicamente parallelizzato assegnando a ciascun membro la responsabilità esclusiva di specifiche aree funzionali, riducendo al minimo le sovrapposizioni:
\begin{description}
    \item[Enea D'Angiò]: Si è occupato principalmente dell'architettura Backend (configurazione Express, gestione e creazione del database MySQL) e della progettazione e integrazione delle API. Ha contribuito parzialmente al Frontend attraverso bugfix, refactoring e implementazioni di piccole feature. Ha inoltre lavorato al deployment.
    \item[Ivan Nedeljkovic]: Ha curato lo sviluppo dei componenti Frontend in Vue.js creando l'intera infrastruttura lato client e gestendo ui e ux. Ha inoltre collaborato al deployment.
    \item[Alessandro Mattarolo]: Ha seguito lo sviluppo con la scrittura della documentazione, l'organizzazione dei file LaTeX e la progettazione del Frontend.
\end{description}
L'analisi dei contributi (visibile nella sezione Insights > Contributors della repository) mostra un totale di 102 commit, distribuiti tra i membri in modo proporzionale ai task assegnati. Lo sbilanciamento fra la quantità di commit eseguiti da Enea D'Angiò e Ivan Nedeljkovic e quelli eseguiti da Alessandro Mattarolo è dovuto al fatto che quest'ultimo ha lavorato principalmente alla documentazione.

%   ---- dependencies ---
\subsection{Dipendenze}

Il progetto si basa su un insieme di librerie gestite tramite il package manager \texttt{npm}. Di seguito sono elencate le dipendenze utilizzate, suddivise per ambito applicativo.

\subsubsection{Backend}
Le seguenti librerie sono state utilizzate per lo sviluppo del server Node.js:

\begin{description}
    \item[cors] Middleware per abilitare il Cross-Origin Resource Sharing, permettendo al frontend di comunicare con il backend.
    \item[dotenv] Modulo per caricare le variabili d'ambiente da un file \texttt{.env} a \texttt{process.env}, garantendo la sicurezza delle credenziali.
    \item[express] Framework web minimalista per Node.js, utilizzato per la gestione del server, del routing e delle API REST.
    \item[google-auth-library] Libreria client di Google per la verifica sicura dei token di autenticazione e l'integrazione del login Google.
    \item[joi] Libreria per la validazione degli schemi di dati, utilizzata per verificare la correttezza degli input nelle richieste API.
    \item[multer] Middleware per la gestione di \texttt{multipart/form-data}, utilizzato per l'upload di file (immagini delle iniziative).
    \item[mysql2] Client MySQL veloce e sicuro per Node.js, supporta Promise e Prepared Statements per prevenire SQL Injection.
    \item[node-cron] Scheduler per l'esecuzione di task periodici sul server (cron jobs).
    \item[nodemailer] Modulo per l'invio di email transazionali (es. notifiche) tramite server SMTP.
    \item[nodemon] Strumento di sviluppo che monitora le modifiche ai file e riavvia automaticamente il server.
\end{description}

\subsubsection{Frontend}
Per lo sviluppo dell'interfaccia utente con Vue.js sono state utilizzate le seguenti dipendenze:

\begin{description}
    \item[vue] Framework progressivo per la creazione di interfacce utente reattive e basate su componenti.
    \item[vue-router] Router ufficiale per Vue.js, gestisce la navigazione SPA (Single Page Application).
    \item[pinia] Store manager per Vue, utilizzato per la gestione centralizzata dello stato dell'applicazione.
    \item[axios] Client HTTP basato su Promise per effettuare richieste asincrone verso il backend.
    \item[vite] Build tool di nuova generazione che offre un ambiente di sviluppo rapido e ottimizzazione per la produzione.
    \item[vue3-google-login] Componente Vue per facilitare l'integrazione del pulsante di accesso Google.
    \item[eslint / eslint-plugin-vue] Strumenti di analisi statica per identificare problemi nel codice JavaScript e Vue.
    \item[prettier] Code formatter per garantire uno stile di codice coerente.
    \item[@vitejs/plugin-vue] Plugin ufficiale per il supporto di Vue.js all'interno di Vite.
    \item[vite-plugin-vue-devtools] Plugin per migliorare l'esperienza di debugging con i devtools di Vue.
\end{description}

%----   Database ---

\subsection{Database}


Il database MySQL è stato strutturato nelle seguenti tabelle principali:

\subsubsection{Categoria}
La tabella \texttt{CATEGORIA} gestisce la classificazione tematica delle iniziative presenti sulla piattaforma.
\begin{itemize}
    \item \textbf{ID\_CATEGORIA} (\texttt{INT}): Chiave primaria autoincrementale che identifica univocamente la categoria.
    \item \textbf{NOME} (\texttt{VARCHAR}): Nome della categoria (es. "Ambiente", "Viabilità"), univoco nel sistema.
\end{itemize}

\subsubsection{Piattaforma}
La tabella \texttt{PIATTAFORMA} memorizza le informazioni relative ai siti esterni (es. Change.org) da cui possono essere importate le iniziative.
\begin{itemize}
    \item \textbf{ID\_PIATTAFORMA} (\texttt{INT}): Chiave primaria autoincrementale per identificare la piattaforma esterna.
    \item \textbf{NOME\_PIATTAFORMA} (\texttt{VARCHAR}): Nome della piattaforma esterna.
    \item \textbf{PATH\_ICONA} (\texttt{VARCHAR}): Percorso del file immagine dell'icona associata alla piattaforma.
    \item \textbf{LINK\_BASE\_PIATTAFORMA} (\texttt{VARCHAR}): URL base della piattaforma esterna.
\end{itemize}

\subsubsection{Utente}
La tabella \texttt{UTENTE} centralizza le informazioni di tutti gli attori che interagiscono con la piattaforma (Cittadini e Amministratori).
\begin{itemize}
    \item \textbf{ID\_UTENTE} (\texttt{INT}): Chiave primaria autoincrementale che identifica univocamente l'utente.
    \item \textbf{NOME} (\texttt{VARCHAR}): Nome anagrafico dell'utente.
    \item \textbf{COGNOME} (\texttt{VARCHAR}): Cognome anagrafico dell'utente.
    \item \textbf{CODICE\_FISCALE} (\texttt{CHAR}): Codice fiscale univoco (formato standard 16 caratteri).
    \item \textbf{EMAIL} (\texttt{VARCHAR}): Indirizzo email univoco, utilizzato per l'accesso e le notifiche.
    \item \textbf{IS\_ADMIN} (\texttt{BOOLEAN}): Definisce se l'utente possiede i privilegi di amministratore.
    \item \textbf{IS\_CITTADINO} (\texttt{BOOLEAN}): Definisce se l'utente è verificato come residente/cittadino.
    \item \textbf{CREATED\_AT} (\texttt{TIMESTAMP}): Data e ora di registrazione dell'utente.
\end{itemize}

\subsubsection{Pre\_Autorizzato}
La tabella \texttt{PRE\_AUTORIZZATO} funge da whitelist contenente i codici fiscali degli amministratori non cittadini e che quindi otterranno un account con privilegi da admin ma non da cittadino. 
\begin{itemize}
    \item \textbf{CODICE\_FISCALE} (\texttt{CHAR}): Chiave primaria, contiene il codice fiscale pre-validato.
    \item \textbf{DATA\_INSERIMENTO} (\texttt{TIMESTAMP}): Data in cui il codice è stato inserito nel sistema.
    \item \textbf{INSERITO\_DA} (\texttt{INT}): Chiave esterna verso \texttt{UTENTE}, indica l'amministratore che ha inserito il dato.
\end{itemize}

\subsubsection{Notifica}
La tabella \texttt{NOTIFICA} gestisce i messaggi e gli avvisi inviati dal sistema agli utenti.
\begin{itemize}
    \item \textbf{ID\_NOTIFICA} (\texttt{INT}): Chiave primaria autoincrementale della notifica.
    \item \textbf{ID\_UTENTE} (\texttt{INT}): Chiave esterna che associa la notifica al destinatario.
    \item \textbf{TESTO} (\texttt{TEXT}): Contenuto testuale del messaggio.
    \item \textbf{LETTA} (\texttt{BOOLEAN}): Flag che indica se la notifica è stata visualizzata dall'utente.
    \item \textbf{DATA\_CREAZIONE} (\texttt{TIMESTAMP}): Data e ora di generazione della notifica.
    \item \textbf{LINK\_RIF} (\texttt{VARCHAR}): Collegamento opzionale per reindirizzare l'utente all'evento specifico.
\end{itemize}

\subsubsection{Bilancio Partecipativo}
La tabella \texttt{BILANCIO\_PARTECIPATIVO} definisce le consultazioni pubbliche create dall'amministrazione per l'allocazione di risorse.
\begin{itemize}
    \item \textbf{ID\_BIL} (\texttt{INT}): Chiave primaria autoincrementale dell'evento di bilancio.
    \item \textbf{ID\_CREATOR} (\texttt{INT}): Chiave esterna verso l'amministratore che ha creato l'evento.
    \item \textbf{TITOLO} (\texttt{VARCHAR}): Titolo descrittivo del bilancio partecipativo.
    \item \textbf{CREATED\_AT} (\texttt{TIMESTAMP}): Data di creazione dell'evento.
    \item \textbf{DATA\_SCADENZA} (\texttt{DATE}): Data termine oltre la quale non è più possibile votare.
\end{itemize}

\subsubsection{Iniziativa}
La tabella \texttt{INIZIATIVA} costituisce il nucleo centrale dell'applicativo web ed include iniziative sia interne che esterne.
\begin{itemize}
    \item \textbf{ID\_INIZIATIVA} (\texttt{INT}): Chiave primaria autoincrementale dell'iniziativa.
    \item \textbf{TITOLO} (\texttt{VARCHAR}): Titolo dell'iniziativa.
    \item \textbf{DESCRIZIONE} (\texttt{TEXT}): Descrizione dettagliata della proposta.
    \item \textbf{LUOGO} (\texttt{VARCHAR}): Indicazione geografica o zona di interesse.
    \item \textbf{STATO} (\texttt{ENUM}): Stato corrente ('In corso', 'Approvata', 'Respinta', 'Scaduta').
    \item \textbf{NUM\_FIRME} (\texttt{INT}): Contatore delle firme raccolte (aggiornato per iniziative interne o importato per esterne).
    \item \textbf{DATA\_CREAZIONE} (\texttt{TIMESTAMP}): Data di pubblicazione dell'iniziativa.
    \item \textbf{DATA\_SCADENZA} (\texttt{DATE}): Termine ultimo per la raccolta firme.
    \item \textbf{ID\_AUTORE} (\texttt{INT}): Chiave esterna verso l'utente proponente (NULL se di sistema o importata anonimamente).
    \item \textbf{ID\_CATEGORIA} (\texttt{INT}): Chiave esterna per la categorizzazione tematica.
    \item \textbf{ID\_PIATTAFORMA} (\texttt{INT}): Chiave esterna valorizzata solo se l'iniziativa proviene da una piattaforma terza.
    \item \textbf{URL\_ESTERNO} (\texttt{VARCHAR}): Link originale dell'iniziativa se esterna, NULL altrimenti.
\end{itemize}

\subsubsection{Opzioni Bilancio}
La tabella \texttt{OPZIONI\_BILANCIO} contiene le singole voci o progetti votabili all'interno di un evento di bilancio partecipativo.
\begin{itemize}
    \item \textbf{ID\_OB} (\texttt{INT}): Chiave primaria autoincrementale dell'opzione.
    \item \textbf{ID\_BIL} (\texttt{INT}): Chiave esterna che collega l'opzione al relativo bilancio partecipativo.
    \item \textbf{TEXT} (\texttt{VARCHAR}): Descrizione testuale dell'opzione.
    \item \textbf{POSITION} (\texttt{TINYINT}): Ordine numerico di visualizzazione nella lista.
\end{itemize}

\subsubsection{Voti Bilancio}
La tabella \texttt{VOTI\_BILANCIO} registra le preferenze espresse dagli utenti per le opzioni di bilancio.
\begin{itemize}
    \item \textbf{ID\_UTENTE} (\texttt{INT}): Chiave esterna dell'utente votante (parte della chiave primaria composta).
    \item \textbf{ID\_BIL} (\texttt{INT}): Chiave esterna del bilancio (parte della chiave primaria composta per garantire un solo voto per bilancio).
    \item \textbf{OPTION\_ID} (\texttt{INT}): Chiave esterna dell'opzione scelta.
    \item \textbf{VOTED\_AT} (\texttt{TIMESTAMP}): Data e ora dell'espressione del voto.
\end{itemize}

\subsubsection{Firma Iniziativa}
La tabella \texttt{FIRMA\_INIZIATIVA} memorizza le adesioni degli utenti alle iniziative proposte.
\begin{itemize}
    \item \textbf{ID\_UTENTE} (\texttt{INT}): Chiave esterna dell'utente firmatario.
    \item \textbf{ID\_INIZIATIVA} (\texttt{INT}): Chiave esterna dell'iniziativa firmata.
    \item \textbf{DATA\_FIRMA} (\texttt{TIMESTAMP}): Data e ora della firma.
\end{itemize}

\subsubsection{Iniziativa Salvata}
La tabella \texttt{INIZIATIVA\_SALVATA} permette agli utenti di salvare iniziative nei propri preferiti per consultazione futura.
\begin{itemize}
    \item \textbf{ID\_UTENTE} (\texttt{INT}): Chiave esterna dell'utente che salva l'iniziativa.
    \item \textbf{ID\_INIZIATIVA} (\texttt{INT}): Chiave esterna dell'iniziativa salvata.
    \item \textbf{SAVED\_AT} (\texttt{TIMESTAMP}): Data e ora del salvataggio.
\end{itemize}

\subsubsection{Risposta}
La tabella \texttt{RISPOSTA} contiene le comunicazioni ufficiali dell'amministrazione in merito a una specifica iniziativa.
\begin{itemize}
    \item \textbf{ID\_RISPOSTA} (\texttt{INT}): Chiave primaria autoincrementale della risposta.
    \item \textbf{ID\_INIZIATIVA} (\texttt{INT}): Chiave esterna dell'iniziativa a cui la risposta si riferisce.
    \item \textbf{ID\_ADMIN} (\texttt{INT}): Chiave esterna dell'amministratore che ha redatto la risposta.
    \item \textbf{TEXT\_RISP} (\texttt{TEXT}): Contenuto della risposta ufficiale.
    \item \textbf{DATA\_CREAZIONE} (\texttt{TIMESTAMP}): Data di pubblicazione della risposta.
\end{itemize}

\subsubsection{Allegato}
La tabella \texttt{ALLEGATO} gestisce i file multimediali associati alle entità del sistema, implementando un vincolo di esclusività (XOR) tra Iniziative e Risposte.
\begin{itemize}
    \item \textbf{ID\_ALLEGATO} (\texttt{INT}): Chiave primaria autoincrementale del file.
    \item \textbf{FILE\_NAME} (\texttt{VARCHAR}): Nome originale del file.
    \item \textbf{FILE\_PATH} (\texttt{VARCHAR}): Percorso di archiviazione del file nel server.
    \item \textbf{FILE\_TYPE} (\texttt{VARCHAR}): Tipologia del file (estensione).
    \item \textbf{UPLOADED\_AT} (\texttt{TIMESTAMP}): Data di caricamento.
    \item \textbf{ID\_INIZIATIVA} (\texttt{INT}): Chiave esterna (opzionale) se l'allegato appartiene a un'iniziativa.
    \item \textbf{ID\_RISPOSTA} (\texttt{INT}): Chiave esterna (opzionale) se l'allegato appartiene a una risposta.
\end{itemize}

% ----  testing

\subsection{Testing}
Le api e il relativo codice presentano una test-suite che consente di verificarne il corretto funzionamento. I test sono utilizzati all’interno della configurazione di CI/CD.

L’implementazione dei test è organizzata in file .test.js implementati nella cartella \_\_test\_\_. Sono divisi in 4 file, organizzati per argomento.  

Abbiamo implementato tutti i test definiti, per un totale di 94 test. 


% Font piccolo per far stare tutto
\footnotesize 

\begin{xltabular}{\textwidth}{|l|M|S|L|c|S|S|c|}

    % --- INTESTAZIONE PRIMA PAGINA ---
    \hline 
    \textbf{N.} & \textbf{Desc.} & \textbf{\textit{Test Data}} & \textbf{Precondizioni} & \textbf{Dip.} & \textbf{R. Atteso} & \textbf{R. Risc.} & \textbf{Note}\\ 
    \hline 
    \endfirsthead
    
    % --- INTESTAZIONE PAGINE SUCCESSIVE ---
    \multicolumn{8}{l}{\textit{...continua dalla pagina precedente}} \\
    \hline 
    \textbf{N.} & \textbf{Desc.} & \textbf{\textit{Test Data}} & \textbf{Precondizioni} & \textbf{Dip.} & \textbf{R. Atteso} & \textbf{R. Risc.} & \textbf{Note}\\ 
    \hline
    \endhead
    
    % --- PIÈ DI PAGINA ---
    \hline 
    \multicolumn{8}{r}{\textit{continua nella pagina successiva...}}\\ 
    \hline
    \endfoot
    
    % --- FINE TABELLA ---
    \hline
    \endlastfoot
    
    % =========================================================================
    % FILE: auth.test.js (Autenticazione)
    % =========================================================================
    
    1.1 & Login con token Google non valido & 
    \path{token="INVALID"} & 
    Il client invia un token che Google non riesce a validare o che risulta malformato. & - & 
    Err \path{401} "Token non valido" & 
    Err \path{401} "Token non valido" & RF1.1 \\
    \hline
    
    1.2 & Login utente esistente (match Codice Fiscale) & 
    \path{token="VALID_EXISTING"} & 
    L'utente deve essere già registrato nel database e il Codice Fiscale calcolato dal token Google deve corrispondere esattamente a quello salvato. & DB & 
    Status \path{200} \newline JSON Utente & 
    Status \path{200} \newline JSON Utente & RF1.2 \\
    \hline
    
    1.3 & Login utente esistente (fallback Email) & 
    \path{token="VALID_EMAIL"} & 
    L'utente non viene trovato per Codice Fiscale, ma esiste un utente con la stessa email. Il CF nel DB è diverso da quello calcolato. & DB & 
    Status \path{200} \newline CF Aggiornato & 
    Status \path{200} \newline CF Aggiornato & RF1.3 \\
    \hline

    1.4 & Login nuovo utente (non registrato) & 
    \path{token="VALID_NEW"} & 
    L'utente non deve essere presente nel database né per CF né per email, e non deve essere nella lista dei pre-autorizzati. & DB & 
    Err \path{404} \path{NEED_REGISTRATION} & 
    Err \path{404} \path{NEED_REGISTRATION} & RF1.4 \\
    \hline

    1.5 & Login Admin pre-autorizzato & 
    \path{token="PREAUTH_ADMIN"} & 
    L'utente non esiste in anagrafica, ma il suo CF è presente nella tabella \path{PRE_AUTORIZZATO} inserita da un altro admin. & DB & 
    Status \path{200} \newline Profilo Admin creato & 
    Status \path{200} \newline Profilo Admin creato & RF2.2 \\
    \hline

    2.1 & Richiesta OTP: Email non valida & 
    \path{email="not-an-email"} & 
    Il formato dell'email inviata non rispetta la regex standard per gli indirizzi email. & - & 
    Err \path{400} "Email non valida" & 
    Err \path{400} "Email non valida" & RF2.0 \\
    \hline

    2.2 & Richiesta OTP: Email già usata & 
    \path{email="exist@test.com"} & 
    L'indirizzo email specificato è già associato a un utente attivo nel database \path{UTENTE}. & DB & 
    Err \path{409} "Email già utilizzata" & 
    Err \path{409} "Email già utilizzata" & RF2.0 \\
    \hline

    2.3 & Richiesta OTP: Successo & 
    \path{email="new@test.com"} & 
    L'email è valida e non è presente nel database. Il servizio di posta (o mock) deve essere pronto. & SMTP & 
    Status \path{200} "Codice inviato" & 
    Status \path{200} "Codice inviato" & RF2.0 \\
    \hline
    
    2.4 & Richiesta OTP: DevMode & 
    \path{email="dev@test.com"} & 
    Le credenziali SMTP non sono configurate nell'ambiente (variabili d'ambiente mancanti). & Env & 
    Status \path{200} "Check console" & 
    Status \path{200} "Check console" & Dev \\
    \hline

    3.1 & \path{Registrazione} completamento Cittadino & 
    \path{otp="123456"}, \path{token="VALID"} & 
    È stato generato un OTP valido per l'email associata al token Google e l'OTP non è ancora scaduto. & Redis / Mem & 
    Status \path{201} \newline Utente creato & 
    Status \path{201} \newline Utente creato & RF2.1 \\
    \hline

    3.2 & \path{Registrazione} Admin (via email) & 
    \path{otp="654321"}, \path{email="...admin..."} & 
    L'email contiene la sottostringa "admin" (logica di test) e l'OTP è valido. & Redis / Mem & 
    Status \path{201} \newline Admin creato & 
    Status \path{201} \newline Admin creato & RF2.1 \\
    \hline

    3.3 & \path{Registrazione:} OTP Errato & 
    \path{otp="999999"} & 
    L'OTP fornito non corrisponde a quello salvato temporaneamente per l'email specificata. & Redis / Mem & 
    Err \path{400} "Codice OTP errato" & 
    Err \path{400} "Codice OTP errato" & RF2.0 \\
    \hline

    3.4 & \path{Registrazione:} OTP Scaduto & 
    \path{otp="111111"} & 
    L'OTP salvato ha superato il tempo di validità (es. 5 minuti) ed è considerato scaduto. & Redis / Mem & 
    Err \path{400} "Codice OTP scaduto" & 
    Err \path{400} "Codice OTP scaduto" & RF2.0 \\
    \hline
    
    3.5 & \path{Registrazione:} Token Google Invalido & 
    \path{googleToken="INVALID"} & 
    L'OTP è corretto ma il token Google fornito per la verifica finale non è valido o è scaduto. & Google API & 
    Err \path{401} "Token non valido" & 
    Err \path{401} "Token non valido" & RF2.0 \\
    \hline
    
    3.6 & \path{Registrazione:} Email duplicata & 
    \path{email="dup@test.com"} & 
    Si verifica una race condition dove l'email viene inserita nel DB tra la richiesta OTP e la conferma registrazione. & DB & 
    Err \path{409} "Email già registrata" & 
    Err \path{409} "Email già registrata" & RF2.0 \\
    \hline
    
    4.0 & Logout & 
    - & 
    Nessuna precondizione particolare (stateless), la richiesta deve essere processata correttamente. & - & 
    Status \path{200} "Logout effettuato" & 
    Status \path{200} "Logout effettuato" & RF5 \\
    \hline

    % =========================================================================
    % FILE: initiatives.test.js (Gestione Iniziative)
    % =========================================================================

    5.1 & Creazione iniziativa (Successo) & 
    \path{title}, \path{desc}, \path{place}, \path{catId}, \path{file} & 
    L'utente deve essere autenticato come Cittadino. Tutti i campi obbligatori sono presenti e validi. & DB & 
    Status \path{201} \newline ID Iniziativa & 
    Status \path{201} \newline ID Iniziativa & RF1 \\
    \hline

    5.2 & Creazione iniziativa: Validazione & 
    \path{title} (manca descrizione) & 
    L'utente è autenticato. Il payload manca di uno o più campi obbligatori (es. descrizione). & - & 
    Err \path{400} Bad Request & 
    Err \path{400} Bad Request & - \\
    \hline
    
    5.3 & Creazione iniziativa: Non autenticato & 
    \path{title}, \path{desc} & 
    L'intestazione della richiesta non contiene un token di sessione valido o nessun utente è loggato. & - & 
    Err \path{401} Unauthorized & 
    Err \path{401} Unauthorized & - \\
    \hline

    5.4 & Creazione iniziativa: Cooldown & 
    \path{title}, \path{desc} & 
    L'utente ha già creato un'iniziativa ed è trascorsa meno di una soglia di tempo (14 giorni) da tale creazione. & DB & 
    Err \path{422} "Già creato di recente" & 
    Err \path{422} "Già creato di recente" & RF13 \\
    \hline

    6.1 & Lista iniziative: Paginazione & 
    \path{page=1}, \path{limit=2} & 
    Devono esistere nel database più iniziative di quante richieste per pagina per verificare lo split. & DB & 
    Status \path{200} \newline Array limitato & 
    Status \path{200} \newline Array limitato & - \\
    \hline

    6.2 & Lista iniziative: Filtro Stato & 
    \path{status="Approvata"} & 
    Il database deve contenere iniziative con stati diversi (Approvata, In Corso, Respinta). & DB & 
    Status \path{200} \newline Solo "Approvata" & 
    Status \path{200} \newline Solo "Approvata" & RF3 \\
    \hline
    
    6.3 & Lista iniziative: Filtro Categoria & 
    \path{category=3} & 
    Il database deve contenere iniziative associate a diverse categorie. & DB & 
    Status \path{200} \newline Solo Cat. 3 & 
    Status \path{200} \newline Solo Cat. 3 & RF3 \\
    \hline
    
    6.4 & Lista iniziative: Filtro Firme & 
    \path{minSignatures=100} & 
    Devono esistere iniziative con un numero di firme superiore e inferiore alla soglia. & DB & 
    Status \path{200} \newline Firme >= 100 & 
    Status \path{200} \newline Firme >= 100 & RF3 \\
    \hline

    6.5 & Lista iniziative: Ricerca testo & 
    \path{search="Libro"} & 
    Deve esistere almeno un'iniziativa il cui titolo contenga la parola chiave specificata. & DB & 
    Status \path{200} \newline Include "Libro" & 
    Status \path{200} \newline Include "Libro" & - \\
    \hline
    
    6.6 & Lista iniziative: Ordinamento & 
    \path{sortBy=1}, \path{order="asc"} & 
    Le iniziative devono avere date di creazione diverse per verificare l'ordinamento temporale. & DB & 
    Status \path{200} \newline Ordine cronologico & 
    Status \path{200} \newline Ordine cronologico & - \\
    \hline

    7.1 & Dettaglio Iniziativa: Successo & 
    \path{id=101} & 
    L'ID richiesto deve corrispondere a un'iniziativa esistente nel database. & DB & 
    Status \path{200} \newline Oggetto Iniziativa & 
    Status \path{200} \newline Oggetto Iniziativa & - \\
    \hline

    7.2 & Dettaglio Iniziativa: Non Trovato & 
    \path{id=9999} & 
    L'ID richiesto non deve corrispondere ad alcuna iniziativa presente nel database. & DB & 
    Err \path{404} Not Found & 
    Err \path{404} Not Found & - \\
    \hline

    8.1 & Patch Iniziativa: Admin Estende Data & 
    \path{expirationDate="2099..."} & 
    L'utente deve essere Admin. L'iniziativa deve esistere. La nuova data deve essere futura. & DB & 
    Status \path{200} \newline Data aggiornata & 
    Status \path{200} \newline Data aggiornata & RF7 \\
    \hline

    8.2 & Patch Iniziativa: Cittadino non autorizzato & 
    \path{expirationDate="2099..."} & 
    L'utente è autenticato ma ha ruolo di Cittadino (non Admin). & - & 
    Err \path{403} Forbidden & 
    Err \path{403} Forbidden & RF7 \\
    \hline

    9.1 & Risposta Admin: Approvazione & 
    \path{status="Approvata"}, \path{motivo} & 
    L'utente deve essere Admin. L'iniziativa esiste ed è in attesa di valutazione. & DB & 
    Status \path{201} \newline Stato cambiato & 
    Status \path{201} \newline Stato cambiato & RF7 \\
    \hline
    
    9.2 & Risposta Admin: Cittadino vietato & 
    \path{status="Approvata"} & 
    L'utente è un Cittadino e tenta di chiamare l'endpoint riservato alla moderazione. & - & 
    Err \path{403} Forbidden & 
    Err \path{403} Forbidden & RF7 \\
    \hline
    
    9.3 & Risposta Admin: Stato Invalido & 
    \path{status="Inventato"} & 
    L'utente è Admin ma fornisce una stringa di stato non prevista dall'enum del sistema. & - & 
    Err \path{400} Bad Request & 
    Err \path{400} Bad Request & - \\
    \hline

    10.1 & Firma iniziativa: Successo & 
    \path{id=103} & 
    L'utente è Cittadino e non ha ancora apposto la firma digitale a questa specifica iniziativa. & DB & 
    Status \path{201} "Firma registrata" & 
    Status \path{201} "Firma registrata" & RF11 \\
    \hline

    10.2 & Firma iniziativa: Doppia firma & 
    \path{id=103} & 
    L'utente ha già firmato l'iniziativa in precedenza (record presente in \path{FIRMA_INIZIATIVA}). & DB & 
    Err \path{409} "Già firmato" & 
    Err \path{409} "Già firmato" & RF11 \\
    \hline
    
    10.3 & Firma iniziativa: Anonimo & 
    \path{id=103} & 
    Nessun utente è loggato nella sessione corrente. & - & 
    Err \path{401} Unauthorized & 
    Err \path{401} Unauthorized & RF11 \\
    \hline

    11.1 & Follow iniziativa: Successo & 
    \path{id=104} & 
    L'utente è autenticato e non sta seguendo l'iniziativa (non presente in \path{INIZIATIVA_SALVATA}). & DB & 
    Status \path{200} "Aggiunta preferiti" & 
    Status \path{200} "Aggiunta preferiti" & RF14 \\
    \hline

    11.2 & Follow iniziativa: Idempotenza & 
    \path{id=104} & 
    L'utente sta già seguendo l'iniziativa. Il sistema non deve generare errore ma confermare. & DB & 
    Status \path{200} Success & 
    Status \path{200} Success & - \\
    \hline

    11.3 & Unfollow iniziativa & 
    \path{id=104} & 
    L'utente sta seguendo l'iniziativa e richiede la rimozione dai preferiti. & DB & 
    Status \path{200} "Rimossa seguiti" & 
    Status \path{200} "Rimossa seguiti" & RF14 \\
    \hline

    % =========================================================================
    % FILE: participatory.test.js (Bilancio Partecipativo)
    % =========================================================================

    12.1 & Crea Bilancio: Successo Admin & 
    \path{title}, \path{options[]}, \path{expDate} & 
    L'utente è Admin. Non ci sono altri bilanci attivi. Le opzioni sono tra 2 e 5. & DB & 
    Status \path{201} \newline ID Bilancio & 
    Status \path{201} \newline ID Bilancio & RF8 \\
    \hline

    12.2 & Crea Bilancio: Cittadino & 
    \path{title}, \path{options[]} & 
    L'utente è un Cittadino e non ha i permessi per creare un bilancio partecipativo. & - & 
    Err \path{403} Forbidden & 
    Err \path{403} Forbidden & - \\
    \hline

    12.3 & Crea Bilancio: Durata insufficiente & 
    \path{expDate} = oggi + 5gg & 
    L'utente è Admin, ma la data di scadenza è inferiore al minimo richiesto (14 giorni). & - & 
    Err \path{400}/\path{422} Durata < 14gg & 
    Err \path{400}/\path{422} Durata < 14gg & - \\
    \hline
    
    12.4 & Crea Bilancio: Opzioni insufficienti & 
    \path{options} (lunghezza 1) & 
    L'array delle opzioni contiene un solo elemento. Minimo richiesto è 2. & - & 
    Err \path{400} Bad Request & 
    Err \path{400} Bad Request & - \\
    \hline
    
    12.5 & Crea Bilancio: Troppe Opzioni & 
    \path{options} (lunghezza 6) & 
    L'array delle opzioni contiene più di 5 elementi. Massimo consentito è 5. & - & 
    Err \path{400} Bad Request & 
    Err \path{400} Bad Request & - \\
    \hline

    12.6 & Crea Bilancio: Conflitto Attivo & 
    \path{title}, \path{options[]} & 
    Esiste già un bilancio nel sistema con stato 'Attivo' (data scadenza futura). & DB & 
    Err \path{409} "Bilancio attivo" & 
    Err \path{409} "Bilancio attivo" & - \\
    \hline

    13.1 & \path{Consultazione Bilancio Attivo} & 
    - & 
    Esiste un bilancio con data scadenza futura. Nessun filtro utente applicato. & DB & 
    Status \path{200} \newline Dati + Opzioni & 
    Status \path{200} \newline Dati + Opzioni & - \\
    \hline
    
    13.2 & \path{Consultazione con Voto Utente} & 
    Header \path{X-Mock-User} & 
    L'utente specificato ha già votato per questo bilancio. L'API deve restituire l'ID dell'opzione votata. & DB & 
    Status \path{200} \newline \path{votedOptionId} & 
    Status \path{200} \newline \path{votedOptionId} & - \\
    \hline
    
    13.3 & \path{Consultazione: Nessun Attivo} & 
    - & 
    Non ci sono bilanci attivi nel database al momento della richiesta. & DB & 
    Status \path{200} \newline Dati \path{null} & 
    Status \path{200} \newline Dati \path{null} & - \\
    \hline

    14.1 & Votazione: Cittadino Successo & 
    \path{position=1} & 
    L'utente è Cittadino, il bilancio è attivo e l'utente non ha ancora votato. & DB & 
    Status \path{200} \newline Voto registrato & 
    Status \path{200} \newline Voto registrato & RF15 \\
    \hline

    14.2 & Votazione: Doppia (Conflitto) & 
    \path{position=2} & 
    L'utente ha già registrato un voto per questo stesso bilancio in precedenza. & DB & 
    Err \path{409} "Già votato" & 
    Err \path{409} "Già votato" & RF15 \\
    \hline

    14.3 & Votazione: Admin & 
    \path{position=1} & 
    L'utente è autenticato come Amministratore (non può votare nei bilanci partecipativi). & - & 
    Err \path{403} Solo cittadini & 
    Err \path{403} Solo cittadini & RF15 \\
    \hline
    
    14.4 & Votazione: Bilancio Scaduto & 
    \path{position=1} & 
    Il bilancio esiste ma la sua data di scadenza è passata. & DB & 
    Err \path{403} Scaduto & 
    Err \path{403} Scaduto & - \\
    \hline
    
    14.5 & Votazione: Opzione Inesistente & 
    \path{position=99} & 
    L'ID o la posizione dell'opzione inviata non corrisponde a nessuna opzione del bilancio corrente. & DB & 
    Err \path{400} Non esiste & 
    Err \path{400} Non esiste & - \\
    \hline
    
    14.6 & Votazione: Bilancio non trovato & 
    \path{id=9999} & 
    L'ID del bilancio passato nell'URL non esiste nel database. & DB & 
    Err \path{404} Non trovato & 
    Err \path{404} Non trovato & - \\
    \hline

    15.1 & Archivio Storico: Lista Admin & 
    \path{page=1} & 
    L'utente è Admin. Esistono bilanci scaduti nel database. & DB & 
    Status \path{200} \newline Lista bilanci & 
    Status \path{200} \newline Lista bilanci & RF9 \\
    \hline
    
    15.2 & Archivio Storico: Paginazione & 
    \path{limit=2} & 
    Esistono più bilanci scaduti del limite per pagina impostato. & DB & 
    Status \path{200} \newline Limite rispettato & 
    Status \path{200} \newline Limite rispettato & - \\
    \hline
    
    15.3 & Archivio Storico: Cittadino & 
    - & 
    L'utente è un Cittadino e tenta di accedere all'archivio storico (riservato admin). & - & 
    Err \path{403} Forbidden & 
    Err \path{403} Forbidden & - \\
    \hline
    
    15.4 & Archivio Storico: Conteggio & 
    - & 
    I bilanci restituiti devono includere il campo dei voti totali per ogni opzione. & DB & 
    Status \path{200} \newline \path{votes} presente & 
    Status \path{200} \newline \path{votes} presente & - \\
    \hline

    % =========================================================================
    % FILE: users.test.js (Gestione Utenti e Profilo)
    % =========================================================================

    16.1 & Profilo Utente: Non Autenticato & 
    - & 
    Nessun token di sessione fornito nell'header della richiesta. & - & 
    Err \path{401} Accesso negato & 
    Err \path{401} Accesso negato & - \\
    \hline

    16.2 & Profilo Utente: Successo & 
    - & 
    L'utente è autenticato. Il database contiene i dati anagrafici corretti per l'ID utente. & DB & 
    Status \path{200} \newline JSON Profilo & 
    Status \path{200} \newline JSON Profilo & RF12 \\
    \hline

    16.3 & Iniziative Utente: Manca Relation & 
    - & 
    Richiesta GET senza il parametro query obbligatorio \path{relation}. & - & 
    Err \path{400} Bad Request & 
    Err \path{400} Bad Request & - \\
    \hline

    16.4 & Iniziative Utente: Create & 
    \path{relation=created} & 
    L'utente ha creato almeno un'iniziativa in passato. & DB & 
    Status \path{200} \newline Lista Create & 
    Status \path{200} \newline Lista Create & - \\
    \hline

    16.5 & Iniziative Utente: Firmate & 
    \path{relation=signed} & 
    L'utente ha firmato almeno un'iniziativa creata da altri. & DB & 
    Status \path{200} \newline Lista Firmate & 
    Status \path{200} \newline Lista Firmate & - \\
    \hline

    16.6 & Iniziative Utente: Seguite & 
    \path{relation=followed} & 
    L'utente segue almeno un'iniziativa (preferiti). & DB & 
    Status \path{200} \newline Lista Seguite & 
    Status \path{200} \newline Lista Seguite & - \\
    \hline
    
    16.7 & Iniziative Utente: Paginazione & 
    \path{page=1}, \path{limit=5} & 
    La lista risultante (create/firmate/seguite) contiene elementi paginabili. & DB & 
    Status \path{200} \newline Meta pagination & 
    Status \path{200} \newline Meta pagination & - \\
    \hline

    17.1 & Notifiche: Lista & 
    - & 
    L'utente ha delle notifiche assegnate nel database. & DB & 
    Status \path{200} \newline Lista Notifiche & 
    Status \path{200} \newline Lista Notifiche & RF20 \\
    \hline

    17.2 & Notifiche: Isolamento Utenti & 
    - & 
    Esistono notifiche per altri utenti (es. Admin) nel DB. Queste non devono apparire. & DB & 
    Status \path{200} \newline Solo proprie & 
    Status \path{200} \newline Solo proprie & RF20 \\
    \hline
    
    17.3 & Notifiche: Filtro Non Lette & 
    \path{read=false} & 
    L'utente ha sia notifiche lette che non lette nel database. & DB & 
    Status \path{200} \newline Solo non lette & 
    Status \path{200} \newline Solo non lette & - \\
    \hline

    17.4 & Notifiche: Segna come Letta & 
    \path{id=ID_NOTIF}, \path{isRead=true} & 
    La notifica esiste, appartiene all'utente ed è attualmente non letta. & DB & 
    Status \path{200} \newline DB Aggiornato & 
    Status \path{200} \newline DB Aggiornato & RF20 \\
    \hline
    
    17.5 & Notifiche: Body Invalido & 
    \path{isRead=false} & 
    Il payload della richiesta non è coerente con l'operazione (es. settare false o campo mancante). & - & 
    Status \path{200}/\path{400} & 
    Status \path{200}/\path{400} & - \\
    \hline
    
    17.6 & Notifiche: Non Trovata & 
    \path{id=88888} & 
    L'ID della notifica non esiste nel database. & DB & 
    Err \path{404} Not Found & 
    Err \path{404} Not Found & - \\
    \hline
    
    17.7 & Notifiche: Sicurezza Accesso & 
    \path{id=ID_ADMIN_NOTIF} & 
    La notifica esiste ma appartiene a un altro utente (es. Admin). L'utente corrente non deve poterla modificare. & DB & 
    Err \path{404} (Security) & 
    Err \path{404} (Security) & RF20 \\
    \hline

    18.1 & Gestione Admin: Lista (401) & 
    \path{isAdmin=true} & 
    Utente non autenticato tenta di accedere alla lista utenti. & - & 
    Err \path{401} Unauthorized & 
    Err \path{401} Unauthorized & - \\
    \hline
    
    18.2 & Gestione Admin: Lista (403) & 
    \path{isAdmin=true} & 
    Utente Cittadino (autenticato) tenta di accedere alla lista admin. & - & 
    Err \path{403} Forbidden & 
    Err \path{403} Forbidden & - \\
    \hline
    
    18.3 & Gestione Admin: Lista Successo & 
    \path{isAdmin=true} & 
    Utente Admin richiede la lista. Nel DB sono presenti almeno 2 admin. & DB & 
    Status \path{200} \newline Lista Admin & 
    Status \path{200} \newline Lista Admin & RF10 \\
    \hline
    
    18.4 & Gestione Admin: Filtro CF & 
    \path{fiscalCode="ADMIN2"} & 
    Esiste un utente admin con il codice fiscale parziale specificato. & DB & 
    Status \path{200} \newline Match CF & 
    Status \path{200} \newline Match CF & - \\
    \hline

    19.1 & Modifica Ruolo: Non Auth & 
    \path{isAdmin=true} & 
    Utente non autenticato tenta di promuovere un utente. & - & 
    Err \path{401} Unauthorized & 
    Err \path{401} Unauthorized & - \\
    \hline
    
    19.2 & Modifica Ruolo: Cittadino & 
    \path{isAdmin=true} & 
    Utente Cittadino tenta di promuovere un altro utente. & - & 
    Err \path{403} Forbidden & 
    Err \path{403} Forbidden & - \\
    \hline
    
    19.3 & Modifica Ruolo: Promozione & 
    \path{isAdmin=true} & 
    Utente Admin modifica utente Cittadino. L'utente target esiste. & DB & 
    Status \path{200} \newline Ruolo Aggiornato & 
    Status \path{200} \newline Ruolo Aggiornato & RF10 \\
    \hline
    
    19.4 & Modifica Ruolo: Revoca & 
    \path{isAdmin=false} & 
    Utente Admin modifica un altro Admin per degradarlo a Cittadino. & DB & 
    Status \path{200} \newline Ruolo Aggiornato & 
    Status \path{200} \newline Ruolo Aggiornato & RF10 \\
    \hline
    
    19.5 & Modifica Ruolo: Self-Demotion & 
    \path{isAdmin=false} & 
    Utente Admin tenta di revocare i privilegi a se stesso (vietato per sicurezza). & - & 
    Err \path{400}/\path{403} & 
    Err \path{400}/\path{403} & - \\
    \hline
    
    19.6 & Modifica Ruolo: Body Invalido & 
    \path{isAdmin="string"} & 
    Il valore passato per \path{isAdmin} non è un booleano valido. & - & 
    Err \path{400} Bad Request & 
    Err \path{400} Bad Request & - \\
    \hline
    
    19.7 & Modifica Ruolo: Utente Inesistente & 
    \path{id=99999} & 
    L'ID utente target non esiste nel database. & DB & 
    Err \path{404} Not Found & 
    Err \path{404} Not Found & - \\
    \hline

    20.1 & Ricerca Utente: Per CF & 
    \path{fiscalCode="CITIZEN1..."} & 
    L'utente Admin cerca un utente specifico esistente tramite CF esatto. & DB & 
    Status \path{200} \newline Utente Trovato & 
    Status \path{200} \newline Utente Trovato & - \\
    \hline
    
    20.2 & Ricerca Utente: Parametri Mancanti & 
    - & 
    Richiesta GET senza parametri di ricerca obbligatori (es. \path{fiscalCode} o \path{isAdmin}). & - & 
    Err \path{400} Bad Request & 
    Err \path{400} Bad Request & - \\
    \hline
    
    20.3 & Ricerca Utente: Non Trovato & 
    \path{fiscalCode="NONEXIST"} & 
    Nessun utente corrisponde al criterio di ricerca fornito. & DB & 
    Err \path{404} Not Found & 
    Err \path{404} Not Found & - \\
    \hline

    21.1 & \path{Pre-autorizzazione: Non Admin} & 
    \path{fiscalCode="NEW..."} & 
    Utente Cittadino tenta di pre-autorizzare un nuovo amministratore. & - & 
    Err \path{403} Forbidden & 
    Err \path{403} Forbidden & - \\
    \hline
    
    21.2 & \path{Pre-autorizzazione: Successo} & 
    \path{fiscalCode="PREAUTH..."} & 
    Utente Admin inserisce un nuovo CF non esistente né in UTENTE né in \path{PRE_AUTORIZZATO}. & DB & 
    Status \path{201} \newline Inserito & 
    Status \path{201} \newline Inserito & - \\
    \hline
    
    21.3 & \path{Pre-autorizzazione: Conflitto} & 
    \path{fiscalCode="EXISTING"} & 
    Il CF inserito appartiene già a un utente registrato o è già pre-autorizzato. & DB & 
    Err \path{409} Conflict & 
    Err \path{409} Conflict & - \\
    \hline

    22.1 & Sicurezza: Accesso Risorse Altrui & 
    \path{id=NOTIF_ADMIN} & 
    Un utente Cittadino tenta di accedere tramite API diretta a una risorsa (es. notifica) di un Admin. & DB & 
    Err \path{404} Not Found & 
    Err \path{404} Not Found & - \\
    \hline

\end{xltabular}
\section{FrontEnd}
Di seguito viene illustrata la parte FrontEnd del progetto.
\textbf{Premessa: } le immagini riportate mostrano l'applicazione con il tema "dark", ma è disponibile anche una versione "light", ottenibile premendo l'icona del sole raffigurata nella parte più a destra dell'header.

\subsection{Dettaglio dell'header}
\begin{figure}[H]
    \centering
    \includegraphics[width=0.9\linewidth]{images/0header.jpeg}
    \caption{Header di un cittadino amministratore autenticato}
    \label{fig:screen_header}
\end{figure}
Nella Figura \ref{fig:screen_header} è riportato l'header presente nelle varie pagine dell'applicazione. Come già specificato nel Desing del FrontEnd (si veda il documento D1), il suo aspetto varia in funzione della tipologia d'utente. Nell'esempio riportato, ad esempio, assumiamo che l'utente sia autenticato e che abbia sia i privilegi da cittadino sia quelli da amministratore. Le altre varianti dell'header sono:
\begin{itemize}
    \item Se l’utente non `e autenticato, visualizza solo le sezioni ”Home” e ”Iniziative” nella barra in alto. Inoltre, al posto della campanella delle notifiche, dell’icona del suo profilo, del suo nome e del pulsante di logout visualizza un pulsante con la scritta ”Accedi”.
    \item Se l'utente autenticato non ha il ruolo di amministratore, non visualizza la sezione "Area Admin".
    \item Se l'utente non è un cittadino, non visualizza la sezione "Dashboard" e la campanella delle notifiche.
\end{itemize}
Cliccando su una delle sezioni, l'utente viene reindirizzato alla pagina dedicata. Se l'utente clicca sulla campanella visualizza le notifiche recenti. 

\subsection{Home Page}
\begin{figure}[H]
    \centering
    \includegraphics[width=0.9\linewidth]{images/1home.jpeg}
    \caption{Home Page di Trento Partecipa}
    \label{fig:screen_home}
\end{figure}
La Figura \ref{fig:screen_home} raffigura la home page di \textbf{\texttt{"TRENTO PARTECIPA"}}. Osserviamo che nell'header, oltre alla lista di sezioni in cui l'utente può navigare, c'è pure il logo e il nome dell'applicazione e una barra di ricerca il cui scopo è quello di permettere la ricerca di iniziative tramite parole chiave. \\Nella home sono inoltre presenti il bilancio partecipativo attualmente in corso, il pulsante per creare una nuova iniziativa e una preview della lista di iniziative, ridotta alle iniziative in corso con più firme raccolte.
\newpage
\begin{itemize}
    \item \textbf{Bilancio partecipativo -} Come da specifiche, il bilancio partecipativo riporta la sua data di scadenza, le varie opzioni votabili, il numero di voti totale (in basso) e per opzione (a lato di ogni opzione), associati alla percentuale sul totale. I pulsanti per votare sono disabilitati per gli utenti che non sono cittadini autenticati. 
    \item \textbf{Pulsante per creare un'iniziativa -} Cliccando il tasto, l'utente cittadino viene condotto alla schermata in Figura \ref{fig:screen_nuova_iniziativa}. Qualora un utente senza quel ruolo provi a votare, viene condotto alla schermata di login (Figura \ref{fig:screen_login}).
    \item \textbf{Iniziative più popolari -} Rispecchiano le caratteristiche generali di tutte le iniziative, discusse successivamente in relazione alla Figura \ref{fig:screen_iniziative}.
\end{itemize}


\subsection{Schermata di login}
\begin{figure}[H]
    \centering
    \includegraphics[width=0.5\linewidth]{images/2login.jpeg}
    \caption{Schermata di login}
    \label{fig:screen_login}
\end{figure}
Nella Figura \ref{fig:screen_login} è raffigurata la schermata di login dell'applicazione, a cui l'utente viene condotto quando clicca sul tasto "Accedi" presente nell'header o, in alcuni casi, quando tenta di compiere delle azioni che possono essere compiute solo da utenti autenticati (ad esempio creare una nuova iniziativa). Date le complicanze introdotte dall'integrazione di SPID/CIE nel sistema, ci siamo limitati a implementare l'accesso tramite Google. Se l'utente sta eseguendo il suo primo accesso, veiene condotto alla schermata di creazione di un nuovo profilo (Figura \ref{fig:screen_nuovo_profilo}), altrimenti viene ricondotto alla home ma con i privilegi aggiornati in base al suo ruolo.

\newpage
\subsection{Schermata di creazione di un nuovo profilo}
\begin{figure}[H]
    \centering
    \includegraphics[width=0.5\linewidth]{images/3.1crea_profilo1.jpeg}
\end{figure}
\begin{figure}[H]
    \centering
    \includegraphics[width=0.5\linewidth]{images/3.2crea_profilo2.jpeg}
    \caption{Le due fasi di creazione di un nuovo profilo}
    \label{fig:screen_nuovo_profilo}
\end{figure}
In Figura \ref{fig:screen_nuovo_profilo} sono raffigurate le due fasi di creazione di un nuovo profilo da parte di un utente. L'utente inserisce il proprio recapito email e attende poi di ricevere un codice OTP da inserire per la verifica di sicurezza. Nella prima fase sono presenti, chiaramente, il box per digitare l'email e il pulsante per l'invio del codice OTP. Nella seconda schermata sono presenti un timer che indica il tempo rimasto prima che il codice scada, i box per l'inserimento delle cifre del codice, il pulsante per confermare la propria registrazione e un pulsante "Ho sbagliato email" per tornare alla fase precedente.


\subsection{Schermata di creazione di una nuova iniziativa}
\begin{figure}[H]
    \centering
    \includegraphics[width=0.9\linewidth]{images/4nuova_iniz.jpeg}
    \caption{Schermata di creazione di una nuova iniziativa}
    \label{fig:screen_nuova_iniziativa}
\end{figure}
In Figura \ref{fig:screen_nuova_iniziativa} è raffigurata la schermata da cui l'utente può compilare i campi per creare e pubblicare una nuova iniziativa. Sono presenti i form per digitare il titolo, l'eventuale luogo e la descrizione, selezionare una categoria e allegare eventualmente delle immagini o dei documenti. In basso a destra sono raffigurati i pulsanti per annullare o confermare la pubblicazione.


\subsection{Schermata di visualizzazione delle iniziative}
\begin{figure}[H]
    \centering
    \includegraphics[width=0.9\linewidth]{images/5lista_iniz.jpeg}
    \caption{Schermata di visualizzazione delle iniziative}
    \label{fig:screen_iniziative}
\end{figure}
In Figura \ref{fig:screen_iniziative} è raffigurata la pagina sotto la voce "Iniziative", accessibile da qualsiasi utente. In essa sono presenti le preview delle varie iniziative presenti in piattaforma e un box dove l'utente può applicare dei parametri di ricerca per visualizzare distintamente un certo sottoinsieme di iniziative.

\newpage
\subsubsection*{Dettaglio preview di un'iniziativa}
\begin{figure}[H]
    \centering
    \includegraphics[width=0.6\linewidth]{images/5.2dettaglio_preview_iniz.jpeg}
\end{figure}
Ogni preview di iniziativa riporta un titolo, la categoria, la data di pubblicazione, il numero di firme raccolte ed eventualmente un'immagine e il luogo. Se l'iniziativa è stata importata da una piattaforma esterna, questa viene indicata esplicitamente (si vedano ad esempio le preview presenti nella home in Figura \ref{fig:screen_home}). La preview presenta inoltra una label che ne indica lo stato (ad esempio "In corso") e un pulsante per seguirla, ossia aggiungerla alla dashboard personale. Se un utente non cittadino prova a seguire un'iniziativa, viene condotto alla schermata da cui fare il login (Figura \ref{fig:screen_login}). Cliccando sulla preview, si apre la pagina di dettaglio dell'iniziativa (Figura \ref{fig:screen_dettaglio_iniziativa}).

\subsubsection*{Dettaglio selezione dei criteri di ricerca}
\begin{figure}[H]
    \centering
    \includegraphics[width=0.3\linewidth]{images/5.1dettaglio_filtri.jpeg}
\end{figure}
Oltre alla ricerca per parole chiave, possibile grazie alla barra di ricerca nell'header, l'utente può applicare anche dei criteri di ricerca tramite l'interfaccia in figura: è possibile filtrare le iniziative per stato, piattaforma, categoria, intervallo temporale (riferito alla data di creazione delle iniziative) e ordinarle in base al valore di certi parametri. I filtri possono essere rimossi manualmente o con un apposito pulsante "Azzera filtri".

\subsection{Schermata di visualizzazione della singola iniziativa}
\begin{figure}[H]
    \centering
    \includegraphics[width=0.8\linewidth]{images/6.1singola_iniz.jpeg}
    \caption{Schermata di visualizzazione di una singola iniziativa}
    \label{fig:screen_dettaglio_iniziativa}
\end{figure}
In Figura \ref{fig:screen_dettaglio_iniziativa} è raffigurata la pagina di dettaglio di un'iniziativa. Come da specifiche riporta titolo, stato, categoria, eventuale luogo, data di creazione e di scadenza, autore, descrizione, numero di firme, eventuali immagini ed allegati e, sotto la descrizione, la risposta del Comune, se presente (Figura \ref{fig:screen_risposta_pubblicata}). Sono poi presenti un tasto per copiare il link dell'iniziativa, in caso l'utente volesse condividerla, e il pulsante per firmarla. Se un utente non cittadino tenta di firmare, viene condotto alla schermata di login (Figura \ref{fig:screen_login}). Per tornare alla lista di preview, l'utente può cliccare su "Torna Indietro", in alto a sinistra. Se l'iniziativa è stata importata da una piattaforma esterna, il pulsante per firmare conduce l'utente alla piattaforma di provenienza, ed ha il seguente aspetto:
\begin{figure}[H]
    \centering
    \includegraphics[width=0.2\linewidth]{images/6.2singola_iniz_esterna.jpeg}
\end{figure}

\subsection{Dashboard personale}
\begin{figure}[H]
    \centering
    \includegraphics[width=0.9\linewidth]{images/7.1dashboard_iniziative.jpeg}
\end{figure}
\begin{figure}[H]
    \centering
    \includegraphics[width=0.9\linewidth]{images/7.2_dashboard_notifiche.jpeg}
    \caption{Dashboard personale}
    \label{fig:screen_dashboard}
\end{figure}
In Figura \ref{fig:screen_dashboard} sono raffigurate le schermate della dashboard personale di un cittadino, dalla quale può visualizzare le iniziative da lui create, firmate e seguite, oltre alle notifiche che gli sono giunte (visualizzabili anche cliccando sull'icona della campanella nell'header).

\newpage
\subsection{Area Admin}
\begin{figure}[H]
    \centering
    \includegraphics[width=0.9\linewidth]{images/8areaadmin.jpeg}
    \caption{Area Admin}
    \label{fig:screen_area_admin}
\end{figure}
In Figura \ref{fig:screen_area_admin} è raffigurata l'area admin, accessibile dagli utenti amministratori. L'area admin consiste in un pannello di controllo dal quale l'utente può scegliere che attività svolgere. Selezionando una tra le quattro aree l’utente sarà ricondotto alla pagina dedicata a quell’attività. Le pagine in questione sono descritte di seguito (Figure \ref{fig:screen_monitoraggio}, \ref{fig:screen_creaBP}, \ref{fig:screen_gestione_personale}, \ref{fig:screen_archivioBP}).

\subsection{Schermata di monitoraggio delle iniziative}
\begin{figure}[H]
    \centering
    \includegraphics[width=0.9\linewidth]{images/9monitoraggio_iniz.jpeg}
    \caption{Schermata di monitoraggio delle iniziative}
    \label{fig:screen_monitoraggio}
\end{figure}
In Figure \ref{fig:screen_monitoraggio} è raffigurata la schermata da cui l'utente amministratore può monitorare le iniziative in corso, in attesa di risposta. Ogni iniziativa riporta un timer che indica il tempo rimasto prima che scada. Sono inoltre presenti, per ogni iniziativa, un pulsante per prorogarne la data di scadenza di 60 giorni e un pulsante per visualizzarla e scrivere una risposta (Figura \ref{fig:screen_risposta}).

\subsection{Schermata di compilazione di una risposta del Comune}
\begin{figure}[H]
    \centering
    \includegraphics[width=0.9\linewidth]{images/10.1risposta.jpeg}
    \caption{Schermata di compilazione di una risposta del Comune}
    \label{fig:screen_risposta}
\end{figure}
In Figura \ref{fig:screen_risposta} è raffigurata la schermata da cui l'utente amministratore può compilare una risposta per un'iniziativa. Il box con i campi da compilare appare all'amministratore sotto le informazioni presenti nella pagina di dettaglio dell'iniziativa selezionata. L'amministratore deve selezionare l'esito finale dell'iniziativa ("Approvata" o "Respinta"), scrivere una motivazione ufficiale e allegare eventualmente dei documenti. Tramite un apposito pulsante può poi pubblicarla. Anche qui è presente un pulsante per prorogare la data di scadenza dell'iniziativa. 
\newpage
Una volta pubblicata, la risposta apparirà agli altri utenti nella pagina di dettaglio dell'iniziativa (Figura \ref{fig:screen_dettaglio_iniziativa}), come nella figura sotto:
\begin{figure}[H]
    \centering
    \includegraphics[width=0.9\linewidth]{images/10.2singola_inizi_con_risposta.jpeg}
    \caption{Risposta del Comune all'iniziativa}
    \label{fig:screen_risposta_pubblicata}
\end{figure}

\subsection{Schermata di creazione del bilancio partecipativo}
\begin{figure}[H]
    \centering
    \includegraphics[width=0.9\linewidth]{images/11crea_bp.jpeg}
    \caption{Schermata di creazione del bilancio partecipativo}
    \label{fig:screen_creaBP}
\end{figure}
In Figura \ref{fig:screen_creaBP} è raffigurata la schermata da cui l'utente amministratore può compilare i campi relativi a un nuovo bilancio partecipativo che intende pubblicare: questi campi, digitabili nei form, includono il titolo, la data di scadenza e un certo numero di opzioni compreso tra 3 e 5. Un apposito pulsante permette di aggiungere un'opzione, se possibile. Sono poi presenti dei pulsanti per annullare o confermare l'operazione.

\subsection{Schermata di gestione del personale}
\begin{figure}[H]
    \centering
    \includegraphics[width=0.9\linewidth]{images/12.1lista_amministratori.jpeg}
    \caption{Schermata di gestione del personale}
    \label{fig:screen_gestione_personale}
\end{figure}
In Figura \ref{fig:screen_gestione_personale} è raffigurata la schermata dalla quale l'utente amministratore può visualizzare la lista completa degli utenti amministratori con le loro informazioni (cognome, nome, codice fiscale, email). Una barra di ricerca permette di digitare un codice fiscale per verificarne la presenza nell'elenco. Per ogni utente nella lista, ad esclusione dell'utente stesso, è presente un pulsante "Rimuovi" che serve per revocare i privilegi da amministratore ad un certo utente. Per aggiungere un nuovo amministratore (promuovendo un cittadino o creando una pre-autorizzazione), l'utente può cliccare il pulsante raffigurante il \textbf{"+}, in alto a destra: così facendo apparirà un box (come nella figura sotto) dove l'utente potrà digitare il codice fiscale dell'utente che intende promuovere.
\begin{figure}[H]
    \centering
    \includegraphics[width=0.5\linewidth]{images/12.2ricerca_cf.jpeg}
\end{figure}

\subsection{Archivio dei bilanci partecipativi}
\begin{figure}[H]
    \centering
    \includegraphics[width=0.9\linewidth]{images/13.1archivio_bp.jpeg}
    \caption{Archivio dei bilanci partecipativi}
    \label{fig:screen_archivioBP}
\end{figure}
In Figura \ref{fig:screen_archivioBP} è raffigurata la schermata da cui l'utente amministratore può consultare l'archivio dei bilanci partecipativi conclusi. Ogni bilancio partecipativo riporta titolo, data di scadenza, numero di opzioni e un pulsante per visualizzarne i risultati. Cliccandolo si apre una schermata come la seguente:
\begin{figure}[H]
    \centering
    \includegraphics[width=0.7\linewidth]{images/13.2risultati_bp.jpeg}
    \caption{Risultati di un bilancio partecipativo concluso}
    \label{fig:screen_risultati}
\end{figure}
In Figura \ref{fig:screen_risultati} sono raffigurati i risultati di un bilancio partecipativo concluso: vengono riportati il titolo del sondaggio, la data di scadenza, il numero totale dei voti e le opzioni. A fianco ad ogni opzione sono riportati il numero di voti e la percentuale sul totale, rispecchiata dalla barra verde sotto l'opzione. Per tornare all'archivio, l'utente può cliccare sul pulsante in basso a sinistra.

\section{Deployment}
Un'istanza del backend è ospitata sulla piattaforma render.com ed è disponibile al link \url{https://trentopartecipa.onrender.com}. Il database relazionale invece è attualmente ospitato sulla piattaforma DigitalOcean che grazie all'account education di github abbiamo potuto utilizzare gratuitamente. Il frontend invece è deployato separatamente al link \url{trentopartecipa.me}.

L'utilizzo del piano gratuito (Free Tier) su Render comporta una limitazione prestazionale intrinseca nota come \textbf{Cold Start}: in caso di inattività prolungata (superiore ai 15 minuti), la piattaforma sospende l'istanza per risparmiare risorse. Di conseguenza, la prima richiesta successiva a un periodo di pausa potrebbe richiedere circa 50-60 secondi per il riavvio del servizio; le richieste successive avverranno invece con latenza standard.

Per superare invece le restrizioni funzionali (rete e storage) imposte dalla piattaforma, sono state adottate le seguenti soluzioni architetturali:

\begin{itemize}
    \item \textbf{Migrazione da SMTP a API HTTP (Resend):}
    Poiché Render blocca le porte SMTP in uscita, abbiamo sostituito l'invio tradizionale delle email con la libreria \textbf{Resend}. Questo servizio permette di inviare i codici OTP per il login tramite chiamate API HTTP (REST), aggirando le restrizioni di rete.

    \item \textbf{Verifica del Dominio e Configurazione DNS:}
    Per superare il blocco di sicurezza di Resend (che impedisce l'invio a indirizzi terzi senza verifica), abbiamo registrato il dominio \texttt{trentopartecipa.me} tramite \textit{Namecheap}. 
    Abbiamo quindi configurato i record DNS \textbf{SPF} e \textbf{DKIM} per autorizzare l'invio delle email e i record \textbf{A} per puntare agli indirizzi IP di GitHub Pages, servendo così anche il frontend dal dominio personalizzato.

    \item \textbf{Persistenza dei Media (Cloudinary):}
    Dato che il filesystem di Render è effimero (i file locali vengono persi allo spegnimento dell'istanza), le immagini caricate dagli utenti vengono salvate sul cloud di \textbf{Cloudinary}, memorizzando nel database solo l'URL remoto sicuro.
\end{itemize}

\paragraph{Configurazione CI/CD}
La configurazione CI/CD basata sulle GitHub Actions esegue il deploy del banch main solo quando vengono superati tutti i test con successo. 





\subsection{Accesso docenti}
I docenti potranno testare interamente l'applicazione sia tramite il normale login con google (che permette l'accesso come utente cittadino) sia tramite un accesso hardcoded utilizzando il codice secret. Sarà poi possibile scegliere il tipo di utente nell'applicativo al link: \url{https://trentopartecipa.me/teacher-login}. 

È importante specificare che questo tipo di accesso è a solo scopo di testing, in modo da poter velocizzare la revisione di tutte le funzionalità per utenti che non hanno direttamente accesso al database. 
\begin{center}
    \item \textbf{Secret: } \path{x7K9pLm2Q5nRt8vW4yB1jD6fG3hC0zXs}
\end{center}




\end{document}